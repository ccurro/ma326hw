Let
\[
A = \begin{pmatrix}
1 & 2 & \cdots & n\\
n+1 & n+2 & \cdots & 2n\\
\vdots & \vdots & & \vdots\\
n^2-n + 1 & n^2 -n +2 & \cdots & n^2
\end{pmatrix}
\]
Find the characteristic polynomial of $A.$ {\it Hint:} First prove
that $A$ has rank 2 and that
$\text{span}(\{(1,1,\dotsc,1),(1,2,\dotsc,n)\})$ is
$\mathsf{L}_A$-invariant.
\paragraph{Hint \#2} Show that $\text{span}(\{(1,1,\dotsc,1),(1,2,\dotsc,n)\})$ is
$\mathsf{L}_A$-invariant.
\begin{align}
v_1 = \begin{pmatrix}1\\1\\\vdots\\1\end{pmatrix} & & v_2
= \begin{pmatrix} 1\\2\\\vdots\\n\end{pmatrix}
\end{align}
Let 
\begin{equation}
b = \frac{(n)(n+1)}{2}
\end{equation}
\begin{align}
Av_1 &= \begin{pmatrix}
b\\
n^2 +b\\
2n^2 + b\\
\cdots\\
(n-1)(n^2) + b
\end{pmatrix}
+
\begin{pmatrix}
0\\
n^2\\
2n^2\\
\cdots\\
(n-1)n^2
\end{pmatrix}
+
b\begin{pmatrix}
1\\
1\\
\cdots\\
1
\end{pmatrix}\\
&= n^2(v_2-v_1) + b(v_1)\\
&= (b-n^2)v_1 + n^2(v_2)
\end{align}
\begin{align}
Av_2 &=
\begin{pmatrix}
 \sum\limits_{i=1}^n i^2\\
 \sum\limits_{i=1}^n (n+i)i\\
 \sum\limits_{i=1}^n (2n+i)i\\
 \vdots\\
 \sum\limits_{i=1}^n \left((n)(n-1)+i\right)i
\end{pmatrix}\\
&=v_1\left(\sum\limits_{i=1}^n i^2\right) + \begin{pmatrix}
0\\
 \sum\limits_{i=1}^n ni\\
 \sum\limits_{i=1}^n 2ni\\
\vdots\\
 \sum\limits_{i=1}^n(n-1)(n)i
\end{pmatrix}
\end{align}
\begin{equation}
Av_2 =  \sum\limits_{i=1}^n(i^2)v_1 + nb(v_2-v_1)
\end{equation}
\begin{equation}
\sum\limits_{i=1}^n(i^2) = \frac{2n^3 + 2n^2 +n}{6} = a
\end{equation}
\begin{align}
\implies Av_2 &= av_1 + nb(v_2-v_1)\\
&= (a-nb)v_1 + nb(v_2)
\end{align}
\paragraph{Hint \#2} Show $A$ has rank 2

Suppose $x$ is a column of $A$
\begin{align}
\implies x &= \begin{pmatrix}
0 + k\\
n + k\\
\vdots\\
(n-1)(n)+k
\end{pmatrix}\\
&= kv_1 + n\begin{pmatrix}
0\\
1\\
2\\
\vdots\\
n-1
\end{pmatrix}\\
&= kv_1 + n(v_2-v_1)\\
&= (k-n)v_1 + nv_2
\end{align}
Every column of $A$ is in
$\text{span}(\{(1,1,\dotsc,1),(1,2,\dotsc,n)\})$ 
\begin{gather}
\implies \text{Col}(A) \subseteq
\text{span}(\{(1,1,\dotsc,1),(1,2,\dotsc,n)\})\\
\implies \rank{A} \leq
\dim{\left(\text{span}(\{(1,1,\dotsc,1),(1,2,\dotsc,n)\})\right)}=2\
\end{gather}
$\rank{A}\neq 0$ because $A\neq 0$.

Suppose $\rank{A}=1$.

It follows that every column of $A$ could be expressed as a scalar
multiple of one vector $z \in \mathsf{F}^n.$

Suppose $x,y$ are district columns of $A$ and $c \in F$  
\begin{align}
\implies x = (k_1 -n)v_1 + nv_1 & & y = (k_2 -n)v_1 + nv_2
\end{align}
Suppose $cx= y$.
\begin{gather}
\implies c(k_1-n)v_1 + cnv_2  = (k_2 -n)v_1 + nv_2\\
\implies c(k_1-n) = (k_2 - n) \text{ and } cn =n\\
\implies n(c-1) = 0
\end{gather}
\paragraph{Case 1} $c=1$
\begin{equation}
\implies x = y \;\lightning \text{ Contradiction! } x,y \text{ are distinct}
\end{equation}
\paragraph{Case 2} $n=0$

Impossible.

Therefore $\rank{A} \neq 1$ and it follows that $\rank{A} = 2.$

By the dimension theorem,
\begin{gather}
\rank{\mathsf{L}_A} + \text{Nullity}(\mathsf{L}_A) =
\dim{(\mathsf{L}_A)}\\
\dim{(\mathsf{L}_A)} = n\\
\rank{\mathsf{L}_A} = \rank{A} = 2\\
\implies \text{Nullity}(\mathsf{L}_A) = n -2\\
\implies \text{Nullity}(A) = n-2
\end{gather}
Let the eigenspace of eigenvalue zero be $E_0$
\begin{gather}
E_0 = \text{N}(A = 0 \cdot I_n) = \text{N}(A)\\
\implies \dim{E_0} = \text{Nullity}(A) = n-2
\end{gather}
Let $m$ be the algebraic multiplicity of eigenvalue zero.
\begin{equation}
m \geq \dim{E_0} = n - 2 \text{ by theorem 5.7}
\end{equation}

Suppose $\mathsf{W} = \text{span}(\{(1,1,\dotsc,1),(1,2,\dotsc,n)\})$

Suppose $\alpha = \{(1,1,\dotsc,1),(1,2,\dotsc,n)\}$ 

$\mathsf{W}$ is $\mathsf{L}_A$-invariant. It suffices to calculate the
characteristic polynomial of the restriction of $\mathsf{L}_A$ to
$\mathsf{W}$ because its characteristic polynomial divides the
characteristic polynomial of $\mathsf{L}_A$ by theorem
5.21. $\mathsf{L}_A$ and $A$ posses the same characteristic polynomial
because $\left[\mathsf{L}_A\right]_\gamma = A$ for some ordered basis
$\gamma$ of $\mathsf{F}^n$.
\begin{gather}
Av_1 = (b-n^2)v_1 + (n^2)v_2\\
Av_2 = (a-nb)v_1 + (nb)v_2
\end{gather}
\begin{equation}
\implies \left[\mathsf{L}_{A_\mathsf{W}}\right]_\alpha =
\begin{pmatrix}
b-n^2 & a-nb\\
n^2 & nb
\end{pmatrix}
\end{equation}
\begin{equation}
\implies
\det{\left(\left[\mathsf{L}_{A_\mathsf{W}}\right]_\alpha-tI_2\right)}
= \det{\begin{pmatrix}
b-n^2 -t & a-nb\\
n^2 & nb -t
  \end{pmatrix}
}
\end{equation}
\begin{equation}
\implies t = \frac{1}{2}\left(\frac{n^3+n}{2} \pm
  \frac{n}{2\sqrt{2}}\sqrt{3n^4 +4n^3+6n^2-4n+3}\right)
\end{equation}
It follows that the characteristic polynomial of
$\mathsf{L}_{A_\mathsf{W}}$ is 
\begin{equation}
f_{\mathsf{L}_{A_\mathsf{W}}}(t) = (t_1-t)(t_2-t)
\end{equation}
\begin{gather}
f_{\mathsf{L}_{A_\mathsf{W}}}(t) | f(t) \\
\implies f(t) = g(t)f_{\mathsf{L}_{A_\mathsf{W}}}(t) = g(t)(t_1-t)(t_2-t)
\end{gather}
Since the degree of $g(t)f_{\mathsf{L}_{A_\mathsf{W}}}(t)$ is 2, the
degree of $g(t)$ is $n-2$ It follows that the algebraic multiplicity
$m= n-2.$
\begin{equation}
\implies g(t) = (-t)^{n-2}
\end{equation}
\begin{align}
\implies f(t) &= (t_1-t)(t_2-t)(-t)^{n-2} \\
&= (-1)^n(t)^{n-2}(t_1-t)(t_2-t)
\end{align}
