Prove the corollary to Theorem 3.16: The reduced row echelon form of a
matrix is unique. 
\paragraph{} Suppose $A \in \M{m}{n}{F}$ such that $\rank{A} = r
\leq\text{min}\{m,n\}$

Proof by induction 

Suppose $n=1$ 
\paragraph{Case 1}
\begin{equation}
A = \begin{pmatrix} 0 \\ 0 \\ \vdots \\ 0\end{pmatrix}
\end{equation}
$A$ is in reduced row echelon form and this the unique representation.
\paragraph{Case 2}
\begin{equation}
A = \begin{pmatrix}a_{11}\\a_{21}\\\vdots\\A_{m1}\end{pmatrix} \quad
\text{for some } a_{i1} \neq 0
\end{equation}
\paragraph{} Execute the following sequence of row operations on
$A$. Perform type 1 row operation to move $a_{i1}$ to the first
row. Perform type 3 row operations to eliminate all lower
terms. Perform type 2 row operation to the change the first term in
the column to 1.
\begin{equation}
A \leadsto \begin{pmatrix}1\\0\\0\\\vdots\\0\end{pmatrix}
\end{equation}
The reduced row echelon form of a column must be of this form. It
follows that this matrix is the unique row reduced echelon form.

Suppose true for $1\leq n \leq k$

Suppose $n = k+1$

Suppose $A \in \M{m}{k+1}{F}$. Suppose $A = (A^\prime\,|b)$ such that
$A^\prime \in \M{m}{k}{F}$ and $ b \in \M{m}{1}{F}$

Let $(B^\prime\,|b^\prime)$ be the reduced row echelon form of
$(A^\prime\,|b)$ By HW.3.4.14, $B^\prime$ is in reduced row echelon
form. It follows from $B^\prime \in \M{m}{k}{F}$ that, by the
induction hypothesis, that $B^\prime$ is in the unique row reduced
echelon form of $A^\prime$.

It remains to show that the column $b^\prime$ is unique.

\paragraph{Case 1:} $b \notin \text{col}(A^\prime)$

If $b$ is not in the column space of $A^\prime$, then $b^\prime$ is not in
the column space of $B^\prime$. It follows that $b^\prime$ cannot be
replaced as a linear combination of vectors from a basis for the
column space of $B^\prime$. Column $b^\prime$ is replaced as:
\begin{equation}
\begin{pmatrix}
0\\0\\\vdots\\1\\\vdots\\0\\0
\end{pmatrix}
\end{equation}
Where $b^\prime_{r+1,1}=1$. This is the unique representation. Suppose
$b^\prime$ were
\begin{equation}
\begin{pmatrix}
0\\0\\\vdots\\1\\\vdots\\0\\0
\end{pmatrix}
\end{equation}
where $b^\prime_{r+j,1}=1, j > 1$. $B^\prime$ is in reduced row
echelon form and is of rank $r$ . It follows that all rows after the
$r^{\text{th}}$ are entirely zero. $(B^\prime\,|b^\prime)$ is not in
reduced row echelon form because there is a row of zeros above a row
containing a nonzero value.
\paragraph{Case 2:} $b \in \text{col}(A^\prime)$
\begin{gather}
\implies b^\prime \in \text{col}(B^\prime)\\
b^\prime = \sum\limits_{i=1}^r c_iv_i
\end{gather}
For $c_i \in F$ and $v_i\in \beta$ where $\beta$ is a basis for
$\text{col}(B^\prime)$ such that $\beta$ is a subset of the standard ordered basis
for $\mathsf{F}^m$. Because $\beta$ is linearly independent, the
coefficients of the linear combination are unique.
\begin{gather}
b^\prime \in \text{col}(B^\prime)\\
\implies \rank{B^\prime} = \rank{B^\prime\,|b^\prime}
\end{gather}
It follows that all rows of $(B^\prime\,|b^\prime)$ after the
$r^{\text{th}}$  are entirely zero. Therefore $(B^\prime\,|b^\prime)$.
