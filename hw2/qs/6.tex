$\mathsf{T}\colon\mathsf{M}_{n\times n}(F) \to F$ defined by
$\mathsf{T}(A) = \text{tr}(A)$. Recall (Example 4, Section 1.3) that
\[
\text{tr}(A) = \sum\limits_{i=1}^n A_{ii}
\]
\begin{enumerate}[(a)]
\item Suppose $A,B \in \mathsf{M}_{n\times n}(F)$ and $c\in F$
\begin{align}
\mathsf{T}(cA +B) &= \sum\limits_{i=1}^n(ca_{ii}+b_{ii})\\
&=\sum\limits_{i=1}^nca_{ii}+\sum\limits_{i=1}^nb_{ii}\\
&=c\sum\limits_{i=1}^na_{ii}+\sum\limits_{i=1}^nb_{ii}\\
&= c\mathsf{T}(A) + \mathsf{T}(B)
\end{align}
\item 
\begin{equation}
\text{dim}(\mathsf{M}_{n\times n}(F)) = n^2
\end{equation}
\begin{equation}
R(\mathsf{T}) =\{\mathsf{T}(x) \colon x \in \mathsf{M}_{n\times n}(F)\}
\end{equation}
Claim $R(\mathsf{T}) =F $
\\$R(\mathsf{T})\subseteq F$ by definition of $\mathsf{T}$
\\Suppose $c \in F$, and  $x \in \mathsf{M}_{n\times n}(F)$ such that 
\[
x = \begin{pmatrix}
1& 0 &\cdots & 0\\
0 & \cdots &\cdots& 0\\
\vdots & & &\vdots\\
0 & \cdots & \cdots& 0 
\end{pmatrix}
\]
\begin{align}
\text{tr}(cx) &= c\cdot\text{tr}(x) \\
&= c \cdot 1 \\
&= c
\end{align}
\begin{equation}
\therefore c \in R(\mathsf{T})
\end{equation}
\begin{equation}
\implies \mathsf{T}\;\text{is onto}
\end{equation}
% \newpage{}
\\Claim: $\{1\}$ is a basis for $F$
\begin{equation}
c \cdot 1 = 0
\end{equation}
\begin{equation}
\implies c = 0\;\; \text{(by cancellation law)}
\end{equation}
\begin{equation}
\therefore \{1\}\;\; \text{is linearly independent}
\end{equation}
\begin{equation}
c\cdot 1= c\;\;\text{for}\; c \in F
\end{equation}
\begin{equation}
\text{span}(\{1\}) = F
\end{equation}
\begin{equation}
\therefore \{1\}\;\;\text{is a basis for}\; R(\mathsf{T})
\end{equation}
\begin{equation}
\implies \text{rank}(\mathsf{T}) =1
\end{equation}
\begin{equation}
\implies \text{nullity}(\mathsf{T})=n^2 -1\;\;\text{(by the dimension theorem)}
\end{equation}
\\Claim: Basis $\beta_n$ for $N(\mathsf{T})$ is a modification of a standard
basis for $\mathsf{M}_{n\times n}(F)$ in which each matrix containing
a $1$ in a diagonal entry is replaced with a matrix containing 1 in the
same entry and $-1$ in entry $(n,n)$ and the matrix where all entries
but $(n,n)=1$ are zero are removed from the set.
\\Claim: $\beta_n$ is linearly independent 
\\Suppose $x \in \text{span}(\beta_n)$ such that
$x=a_1u_1+a_2u_1+\cdots+a_{n^2-1}u_{n^2-1}$
\begin{equation}
\begin{pmatrix}
a_{1,1} &\cdots & \cdots& \cdots& a_{1,n}\\
\vdots & a_{2,2}\\
\vdots &       & \ddots &\\
\vdots &       &        & a_{n-1,n-1}\\
a_n    & \cdots     &    \cdots   &   \cdots       & A
\end{pmatrix}
= \begin{pmatrix}
0 & \cdots & \cdots& \cdots& 0\\
\vdots & \ddots & & & \vdots\\
\vdots & & \ddots & & \vdots\\
\vdots & & & \ddots& \vdots\\
0 & \cdots & \cdots& \cdots& 0
\end{pmatrix}_{n\times n}
\end{equation}
Where $A= (-a_{1,1})+(-a_{2,2})+\cdots+(-a_{n-1,n-1})$
\\
Therefore all the entries of the matrix are zero. Furthermore there
only exists the trivial representation. As such by corollary 2 of
theorem 1.10 $\beta_n$ is a basis for $N(\mathsf{T})$.
\\Suppose: \[x =\begin{pmatrix}
1&\cdots&0\\
 \vdots&\ddots&\vdots\\
0&\cdots&-1
\end{pmatrix} \]
\begin{equation}
\text{tr}\begin{pmatrix}
1&\cdots&0\\
 \vdots&\ddots&\vdots\\
0&\cdots&-1
\end{pmatrix}=0
\end{equation}
\begin{equation}
\implies \begin{pmatrix}
1&\cdots&0\\
 \vdots&\ddots&\vdots\\
0&\cdots&-1
\end{pmatrix} \in N(\mathsf{T})
\end{equation}
\begin{equation}
\begin{pmatrix}
1&\cdots&0\\
 \vdots&\ddots&\vdots\\
0&\cdots&-1
\end{pmatrix} \neq 
\begin{pmatrix}
0&\cdots&0\\
 \vdots&\ddots&\vdots\\
0&\cdots&0
\end{pmatrix}
\end{equation}
\begin{equation}
\implies N(\mathsf{T}) \neq \{0\}
\end{equation}
Therefore $\mathsf{T}$ is not one-to-one by theorem 2.4.
\end{enumerate}
