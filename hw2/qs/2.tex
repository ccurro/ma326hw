Determine whether the following sets are linearly dependent or
linearly independent.
\begin{enumerate}
\item[(b)] 
\[
\left\{\begin{pmatrix}
1 & -2\\
-1 & 4
\end{pmatrix}
,
\begin{pmatrix}
-1 & 1\\
2 & -4
\end{pmatrix}\right\} 
\text{in}\;\mathsf{M}_{2\times2}(\mathbb{R})\]

\begin{equation}
a_1\begin{pmatrix}
1 & -2\\
-1 & 4
\end{pmatrix} +a_2
\begin{pmatrix}
-1 & 1\\
2 & -4
\end{pmatrix} =
\begin{pmatrix}
0 & 0\\
0 & 0
\end{pmatrix}
\end{equation}
\begin{align}
a_1 &= -a_2\\
-a_1 &= a_2\\
\implies a_1 &= a_2 = 0
\end{align}
Only the trivial solution exists. The set is linearly independent.


\item[(d)]
\[
\left\{x^3 - , 2x^2 + 4, -2x^3 + 3x^2 + 2x +6\right\} \;\text{in}\;  \mathsf{P}_3\left(\mathbb{R}\right)
\]
\begin{equation}
a_1\left(x^3-x\right) +a_1\left(2x^2 + 4\right) + a_3\left(-2x^3
  +3x^2+2x+6\right) = 09
\end{equation}
\begin{align}
a_1 &= 2a_3\\
2a_2 &= -3a_3\\
4a_2 &= -6a_3
\end{align}
\begin{align}
a_1 &= t\\
a_2 &= -\left(^3/_4\right)t\\
a_3 &= -\left(^1/_2\right)t
\end{align}
The set is linearly dependent.
\newpage{}
\item[(f)]
\[
\left\{\left(1,-1,2\right),\left(1,-2,1\right),\left(1,1,4\right)\right\}
\;\text{in}\; \mathsf{R}^3
\]
\begin{equation}
a\begin{bmatrix}
1\\
-1\\
2
\end{bmatrix}
+ b\begin{bmatrix}
2\\
0\\
1
\end{bmatrix}
+ c\begin{bmatrix}
-1\\
2\\
-1
\end{bmatrix}
=0
\end{equation}

\begin{align}
1 +2b -c = 0\\
-a +2c =0\\
2a+b-c=0
\end{align}
\[
\begin{gmatrix}[p]
1 & 2 &-1 &0\\
-1 & 0 & 2 & 0\\
2 & 1 & -1 & 0
\rowops
\add[1]{0}{1}
\add[-2]{0}{2}
\mult{1}{\cdot \frac{1}{2}}
\add[3]{1}{2}
\mult{2}{\cdot \frac{2}{5}}
\end{gmatrix}
\rightarrow
\begin{pmatrix}
1 & 0 & 0 & 0\\
0 & 1 & 0 & 0\\
0 & 0 & 1 & 0
\end{pmatrix}
\]
Only the trivial solution exists. This set is linearly independent.
\item[(g)]
\[
\left\{
\begin{pmatrix}
1 & 0\\
-2 & 0
\end{pmatrix},
\begin{pmatrix}
0 & -1\\
1 & 1
\end{pmatrix},
\begin{pmatrix}
-1 & 2\\
1 & 0
\end{pmatrix},
\begin{pmatrix}
2 & 1\\
-4 & 4
\end{pmatrix}
\right\}
\;\text{in}\; \mathsf{M}_{2\times2}(\mathbb{R})
\]
\begin{equation}
a\begin{pmatrix}
1 & 0\\
-2 & 1
\end{pmatrix}
+b\begin{pmatrix}
0 & -1\\
1 & 1
\end{pmatrix}
+c\begin{pmatrix}
-1 & 2\\
1 & 0
\end{pmatrix}
+d\begin{pmatrix}
2 & 1 \\
-4 & 4
\end{pmatrix}
=0
\end{equation}
\[
\begin{gmatrix}[p]
1 & 0 & -1 & 2 & 0\\
-2 & 1 & 1 & -4 &0\\
0 & -1 & 2 & 1 & 0\\
1 & 1 & 0 & 4 & 0
\rowops
\add[2]{0}{1}
\add[-1]{0}{3}
\add[1]{1}{2}
\add[-1]{1}{3}
\add[-2]{2}{3}
\end{gmatrix}
\rightarrow
\begin{pmatrix}
1 & 0 & -1 & 2 & 0\\
0 & 1 & -1 & 0 & 0\\
0 & 0 & 1 & 1 & 0\\
0 & 0 & 0 & 0 & 0
\end{pmatrix}
\]
\begin{align}
a = -3t\\
b = -t\\
c = -t\\
d = t
\end{align}
There exists a non-trivial solution, therefore the set is linearly
dependent.
\end{enumerate}
