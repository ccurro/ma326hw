For each linear operator $\mathsf{T}$ on an inner product space
$\mathsf{V},$ determine whether $\mathsf{T}$ is normal, self-adjoint,
or neither. If possible, produce an orthonormal basis of eigenvectors
of $\mathsf{T}$ for $\mathsf{V}$ and list the corresponding
eigenvalues.
\begin{enumerate}
\item $\mathsf{V} = \mathsf{R}^2$ and $\mathsf{T}$ is defined by
  $\mathsf{T}(a,b) = (2a-2b,-2a+5b)$

Suppose $\beta$ is the standard ordered basis for $\mathsf{R}^2$
\begin{gather}
[\mathsf{T}]_\beta = \begin{pmatrix}
2 & -2 \\
-2 & 5
\end{pmatrix}\\
\implies \left([\mathsf{T}]_\beta\right)^* =
\left([\mathsf{T}^*]\right) = \begin{pmatrix}
2 & -2\\
-2 & 5
\end{pmatrix}\\
\implies \mathsf{T} = \mathsf{T}^*
\end{gather}
\begin{gather}
\det{
\begin{pmatrix}
2-\lambda & -2\\
-2 & 5-\lambda
\end{pmatrix}
} = 0\\
\implies (\lambda -6)(\lambda-1) = 0
\end{gather}
\begin{align}
\implies \lambda_1 &= 6\\
\lambda_2 &= 1
\end{align}
\begin{itemize}
\item For $\lambda_1 = 6$
\begin{gather}
[\mathsf{T}]_\beta - 6I_2 = \begin{pmatrix}
-4 & -2\\
-2 & -1 
\end{pmatrix}\\
\begin{pmatrix}
-4 & -2\\
-2 & -1 
\end{pmatrix}
\begin{pmatrix}
x_1\\
x_2
\end{pmatrix}
=
\begin{pmatrix}
  0\\
  0
\end{pmatrix}\\
\implies
\begin{pmatrix}
0 & 0\\
2 & 1
\end{pmatrix}
\begin{pmatrix}
x_1\\
x_2
\end{pmatrix}
=
\begin{pmatrix}
  0\\
  0
\end{pmatrix}\\
\implies x_1 = -\frac{1}{2}x_2\\
\implies E_{\lambda_1} =
\left\{t\begin{pmatrix}\mfrac{-1}{2}\\1\end{pmatrix}\colon t \in \mathbb{R}\right\}
\end{gather}
\item For $\lambda_2 = 1$
\begin{gather}
[\mathsf{T}]_\beta - I_2 = \begin{pmatrix}
1 & -2\\
-2 & 4
\end{pmatrix}\\
\implies \begin{pmatrix}
1 & -2\\
-2 & 4
\end{pmatrix}
\begin{pmatrix}
x_1\\
x_2
\end{pmatrix}
=
\begin{pmatrix}
  0\\
  0
\end{pmatrix}\\
\implies \begin{pmatrix}
1 & -2\\
0 & 0
\end{pmatrix}
\begin{pmatrix}
x_1\\
x_2
\end{pmatrix}
=
\begin{pmatrix}
  0\\
  0
\end{pmatrix}\\
\implies x_1 = 2x_2\\
\implies E_{\lambda_2} =
\left\{t\begin{pmatrix}2\\1\end{pmatrix}\colon t \in \mathbb{R}\right\}
\end{gather}
\end{itemize}
Suppose
\begin{align}
v_1^\prime = (-\frac{1}{2},1) & & v_2^\prime = (2,1)
\end{align}
Let
\begin{align}
v_1 &= v_1^\prime\\
v_2 &= v_2^\prime - \frac{\langle v_2^\prime, v_1\rangle}{\norm{v_1}^2}v_1
\end{align}
\begin{gather}
\langle v_2^\prime,v_1\rangle = 0\\
\implies v_2 = v_2^\prime\\
\norm{v_1}^2 = \frac{5}{4}\\
\implies \norm{v_1} = \frac{\sqrt{5}}{2}\\
\implies o_1 = \frac{1}{\sqrt{5}}(-1,2)\\
\norm{v_2}^2 = 5 \\
\implies \norm{v_2} =5\\
\implies o_2 = \frac{1}{\sqrt{5}}(2,1)
\end{gather}
An orthonormal basis is
\begin{equation}
\gamma = \left\{\frac{1}{\sqrt{5}}(-1,2),\frac{1}{\sqrt{3}}(2,1)\right\}
\end{equation}
The eigenvector $\frac{1}{\sqrt{5}}(-1,2)$ corresponds to the
eigenvalue 6, and the eigenvector $\frac{1}{\sqrt{3}}(2,1)$
corresponds to the eigenvalue 1.

\item $\mathsf{V} = \mathsf{R}^2$ and $\mathsf{T}$ is defined by
  $\mathsf{T}(a,b,c) = (-a+b,5b,4a-2b+5c)$

Suppose $\beta$ is the standard ordered basis of $\mathsf{R}^3$
\begin{gather}
\implies [\mathsf{T}]_\beta = \begin{pmatrix}
-1 & 1 & 0\\
0 & 5 & 0\\
4 & -2 & 5
\end{pmatrix}\\
([\mathsf{T}]_\beta)^* = [\mathsf{T}^*]_\beta = \begin{pmatrix}
-1 & 0 & 4\\
1 & 5 & -2\\
0 & 0 & 5
\end{pmatrix}\\
\implies \mathsf{T}^* \neq \mathsf{T}\\
([\mathsf{T}]_\beta)^*[\mathsf{T}]_\beta \neq ([\mathsf{T}]_\beta)^*
\end{gather}
$\mathsf{T}$ is neither normal nor adjoint.
\item $\mathsf{V}= \mathsf{C}^2$ and $\mathsf{T}$ is defined by
  $\mathsf{T}(a,b) = (2a+ib,a+2b)$

Suppose $\beta$ is the standard ordered basis of $\mathsf{C}^2$

\begin{gather}
\implies [\mathsf{T}]_\beta = \begin{pmatrix}
2 & i\\
1 & 2
\end{pmatrix}\\
([\mathsf{T}]_\beta)^* = [\mathsf{T}^*]_\beta = \begin{pmatrix}
2 & 1\\
-i & 2
\end{pmatrix}\\
\begin{pmatrix}
2 & 1\\
-i & 2
\end{pmatrix}
\begin{pmatrix}
2 & i\\
1 & 2
\end{pmatrix}
=
\begin{pmatrix}
2 & i\\
1 & 2
\end{pmatrix}
\begin{pmatrix}
2 & 1\\
-i & 2
\end{pmatrix}\\
\notag \implies \mathsf{T} \text{ is normal.}
\end{gather}
\begin{gather}
\det{\begin{pmatrix}
2- \lambda & i\\
1 & 2 -\lambda
  \end{pmatrix}
}\\
\implies (2 -\lambda)^2 = i
\end{gather}
\begin{align}
\implies \lambda_1 &= \left(2 + \frac{\sqrt{2}}{2}\right) +
i\left(\frac{\sqrt{2}}{2}\right) \\
\lambda_2 &=  \left(2 - \frac{\sqrt{2}}{2}\right) +
i\left(-\frac{\sqrt{2}}{2}\right)
\end{align}
\begin{itemize}
\item For $\lambda_1 = \left(2 + \frac{\sqrt{2}}{2}\right) +
i\left(\frac{\sqrt{2}}{2}\right)$
\begin{gather}
[\mathsf{T}]_\beta -\lambda_1I_n = \begin{pmatrix}
-\frac{\sqrt{2}}{2}(1+i) & i\\
1 & -\frac{\sqrt{2}}{2}(1+i)
\end{pmatrix}\\
\begin{pmatrix}
-\frac{\sqrt{2}}{2}(1+i) & i\\
1 & -\frac{\sqrt{2}}{2}(1+i)
\end{pmatrix}
\begin{pmatrix}
x_1\\x_2
\end{pmatrix}
=
\begin{pmatrix}
0\\
0
\end{pmatrix}\\
\implies
\begin{pmatrix}
0 & 0\\
\frac{\sqrt{2}}{2}(1+i) & -i
\end{pmatrix}
\begin{pmatrix}
x_1\\x_2
\end{pmatrix}
=
\begin{pmatrix}
0\\
0
\end{pmatrix}\\
\implies x_1 = \frac{\sqrt{2}}{2}(1+i)x_2\\
\implies E_{\lambda_1} =
\left\{t\begin{pmatrix}\frac{\sqrt{2}}{2}(1+i)\\1\end{pmatrix} \colon
  t \in \mathbb{C}\right\}
\end{gather}
\item For $\lambda_2 =  \left(2 - \frac{\sqrt{2}}{2}\right) +
i\left(-\frac{\sqrt{2}}{2}\right)$
\begin{gather}
[\mathsf{T}]_\beta -\lambda_2I_2 = \begin{pmatrix}
\frac{\sqrt{2}}{2}(1+i) & i\\
1 & \frac{\sqrt{2}}{2}(1+i)
\end{pmatrix}\\
\begin{pmatrix}
\frac{\sqrt{2}}{2}(1+i) & i\\
1 & \frac{\sqrt{2}}{2}(1+i)
\end{pmatrix}
\begin{pmatrix}
x_1\\x_2
\end{pmatrix}
=
\begin{pmatrix}
0\\
0
\end{pmatrix}\\
\implies
\begin{pmatrix}
\frac{\sqrt{2}}{2}(1+i) & i\\
0 & 0 
\end{pmatrix}
\begin{pmatrix}
x_1\\x_2
\end{pmatrix}
=
\begin{pmatrix}
0\\
0
\end{pmatrix}\\
\implies x_1 = -\frac{\sqrt{2}}{2}(1+i)x_2\\
\implies E_{\lambda_2} =
\left\{t\begin{pmatrix}-\frac{\sqrt{2}}{2}(1+i)\\1\end{pmatrix} \colon
  t \in \mathbb{C}\right\}
\end{gather}
Suppose
\begin{align}
w_1 = \left(\frac{\sqrt{2}}{2}(1+i),1\right) & & w_2 =\left(-\frac{\sqrt{2}}{2}(1+i),1\right)
\end{align}
Let
\begin{align}
v_1 &= w_1\\
v_2 &=  w_2 - \frac{\langle w_2,v_1\rangle}{\norm{v_1}^2}v_1
\end{align}
\begin{gather}
\langle w_2,v_1 \rangle = 0\\
\implies v_2 = w_2\\
\norm{v_1}^2 =2\\
\implies \norm{v_1} = \sqrt{2}\\
\implies o_1 = \frac{v_1}{\norm{v_1}} = \left(\frac{1}{2}(1+i),\frac{\sqrt{2}}{2}\right)\\
\norm{v_2}^2 =2\\
\implies \norm{v_2} = \sqrt{2}\\
\implies o_2 = \frac{v_2}{\norm{v_2}} = \left(-\frac{1}{2}(1+i),\frac{\sqrt{2}}{2}\right)
\end{gather}
An orthonormal basis is
\begin{equation}
\gamma = \left\{  \left(\frac{1}{2}(1+i),\frac{\sqrt{2}}{2}\right),\left(-\frac{1}{2}(1+i),\frac{\sqrt{2}}{2}\right)\right\}
\end{equation}
The eigenvector $\left(\frac{1}{2}(1+i),\frac{\sqrt{2}}{2}\right)$
corresponds to the eigenvalue of $\left(2+\frac{\sqrt{2}}{2}\right) +
i\left(\frac{\sqrt{2}}{2}\right).$ The eigenvector
$\left(-\frac{1}{2}(1+i),\frac{\sqrt{2}}{2}\right)$ corresponds to the
eigenvalue of $\left(2-\frac{\sqrt{2}}{2}\right) + i\left(-\frac{\sqrt{2}}{2}\right).$
\end{itemize}
\item $\mathsf{V}= \mathsf{P}_2(\mathbb{R})$ and $\mathsf{T}$ is
  defined by $\mathsf{T}(f) = f^\prime,$ where
\[
\langle f,g \rangle = \int\limits_0^1 f(t)g(t) \; \mathrm{d}t
\]

Suppose $\beta$ is the standard ordered basis of
$\mathsf{P}_2(\mathbb{R})$
\begin{gather}
[\mathsf{T}]_\beta = \begin{pmatrix}
0 & 1 & 0\\
0 & 0 & 2\\
0 & 0 & 0
\end{pmatrix}\\
([\mathsf{T}]_\beta)^* = [\mathsf{T}^*]_\beta = \begin{pmatrix} 
0 & 0 & 0\\
1 & 0 & 0\\
0 & 2 & 0
\end{pmatrix}\\
\begin{pmatrix} 
0 & 0 & 0\\
1 & 0 & 0\\
0 & 2 & 0
\end{pmatrix}
\begin{pmatrix}
0 & 1 & 0\\
0 & 0 & 2\\
0 & 0 & 0
\end{pmatrix}
\neq
\begin{pmatrix}
0 & 1 & 0\\
0 & 0 & 2\\
0 & 0 & 0
\end{pmatrix}
\begin{pmatrix} 
0 & 0 & 0\\
1 & 0 & 0\\
0 & 2 & 0
\end{pmatrix}
\end{gather}
It follows that $\mathsf{T}$ is neither self-adjoint nor normal.
\item $\mathsf{V} = \mathsf{M}_{2\times 2}(\mathbb{R})$ and
  $\mathsf{T}$ is defined by $\mathsf{T}(A) = A^t$.

Suppose $\beta$ is the standard ordered basis of $\mathsf{M}_{2\times 
  2}(\mathbb{R})$

\begin{gather}
\implies [\mathsf{T}]_\beta = \begin{pmatrix}
1 & 0 & 0 & 0\\
0 & 0 & 1 & 0\\
0 & 1 & 0 & 0\\
0 & 0 & 0 & 1
\end{pmatrix}\\
([\mathsf{T}]_\beta)^* = [\mathsf{T}^*]_\beta = \begin{pmatrix}
1 & 0 & 0 & 0\\
0 & 0 & 1 & 0\\
0 & 1 & 0 & 0\\
0 & 0 & 0 & 1
\end{pmatrix}\\
\implies \mathsf{T} = \mathsf{T}^*
\end{gather}
\begin{gather}
[\mathsf{T}]_\beta - \lambda I_4 = \begin{pmatrix}
1-\lambda & 0 & 0 & 0\\
0 & -\lambda & 1 & 0\\
0 & 1 & -\lambda & 0\\
0 & 0 & 0 & 1 -\lambda
  \end{pmatrix}\\
\det{
\begin{pmatrix}
1-\lambda & 0 & 0 & 0\\
0 & -\lambda & 1 & 0\\
0 & 1 & -\lambda & 0\\
0 & 0 & 0 & 1 -\lambda
  \end{pmatrix}
} = 0\\
\implies (\lambda -1)(\lambda +1) = 0
\end{gather}
\begin{align}
\implies \lambda_1 &=1\\
\lambda_2 &= -1
\end{align}
\begin{itemize}
\item For $\lambda_1 = 1$
\begin{gather}
\begin{pmatrix}
0 & 0 & 0  & 0\\
0 & -1 & 1 & 0\\
0 & 1 & -1 & 0\\
0 & 0 & 0 & 0
\end{pmatrix}
\begin{pmatrix}
x_1\\x_2\\x_3\\x_4
\end{pmatrix}
=
\begin{pmatrix}
0\\0\\0\\0
\end{pmatrix}\\
\implies x_2 = x_3\\
\implies E_{\lambda_1} =
\left\{t\begin{pmatrix}0\\1\\1\\0\end{pmatrix} +
  s \begin{pmatrix}1\\0\\0\\0\end{pmatrix} +
  r\begin{pmatrix}0\\0\\0\\1\end{pmatrix}\colon t,s,r \in \mathbb{R}\right\}
\end{gather}
\item For $\lambda_2 = -1$
\begin{gather}
\begin{pmatrix}
2 & 0 & 0 & 0\\
0 & 1 & 1 & 0\\
0 & 1 & 1 & 0\\
0 & 0 & 0 & 2
\end{pmatrix}
\begin{pmatrix}
x_1\\x_2\\x_3\\x_4
\end{pmatrix}
=
\begin{pmatrix}
0\\0\\0\\0
\end{pmatrix}\\
\implies x_1 = x_4 =0\\
x_2 = -x_3\\
\implies E_{\lambda_2} =
\left\{t\begin{pmatrix}0\\-1\\1\\0\end{pmatrix}\colon t \in \mathbb{R}\right\}
\end{gather}
Suppose
\begin{align}
w_1 = \begin{pmatrix} 0 & 1\\1 & 0\end{pmatrix} & & w_2
= \begin{pmatrix}1 & 0 \\0 & 0 \end{pmatrix} \\
w_3 = \begin{pmatrix} 0 & 0 \\ 0 & 1\end{pmatrix} & &
w_4 =\begin{pmatrix} 0 & -1\\ 1 & 0\end{pmatrix}
\end{align}
Let
\begin{align}
v_1 &= w_1\\
v_2 &= w_2 - \frac{\langle w_2,v_2\rangle}{\norm{v_1}^2}v_1\\
v_3 &= w_3 - \left(\frac{\langle w_3,v_2\rangle}{\norm{v_2}^2}v_2 +
  \frac{\langle w_3,v_1\rangle}{\norm{v_1}^2}v_1\right)\\
v_4 &= w_4 -\left(\sum\limits_{i=1}^3\frac{\langle w_4,v_i\rangle}{\norm{v_1}^2}v_i\right)
\end{align}
\begin{gather}
\langle w_2,v_1\rangle = 0\\
\implies v_2 = w_2\\
\langle w_3,v_1\rangle = 0\\
\langle w_3,v_2\rangle = 0\\
\implies v_3 = w_3\\
\langle w_4,v_1\rangle = 0\\
\langle w_4,v_2\rangle = 0\\
\langle w_4,v_3\rangle = 0\\
\implies v_4 = w_4\\
\norm{v_1}^2 = 2\\
\implies \norm{v_1} = \sqrt{2}\\
\implies o_1 = \begin{pmatrix}
0 & \frac{1}{\sqrt{2}}\\
\frac{1}{\sqrt{2}} & 0
\end{pmatrix}\\
\norm{v_2}^2 = 1\\
\implies \norm{v_2} = 1\\
\implies o_2 = v_2\\
\norm{v_3}^2 = 1\\
\implies \norm{v_3} = 1\\
\implies o_3 = v_3\\
\norm{v_4}^2 =2 \\
\implies \norm{v_4} = \sqrt{2}\\
\implies o_4 = \begin{pmatrix}
0 & \frac{-1}{\sqrt{2}}\\
\frac{1}{\sqrt{2}} & 0
\end{pmatrix}
\end{gather}
An orthonormal basis is
\begin{equation}
\gamma = \left\{
\begin{pmatrix}
0 & \frac{1}{\sqrt{2}}\\
\frac{1}{\sqrt{2}} & 0
\end{pmatrix},
\begin{pmatrix}
1 & 0 \\
0 & 0 
\end{pmatrix}
,
\begin{pmatrix}
0 & 0 \\
0 & 1
\end{pmatrix}
,
\begin{pmatrix}
0 & \frac{-1}{\sqrt{2}}\\
\frac{1}{\sqrt{2}} & 0
\end{pmatrix}
\right\}
\end{equation}
\end{itemize}
\item $\mathsf{V} = \mathsf{M}_{2\times 2}(\mathbb{R})$ and
  $\mathsf{T}$ is defined by $\mathsf{T}\left(\begin{smallmatrix} a &
      b\\ c & d\end{smallmatrix}\right) = \left(\begin{smallmatrix} c
      & d \\ a & b \end{smallmatrix}\right)$ 

Suppose $\beta$ is the standard ordered basis of $\mathsf{M}_{2 \times
  2}(\mathbb{R})$
\begin{gather}
\implies [\mathsf{T}]_\beta = \begin{pmatrix}
0 & 0 & 1 & 0\\
0 & 0 & 0 & 1\\
1 & 0 & 0 & 0\\
0 & 1 & 0 & 0
\end{pmatrix}\\
([\mathsf{T}]_\beta)^* = [\mathsf{T}^*]_\beta = \begin{pmatrix}
0 & 0 & 1 & 0\\
0 & 0 & 0 & 1\\
1 & 0 & 0 & 0\\
0 & 1 & 0 & 0
\end{pmatrix}\\
\implies \mathsf{T} \text{ is self adjoint.} \notag
\end{gather}
\begin{gather}
[\mathsf{T}]_\beta - \lambda I_4 = \begin{pmatrix}
-\lambda & 0 & 1 & 0\\
0 & -\lambda & 0 & 1\\
1 & 0 & -\lambda & 0\\
0 & 1 & 0 & -\lambda
\end{pmatrix}\\
\det{
\begin{pmatrix}
-\lambda & 0 & 1 & 0\\
0 & -\lambda & 0 & 1\\
1 & 0 & -\lambda & 0\\
0 & 1 & 0 & -\lambda
\end{pmatrix}
}=0\\
(\lambda-1)^2(\lambda +1)
\end{gather}
\begin{align}
\implies \lambda_1 &= 1\\
\lambda_2 &= -1
\end{align}
\begin{itemize}
\item For $\lambda_1 = 1$
\begin{gather}
\begin{pmatrix}
-1 & 0 & 1 & 0\\
0 & -1 & 0 & 1\\
1 & 0 & -1 & 0\\
0 & 1 & 0 & -1
\end{pmatrix}
\begin{pmatrix}
x_1\\x_2\\x_3\\x_4
\end{pmatrix}
=
\begin{pmatrix}
0\\0\\0\\0
\end{pmatrix}\\
\implies
\begin{pmatrix}
0 & 0 & 0 & 0\\
0 & 0 & 0 & 0\\
1 & 0 & -1 & 0\\
0 & 1 & 0 & -1
\end{pmatrix}
\begin{pmatrix}
x_1\\x_2\\x_3\\x_4
\end{pmatrix}
=
\begin{pmatrix}
0\\0\\0\\0
\end{pmatrix}\\
\implies E_{\lambda_1} = \left\{
t\begin{pmatrix}1\\0\\1\\0\end{pmatrix} +
s \begin{pmatrix}0\\1\\0\\1\end{pmatrix} \colon t,s \in \mathbb{R}
\right\}\\
\end{gather}
\item For $\lambda_2 = -1$
\begin{gather}
\begin{pmatrix}
1 & 0 & 1 & 0\\
0 & 1 & 0 & 1\\
1 & 0 & 1 & 0\\
0 & 1 & 0 & 1
\end{pmatrix}
\begin{pmatrix}
x_1\\x_2\\x_3\\x_4
\end{pmatrix}
=
\begin{pmatrix}
0\\0\\0\\0
\end{pmatrix}\\
\begin{pmatrix}
1 & 0 & 1 & 0\\
0 & 1 & 0 & 1\\
0 & 0 & 0 & 0\\
0 & 0 & 0 & 0
\end{pmatrix}
\begin{pmatrix}
x_1\\x_2\\x_3\\x_4
\end{pmatrix}
=
\begin{pmatrix}
0\\0\\0\\0
\end{pmatrix}\\
\implies x_1 = -x_3\\
x_2 = -x_4\\
\implies E_{\lambda_2} = \left\{
t\begin{pmatrix}-1\\0\\1\\0\end{pmatrix} +
s\begin{pmatrix}0\\-1\\0\\1\end{pmatrix}\colon s,t \in \mathbb{R}
\right\}
\end{gather}
\end{itemize}
Suppose
\begin{align}
w_1 = \begin{pmatrix}1&0\\1&0\end{pmatrix} & & w_2 = \begin{pmatrix} 0 & 1
  \\ 0 & 1\end{pmatrix} \\
w_3 = \begin{pmatrix}-1&0\\1&0\end{pmatrix} & & w_4 = \begin{pmatrix}0&-1\\0&1\end{pmatrix}
\end{align}
Let
\begin{align}
v_1 &= w_1\\
v_2 &= w_2 - \frac{\langle w_2,v_2\rangle}{\norm{v_1}^2}v_1\\
v_3 &= w_3 - \left(\frac{\langle w_3,v_2\rangle}{\norm{v_2}^2}v_2 +
  \frac{\langle w_3,v_1\rangle}{\norm{v_1}^2}v_1\right)\\
v_4 &=  w_4 -\left(\sum\limits_{i=1}^3\frac{\langle w_4,v_i\rangle}{\norm{v_1}^2}v_i\right)
\end{align}
\begin{gather}
\langle w_2,v_a \rangle = 0\\
\implies v_2 = w_2\\
\langle w_3,v_2\rangle = 0\\
\langle w_3,v_1\rangle = 0\\
\implies v_3 = w_3\\
\langle w_4,v_1 \rangle = 0\\
\langle w_4,v_2\rangle = 0\\
\langle w_4,v_3\rangle = 0\\
\implies v_4 = w_4
\end{gather}
\begin{gather}
\norm{v_1}^2 = 2\\
\implies \norm{v_1} = \sqrt{2}\\
\norm{v_2}^2 = 2\\
\implies \norm{v_2} = \sqrt{2}\\
\norm{v_3}^2 = 2\\
\implies \norm{v_3} = \sqrt{2}\\
\norm{v_4}^2 = 2 \\
\implies \norm{v_4} = \sqrt{2}
\end{gather}
An orthonormal basis is
\begin{equation}
\gamma =\left\{
\frac{1}{\sqrt{2}}
\begin{pmatrix}1&0\\1&0\end{pmatrix},
\frac{1}{\sqrt{2}}
\begin{pmatrix} 0 & 1\\ 0 & 1\end{pmatrix}
\frac{1}{\sqrt{2}}
\begin{pmatrix}-1&0\\1&0\end{pmatrix},
\frac{1}{\sqrt{2}}
\begin{pmatrix}0&-1\\0&1\end{pmatrix}
\right\}
\end{equation}
\end{enumerate}
