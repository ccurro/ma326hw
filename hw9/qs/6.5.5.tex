Which of the following pairs of matrices are unitarily equivalent?
\begin{enumerate}
\item $\begin{pmatrix} 1& 0\\0 & 1\end{pmatrix}$ and $\begin{pmatrix}0
    &1\\1 & 0\end{pmatrix}$
\begin{equation}
\det{\begin{pmatrix} 1-\lambda& 0\\0 & 1-\lambda\end{pmatrix}} = (1-\lambda)^2
\end{equation}
\begin{equation}
\det{\begin{pmatrix}-\lambda & 1\\1 & -\lambda\end{pmatrix}} =
\lambda^2 -1
\end{equation}
Not unitarily equivalent  because they have different eigenvalues.
\item $\begin{pmatrix}0 & 1\\1 & 0 \end{pmatrix}$ and
  $\begin{pmatrix}0 & \mfrac{1}{2}\\ \mfrac{1}{2} & 0 \end{pmatrix}$
\begin{equation}
\det{
\begin{pmatrix}
-\lambda & 1\\
1 & -\lambda
\end{pmatrix}
} = (\lambda-1)(\lambda +1)
\end{equation}
\begin{equation}
\det{
\begin{pmatrix}
-\lambda & \mfrac{1}{2}\\
\mfrac{1}{2} & -\lambda
\end{pmatrix}} = \lambda^2 - \frac{1}{4}
\end{equation}
Not unitarily equivalent  because they have different eigenvalues.
\item$\begin{pmatrix} 0 & 1 & 0\\-1 & 0  & 0\\0 & 0 & 1\end{pmatrix}$
  and $\begin{pmatrix} 2 & 0 & 0\\0 & -1 & 0\\ 0 & 0 & 0\end{pmatrix}$
\begin{equation}
\det{
\begin{pmatrix}
-\lambda & 1 & 0\\
-1 & -\lambda & 0\\
0 & 0 & 1-\lambda
\end{pmatrix}
} = (1-\lambda)(\lambda+i)(\lambda-i)
\end{equation}
\begin{equation}
\det{
\begin{pmatrix}
2-\lambda & 0 & 0\\
0 & -1 -\lambda & 0\\
0 & 0 & -\lambda 
\end{pmatrix}
} = (2-\lambda)(-1-\lambda)(-\lambda)
\end{equation}
Not unitarily equivalent  because they have different eigenvalues.
\item $\begin{pmatrix}
0 & 1 & 0\\
-1 & 0 & 0\\
0 & 0 & 1
\end{pmatrix}$ and
$\begin{pmatrix}
1 & 0 & 0\\
0 & i & 0\\
0 & 0 & -i
\end{pmatrix}$
\begin{gather}
A -I_3 = \begin{pmatrix}
-1 & 1 & 0\\
-1 & -1 & 0\\
0 & 0 & -1
\end{pmatrix}\\
\begin{pmatrix}
-1 & 1 & 0\\
-1 & -1 & 0\\
0 & 0 & -1
\end{pmatrix}
\begin{pmatrix}
x_1\\x_2\\x_3
\end{pmatrix}
=
\begin{pmatrix}
0\\0\\0
\end{pmatrix}\\
\begin{pmatrix}
0 & 2 & 0\\
-1 & -1 & 0\\
0 & 0 & 0
\end{pmatrix}
\begin{pmatrix}
x_1\\x_2\\x_3
\end{pmatrix}
=
\begin{pmatrix}
0\\0\\0
\end{pmatrix}\\
\implies x_2 = 0\\
x_1 = -x_2 = 0\\
\implies E_{\lambda_1} = \left\{
t\begin{pmatrix}0\\0\\1\end{pmatrix} \colon t \in \mathbb{C} 
\right\}
\end{gather}
\begin{gather}
A - i I_3 =\begin{pmatrix}
-i & 1 & 0\\
-1 & -i & 0\\
0 & 0 & 1-i
\end{pmatrix}\\
\begin{pmatrix}
-i & 1 & 0\\
-1 & -i & 0\\
0 & 0 & 1-i
\end{pmatrix}
\begin{pmatrix}
x_1\\x_2\\x_3
\end{pmatrix}
=
\begin{pmatrix}
0\\0\\0
\end{pmatrix}\\
\begin{pmatrix}
-i & 1 & 0\\
-i & 1 & 0\\
0 & 0 & 1
\end{pmatrix}
\begin{pmatrix}
x_1\\x_2\\x_3
\end{pmatrix}
=
\begin{pmatrix}
0\\0\\0
\end{pmatrix}\\
\implies x_3 = 0\\
-ix_1 = -x_2\\
\implies x_1 = -ix_2\\
\implies E_{\lambda_2} = \left\{
t\begin{pmatrix}-i\\1\\0\end{pmatrix} \colon t \in \mathbb{C}
\right\}
\end{gather}
\begin{gather}
A + i I_3 = \begin{pmatrix}
i & 1 & 0\\
-1 & i & 0\\
0 & 0 & i
\end{pmatrix}\\
\begin{pmatrix}
i & 1 & 0\\
-1 & i & 0\\
0 & 0 & i
\end{pmatrix}
\begin{pmatrix}
x_1\\x_2\\x_3
\end{pmatrix}
=
\begin{pmatrix}
0\\0\\0
\end{pmatrix}\\
\begin{pmatrix}
i & 1 & 0\\
-i & -1 & 0\\
0 & 0 & 1+i
\end{pmatrix}
\begin{pmatrix}
x_1\\x_2\\x_3
\end{pmatrix}
=
\begin{pmatrix}
0\\0\\0
\end{pmatrix}\\
\implies ix_1 = x_2\\
x_3 = 0\\
\implies x_1 = ix_2\\
\implies E_{\lambda_3} = \left\{
t \begin{pmatrix}3\\1\\0\end{pmatrix}\colon t \in \mathbb{C}
\right\}
\end{gather}
Suppose
\begin{align}
w_1 = \begin{pmatrix}0\\0\\1\end{pmatrix} & & w_2
= \begin{pmatrix}-i\\1\\0\end{pmatrix} & & w_3 = \begin{pmatrix} i\\1\\0\end{pmatrix}
\end{align}
Let
\begin{align}
v_1 &= w_1\\
v_2 &= w_2 - \frac{\langle w_2,v_2\rangle}{\norm{v_1}^2}v_1\\
v_3 &= w_3 - \left(\frac{\langle w_3,v_2\rangle}{\norm{v_2}^2}v_2 +
  \frac{\langle w_3,v_1\rangle}{\norm{v_1}^2}v_1\right)\\
\end{align}
\begin{gather}
\langle w_2,v_1 \rangle = 0\\
\implies v_2 = w_2 \\
\langle w_3,v_1 \rangle = 0\\
\langle w_3,v_2 \rangle = 0\\
\implies v_3 = w_3\\
\norm{v_1}^2 = 1\\
\implies \norm{v_1} =1\\
\implies o_1 = v_1\\
\norm{v_2}^2 = 2\\
\norm{v_2} = \sqrt{2}\\
\implies o_2 = \begin{pmatrix}
  \frac{-i}{\sqrt{2}}\\\frac{1}{\sqrt{2}}\\0\end{pmatrix}\\
\norm{v_3}^2 = 2\\
\implies \norm{v_3} = \sqrt{2}\\
\implies o_3 = \begin{pmatrix}\frac{i}{\sqrt{2}}\\\frac{1}{\sqrt{2}}\\0\end{pmatrix}
\end{gather}
\begin{gather}
\implies P_1 = \begin{pmatrix}
0 & \frac{-i}{\sqrt{2}} & \frac{i}{\sqrt{2}}\\
0 & \frac{1}{\sqrt{2}} & \frac{1}{\sqrt{2}}\\
1 & 0 & 0
\end{pmatrix}\\
P_2 = \begin{pmatrix}
1 & 0 & 0\\
0 & 1 & 0\\
0 & 0 & 1
\end{pmatrix}\\
%P_1^* = \begin{pmatrix}
%0 & 0 & 1\
%\frac{i}{\sqrt{2}} & \frac{1}{\sqrt{2}} & 0\\
%\frac{-i}{\sqrt{2}} & \frac{1}{\sqrt{2}} & 0
%\end{pmatrix}\\
%\implies P_1^*AP_2 = B
\implies P_1^*AP_1 = P_2^*BP_2
\end{gather}
It follows that $A$ and $B$ are unitarily equivalent.
\item $\begin{pmatrix}
1 & 1 & 0\\
0 & 2 & 2\\
0 & 0 & 3
  \end{pmatrix}$
and 
$
\begin{pmatrix}
1 & 0 & 0\\
0 & 2 & 0\\
0 & 0 & 3
\end{pmatrix}$

The two matrices have the same eigenvalues however since the former
matrix is asymmetric while the latter is symmetric they cannot be
orthogonally equivalent.
\end{enumerate}
