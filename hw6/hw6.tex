\documentclass[letterpaper,12pt]{article}
\usepackage{setspace}
\usepackage[utf8]{inputenc}
\usepackage[english]{babel}
\usepackage{enumerate}
\usepackage{graphicx}
\usepackage{amsmath}
\usepackage{amssymb}
\usepackage{amsfonts}
\usepackage{stmaryrd}
\usepackage{booktabs}
\usepackage{gauss}
\usepackage{mathtools}
\usepackage{multirow}
\usepackage{url}
\usepackage{siunitx}
\usepackage{microtype}
\usepackage{graphicx}
\usepackage{appendix}
\usepackage{verbatim}
\pagenumbering{Roman}
% Extra Commands
\newcommand{\tab}{\hspace*{5em}}
\newcommand{\gap}{\vspace*{0.25cm}}
\renewcommand{\implies}{\Rightarrow}
\newcommand{\T}{\mathsf{T}}
\newcommand{\U}{\mathsf{U}}
\newcommand{\V}{\mathsf{V}}
\newcommand{\W}{\mathsf{W}}
\newcommand{\tr}[1]{\text{tr}({#1})}
\newcommand{\rank}[1]{\text{rank}({#1})}
\newcommand{\dime}[1]{\text{dim}({#1})}
\newcommand{\dete}[1]{\text{det}({#1})}
\newcommand{\mfrac}[2]{^{#1}/_{#2}}
\newcommand{\M}[3]{\mathsf{M}_{#1 \times #2}(#3)}
\newmatrix{(}{|}{left}
\newmatrix{.}{)}{right}
\newcommand\bigzero{\makebox(0,0){\text{\huge0}}}
% Formatting
\usepackage[top=1in, bottom=1in, left=1.2in, right=1.2in]{geometry}
\usepackage{fancyhdr}

% Heading stuffs
% \pagestyle{plain}
\setlength{\headheight}{28pt}
\thispagestyle{fancy}
\fancyhf{}
\lhead{Chris Curro, John Biswakarma, Victor Chen \\ November 16, 2012} 
\chead{\hfill \\ MA326}
\rhead[RE,RO]{HW \#7 \\ Prof. Mintchev}
\cfoot[RO,RE]{\thepage}  %Page # for footer
% This should return the format
% Name         Course #        Assignment #
% Date         Course name     Prof. 
% -------------------------------------------
\begin{document}
%\begin{spacing}{1.2}
\section*{Assignment}
Section 4.4: 1, 5, 6; Section 5.1: 3(bc), 4(ceh), 7, 12, 14, 15, 19, 22
\section*{Work}
\subsection*{4.4}
\begin{enumerate}
\item
Label the statements as true or false.
\begin{center}
\begin{tabular}{r l r l}
\toprule
(a) & True & (g) & True \\
(b) & True & (h) & False \\
(c) & True & (i) & True \\
(d) & False & (j) & True  \\
(e) & False & (k) & True \\
(f) & True & \\
\bottomrule
\end{tabular}
\end{center}

\setcounter{enumi}{4}
\item
Let $A$ be invertible. Prove that $A^t$ is invertible and $(A^t)^{-1}
= (A^{-1})^t$.

\item
For each matrix $A$ and ordered basis $\beta$, find
$[\mathsf{L}_A]_\beta$. Also find an invertible matrix $Q$ such that
$[\mathsf{L}_A]_\beta = Q^{-1}AQ$.
\begin{enumerate}[(a)]
\item[(b)] $A = \begin{pmatrix}1&2\\2&1
  \end{pmatrix}$ and $\beta =\left\{\begin{pmatrix}1\\1\
    \end{pmatrix},\begin{pmatrix}1\\-1
    \end{pmatrix}
  \right\}$
\begin{align}
v_1 =\begin{pmatrix} 1\\ 1 \end{pmatrix} & & v_2 =\begin{pmatrix} 1 \\
  -1 \end{pmatrix}
\end{align}
\begin{gather}
\mathsf{L}_{A}(v_1) = \begin{pmatrix} 1 & 2\\ 2 &
  1\end{pmatrix}\begin{pmatrix}1\\1\end{pmatrix}
= \begin{pmatrix}3\\3\end{pmatrix} = 3\begin{pmatrix}1\\1\end{pmatrix}
+ 0\begin{pmatrix}1\\-1\end{pmatrix}\\
\implies \left[\mathsf{L}_A(v_1)\right]_\beta = \begin{pmatrix}0\\3\end{pmatrix}
\end{gather}
\begin{gather}
\mathsf{L}_A(v_2) =\begin{pmatrix}1 & 2\\2
  &1\end{pmatrix}\begin{pmatrix}1\\-1\end{pmatrix}
= \begin{pmatrix}-1\\1\end{pmatrix} =
0\begin{pmatrix}1\\1\end{pmatrix} +
-1\begin{pmatrix}1\\-1\end{pmatrix}\\
\implies \left[\mathsf{L}_A(v_1)\right]_\beta = \begin{pmatrix}0\\-1\end{pmatrix}
\end{gather}
\begin{equation}
\implies [\mathsf{L}_A] = \begin{pmatrix}3&0\\0 & -1\end{pmatrix}
\end{equation}
\begin{equation}
Q = \begin{pmatrix}1 & 1 \\ 1 & -1\end{pmatrix}
\end{equation}
\begin{equation}
\begin{gmatrix}[left]
1 & 1&\\
1 & -1 &
\end{gmatrix}
\begin{gmatrix}[right]
1 & 0\\
0 & 1
\rowops
\add[-1]{0}{1}
\mult{1}{\cdot -\frac{1}{2}}
\add[-1]{1}{0}
\end{gmatrix}
\rightarrow
\begin{gmatrix}[left]
1 & 0&\\
0 & 1&
\end{gmatrix}
\begin{gmatrix}[right]
^1/_2 & ^1/_2\\
^1/_2  & ^{-1}/_2
\end{gmatrix}
\end{equation}
\begin{equation}
\implies Q^{-1} = \begin{pmatrix}
^1/_2 & ^1/_2\\
^1/_2  & ^{-1}/_2
\end{pmatrix}
\end{equation}
\item[(c)] $A = \begin{pmatrix}1&1&-1\\2&0&1\\1&1&0
    \end{pmatrix}$
    and 
$\beta = \left\{\begin{pmatrix}1\\1\\1
    \end{pmatrix}, \begin{pmatrix}1\\0\\1
    \end{pmatrix},\begin{pmatrix}1\\1\\2
    \end{pmatrix}\right\}$
\begin{align}
v_1 = \begin{pmatrix}1\\1\\1\end{pmatrix} & & v_2 = \begin{pmatrix}1
  \\ 0 \\1\end{pmatrix} & & v_3 = \begin{pmatrix}1\\1\\2\end{pmatrix}
\end{align}
\begin{align}
\mathsf{L}_A(v_1)
&= \begin{pmatrix}1&1&-1\\2&0&1\\1&1&0\end{pmatrix}\begin{pmatrix}1\\0\\1\end{pmatrix}
= \begin{pmatrix}1\\3\\2\end{pmatrix}\\
\mathsf{L}_A(v_2) &= \begin{pmatrix}1 & 1 & -1\\2&0&-1\\1& 1
  &0\end{pmatrix} \begin{pmatrix}1\\0\\1\end{pmatrix}
= \begin{pmatrix}0\\3\\1\end{pmatrix}\\
\mathsf{L}_A(v_3)
&= \begin{pmatrix}1&1&-1\\2&0&1\\1&1&0\end{pmatrix}\begin{pmatrix}1\\1\\2\end{pmatrix}
= \begin{pmatrix}0\\4\\2\end{pmatrix}
\end{align}
\begin{equation}
\begin{pmatrix}1\\3\\2\end{pmatrix}=
a\begin{pmatrix}1\\1\\1\end{pmatrix} +
b\begin{pmatrix}1\\0\\1\end{pmatrix} +
c\begin{pmatrix}1\\1\\2\end{pmatrix}
\end{equation}
\begin{align*}
& a+b+c &=1 & & a+b +c &=1 & & b &= -2\\
\implies& a +c &=3 &\implies& a+c &= 3 &\implies& a &=2\\
& a+b + 2c &= 2 & & c &=1 & &c &=1
\end{align*}
\begin{equation}
\implies \left[\mathsf{L}_A(v_1)\right]_\beta = \begin{pmatrix}2\\-2\\1\end{pmatrix}
\end{equation}

\begin{equation}
\begin{pmatrix}0\\3\\1\end{pmatrix}=
a\begin{pmatrix}1\\1\\1\end{pmatrix} +
b\begin{pmatrix}1\\0\\1\end{pmatrix} +
c\begin{pmatrix}1\\1\\2\end{pmatrix}
\end{equation}
\begin{align*}
& a+b+c &=0 & & a+b +c &=0 & & b &= -3\\
\implies& a +c &=3 &\implies& a &= 2 &\implies& a &=2\\
& a+b + 2c &= 1 & & c &=1 & &c &=1
\end{align*}
\begin{equation}
\implies \left[\mathsf{L}_A(v_2)\right]_\beta = \begin{pmatrix}2\\-3\\1\end{pmatrix}
\end{equation}

\begin{equation}
\begin{pmatrix}0\\4\\2\end{pmatrix}=
a\begin{pmatrix}1\\1\\1\end{pmatrix} +
b\begin{pmatrix}1\\0\\1\end{pmatrix} +
c\begin{pmatrix}1\\1\\2\end{pmatrix}
\end{equation}
\begin{align*}
& a+b+c &=0 & & a+b +c &=0 & & b &= -4\\
\implies& a +c &=4 &\implies& a &= 2 &\implies& a &=2\\
& a+b + 2c &= 2 & & c &=2 & &c &=2
\end{align*}
\begin{equation}
\implies \left[\mathsf{L}_A(v_3)\right]_\beta = \begin{pmatrix}2\\-4\\2\end{pmatrix}
\end{equation}
\begin{equation}
\implies [\mathsf{L}_A]_\beta = \begin{pmatrix}2&2&2\\-2&-2&-4\\1&1&2\end{pmatrix}
\end{equation}
\begin{equation}
Q = \begin{pmatrix}
1 & 1 & 1\\
1 & 0 & 1\\
1 & 1 & 2
\end{pmatrix}
\end{equation}
\begin{equation*}
\begin{gmatrix}[left]
1 & 1 & 1&\\
1 & 0 & 1&\\
1 & 1 & 2&
\end{gmatrix}
\begin{gmatrix}[right]
1 & 0 & 0\\
0 & 1 & 0\\
0 & 0 & 1
\rowops
\add[-1]{0}{1}
\add[-1]{0}{2}
\add[1]{1}{0}
\add[1]{2}{0}
\mult{1}{\cdot -1}
\end{gmatrix}
\end{equation*}
\begin{equation}
\rightarrow
\begin{gmatrix}[left]
1 & 0 & 0& \\
0 & 1 & 0&\\
0 & 0 & 1&
\end{gmatrix}
\begin{gmatrix}[right]
1 & 1 & -1\\
1 & -1 & 0\\
-1 & 0 & 1
\end{gmatrix}
\end{equation}
\begin{equation}
\implies Q^{-1} = \begin{pmatrix}
1 & 1 & -1\\
1 & -1 & 0\\
-1 & 0 & 1
\end{pmatrix}
\end{equation}
\end{enumerate}

\end{enumerate}
\subsection*{5.1}
\begin{enumerate}
\setcounter{enumi}{2}
\item
Using the results of Exercise 2, find all solutions to the following
systems.
\begin{enumerate}
\item \begin{align}
x_1 +3x_2 = 5 & & 2x_1 + 6x_2 = 10
  \end{align}
\begin{equation}
\begin{gmatrix}[left]
1 & 3&\\
2 & 6&
\end{gmatrix}
\begin{gmatrix}[right]
5\\
10
\rowops
\add[-2]{0}{1}
\end{gmatrix}
\leadsto
\begin{gmatrix}[left]
1 & 3&\\
0 & 0&
\end{gmatrix}
\begin{gmatrix}[right]
5 \\ 0
\end{gmatrix}
\end{equation}
\begin{align}
\implies x_2 = t && x_1 = 5-3t
\end{align}
\begin{equation}
x = \left\{\begin{pmatrix}5\\0\end{pmatrix}
  +t \begin{pmatrix}-3\\1\end{pmatrix}\colon t \in \mathbb{R}\right\}
\end{equation}
\setcounter{enumii}{3}
\item\begin{align}
2x_1 + x_2 - x_3 =5 & & x_1 - x_2 +x_3 =1 & & x_1 +2x_2 -2x_3 =4
  \end{align}
\begin{equation}
\begin{gmatrix}[left]
2 & 1 & -1 &\\
1 & -1 & 1 &\\
1 & 2 & -2 &
\end{gmatrix}
\begin{gmatrix}[right]
5\\
1\\
4
\rowops
\swap{0}{1}
\add[-2]{0}{1}
\add[-1]{0}{2}
\add[-1]{1}{2}
\mult{1}{\cdot \frac{1}{3}}
\end{gmatrix}
\leadsto
\begin{gmatrix}[left]
1 & -1 &1&\\
0 & 1& -1&\\
0 & 0 & 0&
\end{gmatrix}
\begin{gmatrix}[right]
1\\
1\\
0
\end{gmatrix}
\end{equation}
\begin{align}
\implies x_3 =t && x_2 1 + t && x_1 =2
\end{align}
\begin{equation}
\implies x = \left\{\begin{pmatrix}2\\1\\0\end{pmatrix} +
  t \begin{pmatrix}0\\1\\1\end{pmatrix}\colon t \in \mathbb{R}\right\}
\end{equation}
\end{enumerate}

\item
Let $\mathsf{T}$ be a linear operator on a vector space $\mathsf{V},$
and let $\mathsf{W}$ be a $\mathsf{T}$-invariant subspace of
$\mathsf{V}.$ Prove that $\mathsf{W}$ is $g(\mathsf{T})$-invariant for
any polynomial $g(t).$

Suppose $\mathsf{T} \in \mathcal{L}(\mathsf{V})$ 

Let $\mathsf{W}$ be a $\mathsf{T}$-invariant subspace of $\mathsf{V}$

\paragraph{Lemma:} $\mathsf{T}^k(\mathsf{W})$ is
$\mathsf{T}$-invariant for all $k \in \mathbb{Z}^+$

Proof by induction.

Base case: Suppose $k=1$
\begin{gather}
\mathsf{T}(\mathsf{W}) \subseteq \mathsf{W}\\
\implies \mathsf{T}^2(\mathsf{W}) = \mathsf{T}(\mathsf{T}(\mathsf{W}))
\subseteq \mathsf{W}
\end{gather}
Suppose $\mathsf{T}^k(\mathsf{W})$ is $\mathsf{T}$-invariant for $1
\leq k \leq n.$

Suppose $k=n+1$
\begin{gather}
\mathsf{T}^{n+1}(\mathsf{W}) \subseteq
\mathsf{T}(\mathsf{T}^n(\mathsf{W}))\\
\mathsf{T}^n(\mathsf{W}) \subseteq \mathsf{W}\\
\implies \mathsf{T}^{n+1}(\mathsf{W}) =
\mathsf{T}(\mathsf{T}^n(\mathsf{W})) \subseteq \mathsf{W}
\end{gather}
\begin{equation}
\notag \therefore \mathsf{T}^k(\mathsf{W}) \text{ is }
\mathsf{T}\text{-invariant for all } k \in \mathbb{Z}^+ \qquad \qedsymbol
\end{equation}
Suppose $w \in \mathsf{W}$ and $g(t) \in \mathsf{P}(F)$ such that 
\begin{equation}
g(t) = a_nx^n +a_{n-1}x^{n-1} + \dotsb + a_1x + a_0 
\end{equation}
\begin{gather}
\implies g(\mathsf{T}) = a_n\mathsf{T}^n + a_{n-1}\mathsf{T}^{n-1} +
\dotsb + a_1\mathsf{T} +a_0\mathsf{I}_\mathsf{V}\\
\implies g(\mathsf{T})(\mathsf{W}) = a_n\mathsf{T}^n(w) +
a_{n-1}\mathsf{T}^{n-1}(w) +\dotsb + a_1\mathsf{T}(w) +
a_0\mathsf{I}_\mathsf{V}\\
\implies g(\mathsf{T})(w) \in \mathsf{W}\; \forall w \in \mathsf{W}
\end{gather}
\begin{equation}
\notag \therefore \mathsf{W} \text{ is } g(\mathsf{T})\text{-invariant}
\end{equation}

\setcounter{enumi}{6}
\item
Let $\mathsf{T}$ be a linear operator on a finite-dimensional vector
space $\mathsf{V}$. We define the {\bf determinant} of $\mathsf{T}$,
denoted $\det{(\mathsf{T})}$, as follows: Choose any ordered basis
$\beta$ for $\mathsf{V}$, and define $\det{(\mathsf{T})}=
\det{([\mathsf{T}]_\beta )}$.
\begin{enumerate}
\item Prove that he preceding definition is independent of the choice
  of an ordered basis for $\mathsf{V}$. That is, prove that if $\beta$
  and $\gamma$ are two ordered bases for $\mathsf{V}$, then $\det{([\mathsf{T}]_\beta)}=\det{([\mathsf{T}]_\gamma)}$.
\item Prove that $\mathsf{T}$ is invertible if and only if
  $\det{\mathsf{T}}\neq 0$.
\item Prove that if $\mathsf{T}$ is invertible, then
  $\det{(\mathsf{T}^{-12})}= [\det{(\mathsf{T})}]^{-1}$.
\item Prove that if $\mathsf{U}$ is also a linear operator on
  $\mathsf{V}$, then $\det{(\mathsf{TU})} = \det{(\mathsf{T})}\cdot\det{(\mathsf{U})}$.
\item Prove that $\det{(\mathsf{T}-\lambda \mathsf{I}_\mathsf{V})} =
    \det{[\mathsf{T}]_\beta -\lambda I)}$ for any scalar $\lambda$ and
      any ordered basis $\beta$ for $\mathsf{V}$.
\end{enumerate}
\begin{enumerate}
\item
Suppose $Q$ is the change of coordinates matrix from $\gamma$ to
$\beta$.
\begin{gather}
Q=[\mathsf{I}_\mathsf{V}]_\gamma^\beta \implies
Q^{-1}=[\mathsf{I}_\mathsf{V}]^\gamma_\beta \\
[\mathsf{I}_\mathsf{V}]_\gamma^\beta
[\mathsf{T}]_\gamma[\mathsf{I}_\mathsf{V}]^\gamma_\beta =
[\mathsf{T}]_\beta\\
\implies \det{([\mathsf{I}_\mathsf{V}]_\gamma^\beta
[\mathsf{T}]_\gamma[\mathsf{I}_\mathsf{V}]^\gamma_\beta)} =
\det{[\mathsf{T}]_\beta}\\
\implies\det{[\mathsf{I}_\mathsf{V}]_\gamma^\beta}
\det{[\mathsf{T}]_\gamma}\det{[\mathsf{I}_\mathsf{V}]^\gamma_\beta}\\
[\mathsf{I}_\mathsf{V}]_\gamma^\beta =
([\mathsf{I}_\mathsf{V}]^\gamma_\beta)^{-1}\\
\det{[\mathsf{I}_\mathsf{V}]_\gamma^\beta}=
\det{([\mathsf{I}_\mathsf{V}]^\gamma_\beta)^{-1}}
\end{gather}
\begin{align}
\implies \det{[\mathsf{I}_\mathsf{V}]_\gamma^\beta}
\det{[\mathsf{T}]_\gamma}\det{[\mathsf{I}_\mathsf{V}]^\gamma_\beta} &=
\det{([\mathsf{I}_\mathsf{V}]^\gamma_\beta)^{-1}}\det{[\mathsf{T}]_\gamma}\det{[\mathsf{I}_\mathsf{V}]^\gamma_\beta}\\
&= \det[\mathsf{T}]_\gamma = \det[\mathsf{T}]_\beta
\end{align}
\item 
($\implies$) 

Suppose $\mathsf{T}$ is invertible\\
Suppose $\beta$ is an ordered basis of $\mathsf{V}$.
\begin{gather}
[\mathsf{I}_\mathsf{V}]_\beta = [\mathsf{T}\cdot\mathsf{T}^{-1}]_\beta
= [\mathsf{T}]_\beta[\mathsf{T}^{-1}]_\beta\\
\implies \det{[\mathsf{I}_\mathsf{V}]_\beta} =
\det{[\mathsf{T}]_\beta}\det{[\mathsf{T}^{-1}]_\beta}\\
\det{[\mathsf{I}_\mathsf{V}]_\beta} =1 \\
\implies \det{[\mathsf{T}]_\beta}\det{[\mathsf{T}^{-1}]_\beta} =1\\
\implies \det{[\mathsf{T}]_\beta} \neq 0\\
\implies \det{\mathsf{T}} \neq 0
\end{gather}
($\Leftarrow$)

Suppose $\det{\mathsf{T}}\neq 0$ 
\begin{gather}
\implies \det[\mathsf{T}]_\beta \neq 0 \quad \text{for some ordered
  basis } \beta \text{ of } \mathsf{V}\\
\det{[\mathsf{T}]_\beta} \neq  0\\
\implies \mathsf{T} \text{ is invertible, by corollary to Th. 2.18}
\end{gather}
\item Suppose $\mathsf{T}$ is invertible and $\beta$ is some ordered
  basis of $\mathsf{V}$.
\begin{align}
\det{\mathsf{I}_\mathsf{V}} &= \det{\mathsf{T}\cdot\mathsf{T}^{-1}}\\
&= \det{[\mathsf{T}\cdot\mathsf{T}^{-1}]_\beta}\\
&= \det{[\mathsf{T}]_\beta[\mathsf{T}^{-1}]_\beta}\\
&= \det{[\mathsf{T}]_\beta}\det{[\mathsf{T}^{-1}]_\beta}\\
&= \det{\mathsf{T}}\det{\mathsf{T}^{-1}}
\end{align}
\begin{gather}
\det{\mathsf{I}_\mathsf{V}} = \det{[\mathsf{I}_\mathsf{V}]_\beta} =
  \det{I_n} =1\\
\implies \det{\mathsf{T}} \det{\mathsf{T}^{-1}} = 1\\
\implies \det{\mathsf{T}^{-1}} = (\det{\mathsf{T}})^{-1}
\end{gather}
\item Suppose $\beta$ is an ordered basis of $\mathsf{V}$.
\begin{align}
\det{\mathsf{TU}} &= \det{[\mathsf{TU}]_\beta}\\
&= \det{[\mathsf{T}]_\beta[\mathsf{U}]_\beta}\\
&= \det{[\mathsf{T}]_\beta}\det{[\mathsf{U}]_\beta}\\
&= \det{\mathsf{T}}\det{\mathsf{U}}
\end{align}
\item 
\begin{align}
\det{([\mathsf{T}]_\beta -\lambda I)} &= \det{([\mathsf{T}]_\beta
  -\lambda[\mathsf{I}_\mathsf{V}]_\beta)}\\
&= \det{([\mathsf{T}]_\beta
  -[\lambda\mathsf{I}_\mathsf{V}]_\beta)}\\
&= \det{[\mathsf{T}_\beta -\lambda \mathsf{I}_\mathsf{V}]_\beta}\\
&= \det{(\mathsf{T}-\lambda \mathsf{I}_\mathsf{V})}
\end{align}
\end{enumerate}

\setcounter{enumi}{11}
\item
Is there a linear transformation $\mathsf{T}\colon\mathsf{R}^3 \to
\mathsf{R}^2$ such that $\mathsf{T}(1,0,3) = (1,1)$ and
$\mathsf{T}(-2,0,-6) = (2,1)$?
\paragraph{} No since if $\mathsf{T}$ is a linear transformation, by the
defination of a linear transformation $\mathsf{T}(cx) =
c\times\mathsf{T}(x)$. But the given $\mathsf{T}$ shows that
\begin{align}
\mathsf{T}(-2(1,0,3)) = \mathsf{T}(-2,0,-6) = (2,1)\\
-2\mathsf{T}(-2(1,0,3)) = -2 \times (1,1) = (-2,-2)
\end{align}
Since $\mathsf{T}(-2(1,0,3)) \neq -2\mathsf{T}(-2(1,0,3))$ the
transformation $\mathsf{T}$ is not a linear transformation.

\setcounter{enumi}{13}
\item
Let $\mathsf{V}$ and $\mathsf{W}$ be vector spaces and
$\mathsf{T}\colon\mathsf{V} \to \mathsf{W}$ be linear.
\begin{enumerate}[(a)]
\item Prove that $\mathsf{T}$ is one-to-one if and only if
  $\mathsf{T}$ carries linearly independent subsets of $\mathsf{V}$
  onto linearly independent subsets of $\mathsf{W}$.
\item Suppose that $\mathsf{T}$ is one-to-one and that $S$ is a subset
  of $\mathsf{V}$. Prove that $S$ is linearly independent if and only
  if $\mathsf{T}(S)$ is linearly independent.
\item Suppose $\beta = \left\{v_1,v_2,\dots,v_n\right\}$ is a basis
  for $\mathsf{V}$ and $\mathsf{T}$ is one-to-one and onto. Prove that
  $\mathsf{T}(\beta) =
  \left\{\mathsf{T}(v_1),\mathsf{T}(v_2),\dots,\mathsf{T}(v_n)\right\}$
  is a basis for $\mathsf{W}$.
\end{enumerate}
\begin{enumerate}[(a)]
\item 
\textbf{Forward Direction:}
\\Suppose $S \subseteq \mathsf{V}$ such that  $S$ is linearly independent and
$\mathsf{T}$ is one-to-one.\\ Let $\mathsf{T}(S) = \{\mathsf{T}(x)
\colon x \in S\}$
\\Claim: $\mathsf{T}(S)$ is linearly independent.
\\Suppose $x \in \text{span}(\mathsf{T}(S))$ such that $ x = c_1v_1
+c_2v_2 + \cdots + c_nu_n =0$ for $c_i \in F$ and $v_i \in
\mathsf{T}(S)$
\paragraph{}
$\mathsf{T}$ is one-to-one $\implies v_i =\mathsf{T}(v_i)$ for some
$v_i \in S$\\
{\huge MAKES NO FUCKING SENSE}
\\\textbf{Reverse Direction:}
\end{enumerate}

\item
Prove the corollary to Theorem 3.16: The reduced row echelon form of a
matrix is unique. 
\paragraph{} Suppose $A \in \M{m}{n}{F}$ such that $\rank{A} = r
\leq\text{min}\{m,n\}$

Proof by induction 

Suppose $n=1$ 
\paragraph{Case 1}
\begin{equation}
A = \begin{pmatrix} 0 \\ 0 \\ \vdots \\ 0\end{pmatrix}
\end{equation}
$A$ is in reduced row echelon form and this the unique representation.
\paragraph{Case 2}
\begin{equation}
A = \begin{pmatrix}a_{11}\\a_{21}\\\vdots\\A_{m1}\end{pmatrix} \quad
\text{for some } a_{i1} \neq 0
\end{equation}
\paragraph{} Execute the following sequence of row operations on
$A$. Perform type 1 row operation to move $a_{i1}$ to the first
row. Perform type 3 row operations to eliminate all lower
terms. Perform type 2 row operation to the change the first term in
the column to 1.
\begin{equation}
A \leadsto \begin{pmatrix}1\\0\\0\\\vdots\\0\end{pmatrix}
\end{equation}
The reduced row echelon form of a column must be of this form. It
follows that this matrix is the unique row reduced echelon form.

Suppose true for $1\leq n \leq k$

Suppose $n = k+1$

Suppose $A \in \M{m}{k+1}{F}$. Suppose $A = (A^\prime\,|b)$ such that
$A^\prime \in \M{m}{k}{F}$ and $ b \in \M{m}{1}{F}$

Let $(B^\prime\,|b^\prime)$ be the reduced row echelon form of
$(A^\prime\,|b)$ By HW.3.4.14, $B^\prime$ is in reduced row echelon
form. It follows from $B^\prime \in \M{m}{k}{F}$ that, by the
induction hypothesis, that $B^\prime$ is in the unique row reduced
echelon form of $A^\prime$.

It remains to show that the column $b^\prime$ is unique.

\paragraph{Case 1:} $b \notin \text{col}(A^\prime)$

If $b$ is not in the column space of $A^\prime$, then $b^\prime$ is not in
the column space of $B^\prime$. It follows that $b^\prime$ cannot be
replaced as a linear combination of vectors from a basis for the
column space of $B^\prime$. Column $b^\prime$ is replaced as:
\begin{equation}
\begin{pmatrix}
0\\0\\\vdots\\1\\\vdots\\0\\0
\end{pmatrix}
\end{equation}
Where $b^\prime_{r+1,1}=1$. This is the unique representation. Suppose
$b^\prime$ were
\begin{equation}
\begin{pmatrix}
0\\0\\\vdots\\1\\\vdots\\0\\0
\end{pmatrix}
\end{equation}
where $b^\prime_{r+j,1}=1, j > 1$. $B^\prime$ is in reduced row
echelon form and is of rank $r$ . It follows that all rows after the
$r^{\text{th}}$ are entirely zero. $(B^\prime\,|b^\prime)$ is not in
reduced row echelon form because there is a row of zeros above a row
containing a nonzero value.
\paragraph{Case 2:} $b \in \text{col}(A^\prime)$
\begin{gather}
\implies b^\prime \in \text{col}(B^\prime)\\
b^\prime = \sum\limits_{i=1}^r c_iv_i
\end{gather}
For $c_i \in F$ and $v_i\in \beta$ where $\beta$ is a basis for
$\text{col}(B^\prime)$ such that $\beta$ is a subset of the standard ordered basis
for $\mathsf{F}^m$. Because $\beta$ is linearly independent, the
coefficients of the linear combination are unique.
\begin{gather}
b^\prime \in \text{col}(B^\prime)\\
\implies \rank{B^\prime} = \rank{B^\prime\,|b^\prime}
\end{gather}
It follows that all rows of $(B^\prime\,|b^\prime)$ after the
$r^{\text{th}}$  are entirely zero. Therefore $(B^\prime\,|b^\prime)$.

\setcounter{enumi}{18}
\item
Let $A$ denote the $k\times k$ matrix
\[
\begin{pmatrix}
0 & 0 & \cdots & 0 & -a_0\\
1 & 0 & \cdots & 0 & -a_1\\
0 & 1 & \cdots & 0 & -a_2\\
\vdots& \vdots & & \vdots & \vdots\\
0 & 0 & \cdots & 0 & -a_{k-2}\\
0 & 0 & \cdots & 1 & -a_{k-1}
\end{pmatrix}
\]
where $a_0,a_1,\dotsc,a_{k-1}$ are arbitrary scalars. Prove that the
characteristic polynomial of $A$ is
\[
(-1)^k(a_0 +a_1t + \dotsb + a_{k-1}t^{k-1} + t^k)
\]

Proof by induction on $k.$

Suppose $k=1$
\begin{align}
\implies A &= -a_0\\
\det{(A- tI_1)} &= \det{(-a_0 -t)} \\
&= -a_0 -t \\
&= (-1)^1(a_0 + t^1)
\end{align}

Suppose true for $2 \leq k \leq n-1$

Suppose $k=n$
\begin{gather}
\implies A =
\begin{pmatrix}
0 & 0 & \cdots & 0 & -a_0\\
1 & 0 & \cdots & 0 & -a_1\\
0 & 1 & \cdots & 0 & -a_2\\
\vdots& \vdots & & \vdots & \vdots\\
0 & 0 & \cdots & 0 & -a_{n-2}\\
0 & 0 & \cdots & 1 & -a_{n-1}
\end{pmatrix}\\
\implies A - tI_n =
\begin{pmatrix}
-t & 0 & \cdots & 0 & -a_0\\
1 & -t & \cdots & 0 & -a_1\\
0 & 1 & \cdots & 0 & -a_2\\
\vdots& \vdots & & \vdots & \vdots\\
0 & 0 & \cdots & -t & -a_{n-2}\\
0 & 0 & \cdots & 1 & -a_{n-1} -t
\end{pmatrix}\\
\implies \det{(A-tI_n)} = (-t)(-1)^2\det{\tilde{A}_{11}} +
(-a_0)(-1)^{n+1}\det{\tilde{A}_{1n}}
\end{gather}
\begin{align}
\det{\tilde{A}_{11}} &= \det{\begin{pmatrix}
-t & 0 & \cdots & 0 & -a_1\\
1 & -t & \cdots & 0 & -a_2\\
0 & 1 & \cdots & 0 & -a_3\\
\vdots& \vdots & & \vdots & \vdots\\
0 & 0 & \cdots & -t & -a_{n-2}\\
0 & 0 & \cdots & 1 & -a_{n-1} -t
  \end{pmatrix}}\\
&= (-1)^{n-1}(a_1 +a_2 t + \dotsb + a_{n-1}t^{n-2} + t^{n-1}) 
\end{align}
\begin{align}
\det{\tilde{A}_{1n}} &= \det{\begin{pmatrix}
1 & -t & 0 & \cdots & 0 \\
0 & 1 & -t & \cdots & 0  \\
0 & 0 & 1 & \cdots & 0  \\
\vdots & \vdots& \vdots & & \vdots \\
0 & 0 & 0 & \cdots & -t \\
0 & 0 & 0 & \cdots & 1 
    \end{pmatrix}}\\
&= 1 \because (\tilde{A}_{1n})_{ii} =1 \forall i (1\leq i \leq n-1)
\end{align}
\begin{align}
\implies \det{(A-tI_n)} &= (-1)(t)(-1)^{n-1}(a_1 +a_2 t + \dotsb +
a_{n-1}t^{n-2} + t^{n-1}) + (a_0)(-1)^n\\
&= (-1)^n(a_0 + a_1 +a_2 t + \dotsb + a_{n-1}t^{n-2} + t^{n-1})
\end{align}

\setcounter{enumi}{21}
\item
\begin{enumerate}
\item Let $\mathsf{T}$ be a linear operator on a vector space
  $\mathsf{V}$ over the field $F$, and let $g(t)$ be a polynomial with
  coefficients form $F$. Prove that if $x$ is an eigenvector of
  $\mathsf{T}$ with corresponding eigenvalue $\lambda$, then
  $g(\mathsf{T})(x)= g(\lambda)(x)$. That is, $x$ is an eigenvector of
  $g(\mathsf{T})$ with corresponding eigenvalue $g(\lambda)$.
\item State and prove the comparable results for matrices.
\item Verify (b) for the matrix $A$ in Exercise 3(a) with a
  polynomial $g(t) 2t^2 -t +1$, eigenvector $x
  = \begin{pmatrix}2\\3\end{pmatrix}$, and corresponding eigenvalue
  $\lambda = 4$.
\end{enumerate}
\begin{enumerate}
\item Suppose $\mathsf{T} \in \mathcal{L}(\mathsf{V})$, $\mathsf{V}$
  is a vector space over $F$.

Let $g(t)$ be a polynomial with coefficients from $F$

Claim: If $x$ is an eigenvector of $\mathsf{T}$ with corresponding
eigenvalue $\lambda$ then $g(\mathsf{T})(x) = g(\lambda)x$

Suppose $g(t)$ is of degree $n$;
\begin{align}
\implies g(t) &= a_nx^n +a_{n-1}x^{n-1} +\dotsb+ a_1x + a_o\\
\implies g(\mathsf{T}) &= a_n\mathsf{T}^n + a_{n-1}\mathsf{T}^{n-1}
+\dotsb + a_1\mathsf{T} +a_0\mathsf{I}_{\mathsf{V}}\\
\implies g(\mathsf{T}) (x)&= a_n\mathsf{T}^n(x) + a_{n-1}\mathsf{T}^{n-1}(x)
+\dotsb + a_1\mathsf{T}(x) +a_0\mathsf{I}_{\mathsf{V}}(x)\\
&=a_n\lambda^nx +a_{n-1}\lambda^{n-1}x +\dotsb +a_1\lambda x+ a_0x
\text{ by 5.1.15.a}\\
&=(a_n\lambda^n +a_{n-1}\lambda^{n-1} +\dotsb +a_1\lambda + a_0)x\\
&=g(\lambda)x
\end{align}
\item Prove that if $x$ is an eigenvector of $A$, with corresponding
  eigenvalue $\lambda$ then $g(A)(x) = g(\lambda)(x)$.

Suppose $A \in \mathsf{M}_{n\times n}(F)$, and Let $g(t) \in
\mathsf{P}_n(F)$ such that g(t) is of degree $n$, and $x$ is an
eigenvector of $A$ corresponding to the eigenvalue $\lambda$
\begin{align}
\implies g(t) &= a_nx^n +a_{n-1}x^{n-1} +\dotsb+ a_1x + a_o\\
\implies g(A) &= a_nA^n + a_{n-1}A^{n-1}
+\dotsb + a_1A +a_0I_n\\
\implies g(A) (x)&= a_nA^n(x) + a_{n-1}A^{n-1}(x)
+\dotsb + a_1A(x) +a_0I_n(x)\\
&=a_n\lambda^nx +a_{n-1}\lambda^{n-1}x +\dotsb +a_1\lambda x+
a_0x\text{ by 5.1.15.b}\\
&=(a_n\lambda^n +a_{n-1}\lambda^{n-1} +\dotsb +a_1\lambda + a_0)x\\
&=g(\lambda)x
\end{align}
\item 
\begin{gather}
A  = \begin{pmatrix}
1 & 2\\
3 & 2\\
\end{pmatrix}\\
A^2 =\begin{pmatrix}
7 & 6\\
9 & 10
\end{pmatrix}\\
g(A) = 2\begin{pmatrix}
7 & 6 \\
9 & 10
\end{pmatrix}
-\begin{pmatrix}
1 & 2 \\
2 & 3
\end{pmatrix}
+
\begin{pmatrix}
1 & 0\\
0 & 1
\end{pmatrix}
\end{gather}
\begin{align}
g(A)\begin{pmatrix}2\\3\end{pmatrix} &= 2\begin{pmatrix}
7 & 6 \\
9 & 10
\end{pmatrix}\begin{pmatrix}2\\3\end{pmatrix}
-\begin{pmatrix}
1 & 2 \\
2 & 3
\end{pmatrix}\begin{pmatrix}2\\3\end{pmatrix}
+
\begin{pmatrix}
1 & 0
0 & 1
\end{pmatrix}\begin{pmatrix}2\\3\end{pmatrix}\\
&=
(2)(16)\begin{pmatrix}2\\3\end{pmatrix} +\begin{pmatrix}2\\3\end{pmatrix}
(-1)(4)\begin{pmatrix}2\\3\end{pmatrix}
+ \begin{pmatrix}2\\3\end{pmatrix}\\
&= \left((2)(4^2)+(-1)(4)+(1)\right)\begin{pmatrix}2\\3\end{pmatrix}\\
&= g(4)\begin{pmatrix}2\\3\end{pmatrix}
\end{align}
\end{enumerate}

\end{enumerate}
% \end{spacing}
\end{document}

