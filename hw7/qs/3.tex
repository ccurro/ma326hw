For each of the following linear operators $\mathsf{T}$ on a vector
space $\mathsf{V}$, test $\mathsf{T}$ for diagonalizability, and if
$\mathsf{T}$ is diagonalizable, find a basis $\beta$ for $\mathsf{V}$
such that $[\mathsf{T}]_\beta$ is a diagonal matrix.
\begin{enumerate}
\item $\mathsf{V}=\mathsf{P}_3(\mathbb{R})$ and $\mathsf{T}$ is
  defined by $\mathsf{T}(f(x)) = f^\prime(x) + f^{\prime\prime}(x),$
  respectively.

Let $\alpha$ be the standard ordered basis of
$\mathsf{P}_3(\mathbb{R})$.
\begin{gather}
\mathsf{T}(\alpha) = \{0,1,2x+2,3x^2+6x\}\\
\implies [\mathsf{T}]_\alpha = \begin{pmatrix}
0 & 1 & 2 & 0\\
0 & 0 & 2 & 6\\
0 & 0 & 0 & 3\\
0 & 0 & 0 & 0
\end{pmatrix}\\
\det{([\mathsf{T}]_\beta -\lambda I_4)} = \det{\begin{pmatrix}
-\lambda & 1 & 2 & 0\\
0 & -\lambda & 2 & 6\\
0 & 0 & -\lambda & 3\\
0 & 0 & 0 & -\lambda
    \end{pmatrix}} = \lambda^4\\
\implies \lambda_1 = 0 \text{ with multiplicity 4}\\
\rank{[\mathsf{T}]_\beta - \lambda_1I_3} = 3\\
4-3\neq 4 \text{ multiplicity of } \lambda_1 \\
\notag \implies \mathsf{T} \text{ is not diagonalizable}
\end{gather}
\setcounter{enumii}{3}
\item $\mathsf{V} = \mathsf{P}_2(\mathbb{R})$ and $\mathsf{T}$ is
  defined by $\mathsf{T}(f(x)) = f(0) +f(1)(x+x^2)$.

Suppose $\alpha$ is the standard ordered basis of
$\mathsf{P}_2(\mathbb{R})$.

\begin{gather}
\implies \mathsf{T}(\alpha) = \{x^2+x+1,x+x^2,x+x^2 \}\\
\implies [\mathsf{T}]_\alpha = \begin{pmatrix}
1 & 0 & 0\\
1 & 1 & 1 \\
1 & 1 & 1
\end{pmatrix}
\end{gather}
\begin{align}
\det{([\mathsf{T}]_\beta -\lambda I_2)} &= \det{\begin{pmatrix}
1-\lambda & 0 & 0\\
1 & 1-\lambda & 0\\
1 & 1 & 1-\lambda
  \end{pmatrix}}\\
&= (\lambda)(1-\lambda)(\lambda - 2)
\end{align}
\begin{align}
\implies \lambda_1 &= 0, \text{ multiplicity } 1\\
\lambda_2 &= 1, \text{ multiplicity } 1\\
\lambda_3 &= 2, \text{ multiplicity } 1
\end{align}
\begin{itemize}
\item For $\lambda_1 = 0$
\begin{gather}
[\mathsf{T}]_\beta -0 I_3= \begin{pmatrix}
1 & 0 & 0\\
1 & 1 & 1\\
1 & 1 & 1
\end{pmatrix}\\
\begin{pmatrix}
1 & 0 & 0\\
1 & 1 & 1\\
1 & 1 & 1
\end{pmatrix}\leadsto
\begin{pmatrix}
1 & 0 & 0\\
0 & 1 & 0\\
0 & 1 & 0
\end{pmatrix}\\
\implies \rank{[\mathsf{T}]_\beta} = 2\\
\text{multiplicity } \lambda_1 =1; \; 3-2 =1 \checkmark
\end{gather}
\item For $\lambda_2 = 1$
\begin{gather}
[\mathsf{T}]_\beta - I_3 = \begin{pmatrix}
0 & 0 & 0\\
1 & 0 & 1\\
1 & 1 & 0
\end{pmatrix}\\
\begin{pmatrix}
0 & 0 & 0\\
1 & 0 & 1\\
1 & 1 & 0
\end{pmatrix}\leadsto
\begin{pmatrix}
0 & 0 & 0\\
0 & 0 & 1\\
0 & 1 & 0
\end{pmatrix}\\
\implies \rank{[\mathsf{T}]_\beta -I_3} = 2 \\
\text{multiplicity } \lambda_2 =1; \; 3-2 =1 \checkmark
\end{gather}
\item For $\lambda_3 = 2$
\begin{gather}
[\mathsf{T}]_\beta - 2 I_3 = \begin{pmatrix}
-1 & 0 & 0\\
1 & -1 & 1\\
1 & 1 & 1
\end{pmatrix}\leadsto
\begin{pmatrix}
-1 & 0 & 0\\
0  & 0 & 1\\
2 & 0 & -1
\end{pmatrix}\\
\implies \rank{[\mathsf{T}]_\beta} =2\\
\text{multiplicity } \lambda_3 =1; \; 3-2 =1 \checkmark
\end{gather}
It follows that $\mathsf{T}$ is diagonalizable.
\end{itemize}
\begin{itemize}
\item For $\lambda_1 = 0$
\begin{equation}
\begin{pmatrix}
1 & 0 & 0\\
1 & 1 & 1 \\
1 & 1 & 1
\end{pmatrix}
\begin{pmatrix}
x_1\\x_2\\x_3
\end{pmatrix}
=
\begin{pmatrix}
0\\0\\0
\end{pmatrix}
\leadsto
\begin{pmatrix}
1 & 0 & 0\\
0 & 1 & 1\\
0 & 0 & 0
\end{pmatrix}
\begin{pmatrix}
x_1\\x_2\\x_3
\end{pmatrix}
=
\begin{pmatrix}
0\\0\\0
\end{pmatrix}
\end{equation}
\begin{gather}
\implies x_1 = 0\\
x_2 = -x_3\\
S_1 = \left\{z\begin{pmatrix}0\\-1\\1\end{pmatrix} \colon z \in
  \mathbb{R}\right \}\\
\notag \implies \begin{pmatrix}0\\-1\\1\end{pmatrix} \text{ is the
  eigenvector corresponding to } \lambda_2
\end{gather}
\item For $\lambda_2 = 1$
\begin{equation}
\begin{pmatrix}
0 & 0 & 0\\
1 & 0 & 1\\
1 & 1 & 0
\end{pmatrix}
\begin{pmatrix}
x_1\\x_2\\x_3
\end{pmatrix}
=
\begin{pmatrix}
0\\0\\0
\end{pmatrix}
\leadsto
\begin{pmatrix}
0 & 0 & 0\\
1 & 0 & 1\\
0 & 1 & -1
\end{pmatrix}
\begin{pmatrix}
x_1\\x_2\\x_3
\end{pmatrix}
=
\begin{pmatrix}
0\\0\\0
\end{pmatrix}
\end{equation}
\begin{gather}
\implies x_1 =- x_3\\
x_2 = x_3\\
S_2 = \left\{ z\begin{pmatrix}-1\\1\\1\end{pmatrix} \colon z \in
  \mathbb{R}\right\}\\
\notag \implies \begin{pmatrix}-1\\1\\1\end{pmatrix} \text{ is the
  eigenvector corresponding to } \lambda_2
\end{gather}
\item For $\lambda_3 = 2$
\begin{equation}
\begin{pmatrix}
-1 & 0 & 0\\
1 & -1 & 1\\
1 & 1 & -1
\end{pmatrix}
\begin{pmatrix}
x_1\\x_2\\x_3
\end{pmatrix}
=
\begin{pmatrix}
0\\0\\0
\end{pmatrix}
\leadsto
\begin{pmatrix}
-1 & 0 & 0\\
0 & 0 & 0\\
0 & 1 & -1
\end{pmatrix}
\begin{pmatrix}
x_1\\x_2\\x_3
\end{pmatrix}
=
\begin{pmatrix}
0\\0\\0
\end{pmatrix}
\end{equation}
\begin{align}
\implies x_1 &= 0\\
x_2 &= x_3\\
\end{align}
\begin{gather}
S_3 = \left\{z\begin{pmatrix}0\\1\\1\end{pmatrix}\colon z \in
  \mathbb{R}\right\}\\
\implies \begin{pmatrix}0\\1\\1\end{pmatrix} \text{ is the eigenvector
  corresponding to } \lambda_2\\
\implies \beta = \left\{\begin{pmatrix}0\\-1\\1
  \end{pmatrix},\begin{pmatrix}-1\\1\\1\end{pmatrix},\begin{pmatrix}0\\1\\1\end{pmatrix}\right\}\\
\implies [\mathsf{T}]_\beta = \begin{pmatrix}
0 & 0 & 0\\
0 & 1 & 0\\
0 & 0 & 2
\end{pmatrix}
\end{gather}
\end{itemize}
\item $\mathsf{V} = \mathsf{C}^2$ and $\mathsf{T}$ is defined by
  $\mathsf{T}(z,w) = (z+iw,iz+2)$

Suppose $\alpha = \left\{(1,0),(-,1) \right\}$ is a basis for $\mathsf{C}^2$
\begin{gather}
\mathsf{T}(\alpha) = \left\{(1,i),(i,1)\right \}\\
\implies [\mathsf{T}]_\alpha = \begin{pmatrix}
1 & i\\
i & 1
\end{pmatrix}
\end{gather}
\begin{align}
\det{([\mathsf{T}]_\beta -\lambda I_2)} &= \det{\begin{pmatrix}
1-\lambda & i\\
i & 1-\lambda 
\end{pmatrix}
}\\
&= \lambda^2-2\lambda +2
\end{align}
$\mathbb{C}$ is algebraically closed so the characteristic polynomial
splits over $\mathbb{C}$
\begin{gather}
\lambda^2-2\lambda +2 =0 \\
\implies 1\pm i
\end{gather}
\begin{align}
\implies \lambda_1 &= 1+i, \text{ multiplicity } 1\\
\lambda_2 &= 1-i, \text{ multiplicity } 1 
\end{align}
\begin{itemize}
\item For $\lambda_1 = 1 +i$
\begin{gather}
[\mathsf{T}]_\beta -(i+1)I_2 = \begin{pmatrix}
-i & i\\
i & -i
\end{pmatrix}\\
\begin{pmatrix}
-i & i\\
i & -i
\end{pmatrix}
\leadsto
\begin{pmatrix}
0 & 1\\
0 & -1
\end{pmatrix}\\
\rank{[\mathsf{T}]_\beta -(i+1)I_2} =1\\
\text{multiplicity } \lambda_1 =i+1; \; 2-1 =1 \checkmark
\end{gather}
\item For $\lambda_2 = 1 -i$
\begin{gather}
[\mathsf{T}]_\beta - (1-i)I_2 = \begin{pmatrix} i & i \\ i &
  i\end{pmatrix}\\
\begin{pmatrix}
 i & i \\
 i & i
\end{pmatrix}
\leadsto
\begin{pmatrix}
1 & 0\\
1 & 0 
\end{pmatrix}\\
\implies \rank{[\mathsf{T}]_\beta-(1-i)I_2}=1\\
\text{multiplicity } \lambda_2 =1-i; \; 2-1 =1 \checkmark\\
\end{gather}
\end{itemize}
It follows that $\mathsf{T}$ is diagonalizable.
\begin{itemize}
\item For $\lambda_1 = 1 +i$
\begin{gather}
\begin{pmatrix}
-i & i\\
i & -1
\end{pmatrix}
\begin{pmatrix}
x_1\\x_2\\x_3
\end{pmatrix}
=
\begin{pmatrix}
0\\0\\0
\end{pmatrix}
=
\begin{pmatrix}
1 & -1\\
0 & 0
\end{pmatrix}
\begin{pmatrix}
x_1\\x_2\\x_3
\end{pmatrix}
=
\begin{pmatrix}
0\\0\\0
\end{pmatrix}\\
\implies x_1 = x_2\\
S_1 = \left\{z\begin{pmatrix}1\\1\end{pmatrix}\colon x \in
  \mathbb{C}\right\}\\
\notag \begin{pmatrix}1\\1\end{pmatrix} \text{ is the eigenvector
  corresponding to } \lambda_1
\end{gather}
\item For $\lambda_2 = 1 -i$
\begin{gather}
\begin{pmatrix}
i & i\\
i & i
\end{pmatrix}
\begin{pmatrix}
x_1\\x_2\\x_3
\end{pmatrix}
=
\begin{pmatrix}
0\\0\\0
\end{pmatrix}
\leadsto
\begin{pmatrix}
1 & 1\\
0 & 0
\end{pmatrix}
\begin{pmatrix}
x_1\\x_2\\x_3
\end{pmatrix}
=
\begin{pmatrix}
0\\0\\0
\end{pmatrix}\\
\implies x_1 = -x_2 \\
S_2 = \left\{z\begin{pmatrix}-1\\1\end{pmatrix}\colon z\in
  \mathbb{C}\right\}
\\\notag \begin{pmatrix}-1\\1\end{pmatrix} \text{ is the eigenvector
  corresponding to } \lambda_2\\
\beta
=\left\{\begin{pmatrix}1\\1\end{pmatrix},\begin{pmatrix}-1\\1\end{pmatrix}\right\}\\
\implies [\mathsf{T}]_\beta =
\begin{pmatrix}
1+i & 0\\
0 & 1-i
\end{pmatrix}
\end{gather}
\end{itemize}
\item $\mathsf{V} = \mathsf{M}_{n \times n}(\mathbb{R})$ and
  $\mathsf{T}$ is defined by $\mathsf{T}(A) = A^t$

Suppose $\alpha$ is the standard ordered basis of $\mathsf{M}_{n\times
  n}(\mathbb{R})$
\begin{gather}
\implies \mathsf{T}(\alpha) =\left\{\begin{pmatrix}1&0\\0 &
    0\end{pmatrix},\begin{pmatrix}0&1\\0& 0\end{pmatrix},\begin{pmatrix}0
    & 0 \\1&0\end{pmatrix},\begin{pmatrix}0& 0
    \\0&1\end{pmatrix}\right\}\\
\implies [\mathsf{T}]_\alpha = \begin{pmatrix}
1 & 0 & 0 & 0\\
0 & 0 & 1 & 0\\
0 & 1 & 0 & 0\\
0 & 0 & 0 & 1
\end{pmatrix}
\end{gather}
\begin{align}
\implies \det{([\mathsf{T}]_\beta-\lambda I_4)} &= \det{
\begin{pmatrix}
1-\lambda & 0 & 0 & 0\\
0 & -\lambda & 1 &  0\\
0 & 1 & -\lambda & 0\\
0 & 0 &0 & 1-\lambda
\end{pmatrix}
}\\
&=(1-\lambda^2)(\lambda^2-1)\\
\end{align}
\begin{align}
\implies \lambda_1 &= 1, \text{ multiplicity } 3\\
\implies \lambda_2 &= -1, \text{ multiplicity } 1
\end{align}
\begin{itemize}
\item For $\lambda_1 = 1$
\begin{gather}
[\mathsf{T}]_\beta -I_4 =
\begin{pmatrix}
0 & 0 & 0 & 0\\
0 & -1 & 1 & 0\\
0 & 1 & -1 & 0\\
0 & 0 & 0 & 0
\end{pmatrix}\\
\begin{pmatrix}
0 & 0 & 0 & 0\\
0 & -1 & 1 & 0\\
0 & 1 & -1 & 0\\
0 & 0 & 0 & 0
\end{pmatrix}\leadsto
\begin{pmatrix}
0 & 0 & 0 & 0\\
0 & 0 & 1 & 0\\
0 & 0 & -1 & 0\\
0 & 0 & 0 & 0
\end{pmatrix}\\
\implies \rank{[\mathsf{T}]_\beta -I_4} = 1\\
\text{multiplicity } \lambda_1 =3; \; 4-1 =3 \checkmark
\end{gather}
\item For $\lambda_2 = -1$
\begin{gather}
[\mathsf{T}]_\beta +I_4 = 
\begin{pmatrix}
2 & 0 & 0 & 0\\
0 & 1 & 1 & 0\\
0 & 1 & 1 & 0\\
0 & 0 & 0 & 2
\end{pmatrix}\\
\begin{pmatrix}
2 & 0 & 0 & 0\\
0 & 1 & 1 & 0\\
0 & 1 & 1 & 0\\
0 & 0 & 0 & 2
\end{pmatrix}
\leadsto
\begin{pmatrix}
2 & 0 & 0 & 0\\
0 & 0 & 1 & 0\\
0 & 0 & 1 & 0\\
0 & 0 & 0 & 2
\end{pmatrix}\\
\implies \rank{[\mathsf{T}]_\beta +I_4} =3\\
\text{multiplicity } \lambda_2 =1; \; 4-3 =1 \checkmark
\end{gather}
\end{itemize}
It follows that $\mathsf{T}$ is diagonalizable.
\begin{itemize}
\item For $\lambda_1 = 1$
\begin{equation}
\begin{pmatrix}
0 & 0 & 0 & 0\\
0 & -1 & 1 & 0\\
0 & 1 & -1 & 0\\
0 & 0 & 0 & 0
\end{pmatrix}
\begin{pmatrix}
x_1\\x_2\\x_3\\x_4
\end{pmatrix}
=\begin{pmatrix}
0\\0\\0\\0
\end{pmatrix}\\
\begin{pmatrix}
0 & 0 & 0 & 0\\
0 & 0 & 0 & 0\\
0 & 1 & -1 & 0\\
0 & 0 & 0 & 0
\end{pmatrix}
\begin{pmatrix}
x_1\\x_2\\x_3\\x_4
\end{pmatrix}
=\begin{pmatrix}
0\\0\\0\\0
\end{pmatrix}
\end{equation}
\begin{align}
\implies x_1 &= x_1\\
x_2 &=x_3\\
x_4 &= x_4
\end{align}
\begin{equation}
\implies S_1 =
\left\{z_1\begin{pmatrix}1&0\\0&0\end{pmatrix}+z_2\begin{pmatrix}0 &
    1\\1&0\end{pmatrix} +z_3\begin{pmatrix}0 & 0\\0 &
    1\end{pmatrix}\colon z_1,z_2,z_3 \in \mathbb{R}\right\}
\end{equation}
\begin{gather}
\implies \begin{pmatrix}1&0\\0&0\end{pmatrix},\begin{pmatrix}0 &
    1\\1&0\end{pmatrix},\begin{pmatrix}0 & 0\\0 &
    1\end{pmatrix} \text{ are eigenvectors corresponding to } \lambda_1\notag
\end{gather}
\item For $\lambda_2 = -1$
\begin{equation}
\begin{pmatrix}
2 &  0 & 0 & 0\\
0 & 1 & 1 &0\\
0 & 1 & 1 &0\\
0 & 0 & 0 & 2
\end{pmatrix}
\begin{pmatrix}
x_1\\x_2\\x_3\\x_4
\end{pmatrix}
=\begin{pmatrix}
0\\0\\0\\0
\end{pmatrix}
\leadsto
\begin{pmatrix}
1 & 0 & 0 & 0\\
0 & 1 & 1 & 0\\
0 & 0 & 0 & 0\\
0 & 0 & 0 & 1
\end{pmatrix}
\begin{pmatrix}
x_1\\x_2\\x_3\\x_4
\end{pmatrix}
=\begin{pmatrix}
0\\0\\0\\0
\end{pmatrix}
\end{equation}
\begin{align}
x_1 &= x_4 =0 \\
x_2 &= -x_3
\end{align}
\begin{equation}
\implies S_2 = \left\{z\begin{pmatrix}0 & -1\\1 & 0\end{pmatrix}\right\}
\end{equation}
\end{itemize}
\begin{gather}
\beta = \left\{\begin{pmatrix}1 & 0\\0 &
    0\end{pmatrix},\begin{pmatrix}0 & 1\\1
    &0\end{pmatrix},\begin{pmatrix}0 & 0 \\0 &
    1\end{pmatrix},\begin{pmatrix}0 & -1 \\1 &
    0\end{pmatrix}\right\}\\
\implies [\mathsf{T}]_\beta =
\begin{pmatrix}
1 & 0 & 0 & 0\\
0 &1 & 0 & 0\\
0 & 0 & 1 & 0\\
0 & 0 & 0 & -1
\end{pmatrix}
\end{gather}
\end{enumerate}
