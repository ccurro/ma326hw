\documentclass[letterpaper,12pt]{article}
\usepackage{setspace}
\usepackage[utf8]{inputenc}
\usepackage[english]{babel}
\usepackage{enumerate}
\usepackage{graphicx}
\usepackage{amsmath}
\usepackage{amssymb}
\usepackage{amsthm}
\usepackage{amsfonts}
\usepackage{stmaryrd}
\usepackage{booktabs}
\usepackage{gauss}
\usepackage{mathtools}
\usepackage{multirow}
\usepackage{url}
\usepackage{siunitx}
\usepackage{microtype}
\usepackage{graphicx}
\usepackage{appendix}
\usepackage{verbatim}
\pagenumbering{Roman}
% Extra Commands
\newcommand{\tab}{\hspace*{5em}}
\newcommand{\gap}{\vspace*{0.25cm}}
\renewcommand{\implies}{\Rightarrow}
\newcommand{\T}{\mathsf{T}}
\newcommand{\U}{\mathsf{U}}
\newcommand{\V}{\mathsf{V}}
\newcommand{\W}{\mathsf{W}}
\newcommand{\tr}[1]{\text{tr}({#1})}
\newcommand{\rank}[1]{\text{rank}({#1})}
\newcommand{\dime}[1]{\text{dim}({#1})}
\newcommand{\dete}[1]{\text{det}({#1})}
\newcommand{\mfrac}[2]{^{#1}/_{#2}}
\newcommand{\M}[3]{\mathsf{M}_{#1 \times #2}(#3)}
\newmatrix{(}{|}{left}
\newmatrix{.}{)}{right}
\newcommand\bigzero{\makebox(0,0){\text{\huge0}}}
% Formatting
\usepackage[top=1in, bottom=1in, left=1.2in, right=1.2in]{geometry}
\usepackage{fancyhdr}

% Heading stuffs
% \pagestyle{plain}
\setlength{\headheight}{28pt}
\thispagestyle{fancy}
\fancyhf{}
\lhead{Chris Curro, John Biswakarma, Victor Chen \\ November 21, 2012} 
\chead{\hfill \\ MA326}
\rhead[RE,RO]{HW \#8 \\ Prof. Mintchev}
\cfoot[RO,RE]{\thepage}  %Page # for footer
% This should return the format
% Name         Course #        Assignment #
% Date         Course name     Prof. 
% -------------------------------------------
\begin{document}
%\begin{spacing}{1.2}
\section*{Assignment}
Appendix E: Prove theorems E.3, E.5, E.6, E.7; Section 5.2: 2(bde), 3(adef), 7, 9, 11
\section*{Work}
\subsection*{Appendix E}
\begin{enumerate}
\setcounter{enumi}{2}
\item
Let $f(x)$ be a polynomial with coefficients from a field $F,$ and let
$\mathsf{T}$ be a linear operator on a vector space $\mathsf{V}$ over
$F$. Then the following statements are true
\begin{enumerate}
\item $f(\mathsf{T})$ is a linear operator on $\mathsf{V}.$
\item If $\beta$ is a finite ordered basis for $\mathsf{V}$ and $A =
  [\mathsf{T}]_\beta,$ then $[f(\mathsf{T})]_\beta = f(A).$
\end{enumerate}

Suppose $f(x)$ is a polynomial with coefficients in $F$

Suppose $\mathsf{T} \in \mathcal{V}$ and $\mathsf{V}$ is a vector
space over $F.$
\begin{enumerate}
\item Claim: $f(\mathsf{T})$ is a linear operator on $\mathsf{V}.$
\begin{align}
f(x) &= a_nx^n + a_{n-1}x^{n-1} + \dotsb + a_1x + a_0 \\
f(\mathsf{T}) &= a_n\mathsf{T}^n + a_{n-1}\mathsf{T}^{n-1} + \dotsb +
a_1\mathsf{T} + a_0\mathsf{I}_\mathsf{V}
\end{align}
\paragraph{Lemma: } $\mathsf{T}^n \in \mathcal{L}(\mathsf{V})\; \forall
n \in \mathbb{Z}^+.$

Suppose $y,z \in \mathsf{V}.$

Suppose $n=1$
\begin{equation}
\mathsf{T}(az+y) = a\mathsf{T}(z) +\mathsf{T}(y)
\end{equation}
Suppose true for $1\leq n\leq k.$

Suppose $n=k+1$
\begin{align}
\mathsf{T}^{k+1}(az+y) &= \mathsf{T}(\mathsf{T}^k(az+y))\\
&= \mathsf{T}(a\mathsf{T}^k(z) + \mathsf{T}(y))\\
&= a\mathsf{T}^{k+1}(z) + \mathsf{T}^{k+1}(y) \\ & & \qedsymbol\notag
\end{align}
\begin{multline}
\therefore f(\mathsf{T}(az+y)) = a_n\mathsf{T}^n(az+y)
+a_{n-1}\mathsf{T}^{n-1}(az+y) +\dotsb + \\  {}+a_1\mathsf{T}(az +y) +
a_0(az+y)
\end{multline}
\begin{multline}
f(\mathsf{T}(az+y)) = a_n(a\mathsf{T}^n(z) + \mathsf{T}^n(y)) +
a_{n-1}(a\mathsf{T}^{n-1}(z) +\mathsf{T}^{n-1}(y)) + \dotsb + \\ {} +
a_1(a\mathsf{T}(z) +\mathsf{T}(y)) +a_0(az +y)
\end{multline}
\begin{multline}
f(\mathsf{T}(az+y)) = a(a_n\mathsf{T}^n(z)+a_{n-1}\mathsf{T}^{n-1}(z)
+ \dotsb + a_1\mathsf{T}(z) +a_oz) + \\
+ (a_n\mathsf{T}^n(y)+a_{n-1}\mathsf{T}^{n-1}(y)
+ \dotsb + a_1\mathsf{T}(y) +a_oy)
\end{multline}
\begin{equation}
f(\mathsf{T}(az+y)) = af(\mathsf{T})(z) + f(\mathsf{T})(y) \in \mathsf{V}
\end{equation}
\begin{equation}
\implies f(\mathsf{T}) \in \mathcal{L}(\mathsf{V})
\end{equation}
\item Claim: If $\beta$ is a finite ordered basis for $\mathsf{V}$ and
  $A=[\mathsf{T}]_\beta,$ then $\left[f(\mathsf{T})\right]_\beta = f(A).$
\begin{equation}
f(\mathsf{T}) = a_n\mathsf{T}^n +a_{n-1}\mathsf{T}^{n-1}+\dotsb
+a_1\mathsf{T} +a_0\mathsf{I}_\mathsf{V}
\end{equation}
\begin{align}
\implies\left[f(\mathsf{T})\right]_\beta &= \left[a_n\mathsf{T}^n +a_{n-1}\mathsf{T}^{n-1}+\dotsb
+a_1\mathsf{T} +a_0\mathsf{I}_\mathsf{V}\right]_\beta\\
&= a_n\left[\mathsf{T}^n\right]_\beta
a_{n-1}\left[\mathsf{T}^{n-1}\right]_\beta + \dotsb +
a_1\left[\mathsf{T}\right]_\beta
  +a_0\left[\mathsf{I}_\mathsf{V}\right]_\beta\\
&= a_n\left(\left[\mathsf{T}\right]_\beta\right)^n
+a_{n-1}\left(\left[\mathsf{T}\right]_\beta\right)^{n-1} + \dotsb +
a_1\left[\mathsf{T}\right]_\beta + a_0\mathsf{I}_\mathsf{V}\\
&= a_nA^n +a_{n-1}A^{n-1} +\dotsb + a_1A +a_0\\
&= f(A)
\end{align}
\end{enumerate}

\setcounter{enumi}{4}
\item
Let $\mathsf{T}$ be a linear operator on a vector space $\mathsf{V}$
over a field $F,$ and let $A$ be an $n\times n$ matrix with entries
from $F.$ If $f_1(x)$ and $f_2(x)$ are relatively prime polynomials
with entries from $F,$ then there exist polynomials $q_1(x)$ and
$q_2(x)$ with entries from $F$ such that
\begin{enumerate}
\item $q_1(\mathsf{T})f_1(\mathsf{T}) + q_2(\mathsf{T})f_2(\mathsf{T})
  = \mathsf{I}$
\item$q_1(A)f_1(A) + q_2(A)f_2(A) = I.$
\end{enumerate}
Suppose $\mathsf{T} \in \mathcal{L}(\mathsf{V}),$ such that
$\mathsf{V}$ is a vector space over $F,$ and $A\in
\mathsf{M}_{n\times n}(F)$

Suppose $f_1(x), f_2(x) \in \mathsf{P}(F)$ such that they are
relatively prime.
\begin{enumerate}
\item Claim: $ \exists q_1(x)$ and $q_2(x)$ such that
  $q_1(\mathsf{T})f_1(\mathsf{T}) + q_2(\mathsf{T})f_2(\mathsf{T}) =
  \mathsf{I}$ 
\paragraph{}
Because $f_1(x)$ and $f_2(x)$ are relatively prime there exists
polynomials $q_1(x)$ and $q_2(x)$ such that 
\begin{equation}
f_1(x)q_1(x) + f_2(x)q_2(x) = 1
\end{equation}
It follows that 
\begin{equation}
f_1(\mathsf{T})q_1(\mathsf{T}) + f_2(\mathsf{T})q_2(\mathsf{T}) = \mathsf{I}_\mathsf{V}
\end{equation}
\item Claim: $\exists q_1(x)$ and $q_2(x)$ such that $q_1(A)f_1(A) +
  q_2(A)f_2(A) = I_n$
\begin{gather}
f_1(x)q_1(x) + f_2(x)q_2(x) = 1\\
\implies f_1(A)q_1(A) + f_2(A)q_2(A) = I_n
\end{gather}
\end{enumerate}

\newpage{}
\item
Let $\phi(x)$ and $f(x)$ be polynomials. If $\phi(x)$ is irreducible
and $\phi(x)$ does not divide $f(x),$ then $\phi(x)$ and $f(x)$ are
relatively prime.

Claim: Let $\phi(x)$ and $f(x)$ be polynomials. If $\phi(x)$ is
irreducible, and $\phi(x)$ does not divide $f(x)$, then $\phi(x)$ and
$f(x)$ are relatively prime.

\paragraph{}
Because $\phi(x)$ is irreducible, $f(x)$ does not divide
$\phi(x)$. Since $\phi(x)$ does not divide $f(x)$ it follows that
$\phi(x)$ and $f(x)$ are relatively prime.

\item
Any two distinct irreducible monic polynomials are relatively prime.

\paragraph{Lemma: } All factors of an irreducible monic polynomial
$\phi(x)$ are either of the form $c \neq 0, c \in F$ of $d\phi(x), d
\neq 0, d \in F$

Suppose $f(x), \phi(x) \in \mathsf{P}(F)$ and $\phi(x)$ is an
irreducible polynomial.

Suppose $f(x)$ divides $\phi(x)$
\begin{equation}
\implies \phi(x) = f(x)g(x)\quad \text{for some } q(x) \in
\mathsf{P}(F)
\end{equation}
\begin{enumerate}[{\bf {Case} 1}]
\item $\text{deg}(f(x)) \notin \mathbb{Z}^+$
\begin{gather}
f(x)\neq 0 \because \phi(x) \neq 0\\
\implies \text{deg}(f(x)) = 0 \\
\implies f(x) = c \quad \text{for some } c \in F
\end{gather}
\item$\text{deg}(f(x)) \in \mathbb{Z}^+$
\begin{equation}
\phi(x) = f(x)q(x)
\end{equation}
Because $\phi(x)$ is irreducible, it cannot be expressed as a product
of polynomials both possessing positive degree.
\begin{gather}
\implies \text{deg}(q(x)) \leq 0 \\
\implies g(x) = \frac{1}{d} \quad \text{for some nonzero } d \in F\\
\implies \phi(x) = \frac{f(x)}{d}\\
\implies d\phi(x) = f(x)
\end{gather}
\begin{align}
\notag & & & & \qedsymbol
\end{align}
\newpage{}
By lemma, suppose the factors of $\phi_1$ are $c$ and $d\phi_1$ where
$c,d \in F$ and $c,d\neq 0.$ suppose the factors of $\phi_2$ are $e$
and $g\phi_2$ where $e,g \in F, e,g \neq 0$

Claim: $d\phi_1 \neq g\phi_2$

Suppose $g\phi_2 | \phi_1$
\begin{gather}
\implies g\phi_2 = d\phi_1\\
\implies \phi_2 = \frac{d}{g}\phi_1
\implies \phi_2 = \frac{d}{g}(x^n +a_{n-1}x^{n-1} + \dotsb +a_1x + a_0)\\
\implies \phi_2 = \frac{d}{g}x^n +\frac{d}{g}a_{n-1}x^{n-1} + \dotsb +
\frac{d}{g}a_1x + a_0\\
\implies \frac{d}{g} = 1\text{ because } \phi_2 \text{ is monic.} \\
\implies \phi_2 = \phi_1\; \lightning \text{ Contradiction!} \\
\text{Theorem states } \phi_1 \text{ and } \phi_2 \text{ are distinct
  polynomials.} \notag
\end{gather}
\end{enumerate}


\end{enumerate}
\newpage{}
\subsection*{5.2}
\begin{enumerate}
\setcounter{enumi}{1}
\item
Use Gaussian elimination to solve the following systems of linear
equations.
\begin{enumerate}
\setcounter{enumii}{5}
\item 
\begin{align*}
x_1 + 2x_2 - x_3 + 3x_4 &= 2\\
2x_1 + 4x_2 - x_3 + 6x_4 &= 5\\
       x_2    +2x_4 &= 3
\end{align*}
\begin{gather}
\begin{gmatrix}[left]
1 & 2 & -1 & 3 &\\
2 & 4 & -1 & 6 &\\
0 & 1 & 0  & 2 &
\end{gmatrix}
\begin{gmatrix}[right]
2\\
5\\
3
\rowops
\add[-2]{0}{1}
\add[-2]{2}{0}
\add[1]{1}{0}
\swap{1}{2}
\end{gmatrix}
\leadsto
\begin{gmatrix}[left]
1& 0 & 9 & -4 &\\
0 & 1 & 0 & 2 &\\
0 & 0 & 1 & 0 &
\end{gmatrix}
\begin{gmatrix}[right]
-3\\
3\\
1
\end{gmatrix}
\end{gather}
\begin{align}
x_1 &= -3 +4x_4\\
x_2 &= 3 -2x_4\\
x_3 &= 1\\
x_4 &= x_4
\end{align}
\begin{equation}
S =  \left\{\begin{pmatrix}-3\\2\\1\\0\end{pmatrix} +
  z\begin{pmatrix}4\\-2\\0\\1\end{pmatrix} \colon z \in \mathbb{R}\right\}
\end{equation}
\setcounter{enumii}{9}
\item \begin{align}
2x_1 + 3x_3 -4x_5 &= 5\\
3x_1-4x_2 +8x_3+3x_4 &=8\\
x_1-x_2+2x_3+x_4-x_5&=2\\
-2x_1 + 5x_2 -9x_3 -3x_4 -5x_5 &= -8
  \end{align}
\begin{equation}
\begin{gmatrix}[left]
2 & 0 & 3 & 0 & -4 &\\
3 & -4 & 8 & 3 & 0 &\\
1 & -1 & 2 & 1 & -1 &\\
-2 & 5 & -9 &-3 & -5 &
\end{gmatrix}
\begin{gmatrix}[right]
5\\
8\\
2\\
-8
\end{gmatrix}
\leadsto
\begin{gmatrix}[left]
1 & 0 & 0 & 0 & -2 &\\
0 & 1 & 0 & 0 & -3 &\\
0 & 0 & 1 & 0 &  0 &\\
0 & 0 & 0 & 1 & -2
\end{gmatrix}
\begin{gmatrix}[right]
1\\
0\\
-1\\
-1
\end{gmatrix}
\end{equation}
\begin{align}
x_1 &= 2x_5 +1\\
x_2 &= 3x_5\\
x_3 &= -1\\
x_4 &= 2x_5 -1\\
x_5 &= x_5
\end{align}
\begin{equation}
S = \left\{\begin{pmatrix}1\\0\\-1\\-1\\0\end{pmatrix} +
  z\begin{pmatrix}2\\3\\0\\2\\1\end{pmatrix}\colon x \in \mathbb{R}\right\}
\end{equation}
\end{enumerate}

\newpage{}
\item
For each of the following linear operators $\mathsf{T}$ on a vector
space $\mathsf{V}$, test $\mathsf{T}$ for diagonalizability, and if
$\mathsf{T}$ is diagonalizable, find a basis $\beta$ for $\mathsf{V}$
such that $[\mathsf{T}]_\beta$ is a diagonal matrix.
\begin{enumerate}
\item $\mathsf{V}=\mathsf{P}_3(\mathbb{R})$ and $\mathsf{T}$ is
  defined by $\mathsf{T}(f(x)) = f^\prime(x) + f^{\prime\prime}(x),$
  respectively.

Let $\alpha$ be the standard ordered basis of
$\mathsf{P}_3(\mathbb{R})$.
\begin{gather}
\mathsf{T}(\alpha) = \{0,1,2x+2,3x^2+6x\}\\
\implies [\mathsf{T}]_\alpha = \begin{pmatrix}
0 & 1 & 2 & 0\\
0 & 0 & 2 & 6\\
0 & 0 & 0 & 3\\
0 & 0 & 0 & 0
\end{pmatrix}\\
\det{([\mathsf{T}]_\beta -\lambda I_4)} = \det{\begin{pmatrix}
-\lambda & 1 & 2 & 0\\
0 & -\lambda & 2 & 6\\
0 & 0 & -\lambda & 3\\
0 & 0 & 0 & -\lambda
    \end{pmatrix}} = \lambda^4\\
\implies \lambda_1 = 0 \text{ with multiplicity 4}\\
\rank{[\mathsf{T}]_\beta - \lambda_1I_3} = 3\\
4-3\neq 4 \text{ multiplicity of } \lambda_1 \\
\notag \implies \mathsf{T} \text{ is not diagonalizable}
\end{gather}
\setcounter{enumii}{3}
\item $\mathsf{V} = \mathsf{P}_2(\mathbb{R})$ and $\mathsf{T}$ is
  defined by $\mathsf{T}(f(x)) = f(0) +f(1)(x+x^2)$.

Suppose $\alpha$ is the standard ordered basis of
$\mathsf{P}_2(\mathbb{R})$.

\begin{gather}
\implies \mathsf{T}(\alpha) = \{x^2+x+1,x+x^2,x+x^2 \}\\
\implies [\mathsf{T}]_\alpha = \begin{pmatrix}
1 & 0 & 0\\
1 & 1 & 1 \\
1 & 1 & 1
\end{pmatrix}
\end{gather}
\begin{align}
\det{([\mathsf{T}]_\beta -\lambda I_2)} &= \det{\begin{pmatrix}
1-\lambda & 0 & 0\\
1 & 1-\lambda & 0\\
1 & 1 & 1-\lambda
  \end{pmatrix}}\\
&= (\lambda)(1-\lambda)(\lambda - 2)
\end{align}
\begin{align}
\implies \lambda_1 &= 0, \text{ multiplicity } 1\\
\lambda_2 &= 1, \text{ multiplicity } 1\\
\lambda_3 &= 2, \text{ multiplicity } 1
\end{align}
\begin{itemize}
\item For $\lambda_1 = 0$
\begin{gather}
[\mathsf{T}]_\beta -0 I_3= \begin{pmatrix}
1 & 0 & 0\\
1 & 1 & 1\\
1 & 1 & 1
\end{pmatrix}\\
\begin{pmatrix}
1 & 0 & 0\\
1 & 1 & 1\\
1 & 1 & 1
\end{pmatrix}\leadsto
\begin{pmatrix}
1 & 0 & 0\\
0 & 1 & 0\\
0 & 1 & 0
\end{pmatrix}\\
\implies \rank{[\mathsf{T}]_\beta} = 2\\
\text{multiplicity } \lambda_1 =1; \; 3-2 =1 \checkmark
\end{gather}
\item For $\lambda_2 = 1$
\begin{gather}
[\mathsf{T}]_\beta - I_3 = \begin{pmatrix}
0 & 0 & 0\\
1 & 0 & 1\\
1 & 1 & 0
\end{pmatrix}\\
\begin{pmatrix}
0 & 0 & 0\\
1 & 0 & 1\\
1 & 1 & 0
\end{pmatrix}\leadsto
\begin{pmatrix}
0 & 0 & 0\\
0 & 0 & 1\\
0 & 1 & 0
\end{pmatrix}\\
\implies \rank{[\mathsf{T}]_\beta -I_3} = 2 \\
\text{multiplicity } \lambda_2 =1; \; 3-2 =1 \checkmark
\end{gather}
\item For $\lambda_3 = 2$
\begin{gather}
[\mathsf{T}]_\beta - 2 I_3 = \begin{pmatrix}
-1 & 0 & 0\\
1 & -1 & 1\\
1 & 1 & 1
\end{pmatrix}\leadsto
\begin{pmatrix}
-1 & 0 & 0\\
0  & 0 & 1\\
2 & 0 & -1
\end{pmatrix}\\
\implies \rank{[\mathsf{T}]_\beta} =2\\
\text{multiplicity } \lambda_3 =1; \; 3-2 =1 \checkmark
\end{gather}
It follows that $\mathsf{T}$ is diagonalizable.
\end{itemize}
\begin{itemize}
\item For $\lambda_1 = 0$
\begin{equation}
\begin{pmatrix}
1 & 0 & 0\\
1 & 1 & 1 \\
1 & 1 & 1
\end{pmatrix}
\begin{pmatrix}
x_1\\x_2\\x_3
\end{pmatrix}
=
\begin{pmatrix}
0\\0\\0
\end{pmatrix}
\leadsto
\begin{pmatrix}
1 & 0 & 0\\
0 & 1 & 1\\
0 & 0 & 0
\end{pmatrix}
\begin{pmatrix}
x_1\\x_2\\x_3
\end{pmatrix}
=
\begin{pmatrix}
0\\0\\0
\end{pmatrix}
\end{equation}
\begin{gather}
\implies x_1 = 0\\
x_2 = -x_3\\
S_1 = \left\{z\begin{pmatrix}0\\-1\\1\end{pmatrix} \colon z \in
  \mathbb{R}\right \}\\
\notag \implies \begin{pmatrix}0\\-1\\1\end{pmatrix} \text{ is the
  eigenvector corresponding to } \lambda_2
\end{gather}
\item For $\lambda_2 = 1$
\begin{equation}
\begin{pmatrix}
0 & 0 & 0\\
1 & 0 & 1\\
1 & 1 & 0
\end{pmatrix}
\begin{pmatrix}
x_1\\x_2\\x_3
\end{pmatrix}
=
\begin{pmatrix}
0\\0\\0
\end{pmatrix}
\leadsto
\begin{pmatrix}
0 & 0 & 0\\
1 & 0 & 1\\
0 & 1 & -1
\end{pmatrix}
\begin{pmatrix}
x_1\\x_2\\x_3
\end{pmatrix}
=
\begin{pmatrix}
0\\0\\0
\end{pmatrix}
\end{equation}
\begin{gather}
\implies x_1 =- x_3\\
x_2 = x_3\\
S_2 = \left\{ z\begin{pmatrix}-1\\1\\1\end{pmatrix} \colon z \in
  \mathbb{R}\right\}\\
\notag \implies \begin{pmatrix}-1\\1\\1\end{pmatrix} \text{ is the
  eigenvector corresponding to } \lambda_2
\end{gather}
\item For $\lambda_3 = 2$
\begin{equation}
\begin{pmatrix}
-1 & 0 & 0\\
1 & -1 & 1\\
1 & 1 & -1
\end{pmatrix}
\begin{pmatrix}
x_1\\x_2\\x_3
\end{pmatrix}
=
\begin{pmatrix}
0\\0\\0
\end{pmatrix}
\leadsto
\begin{pmatrix}
-1 & 0 & 0\\
0 & 0 & 0\\
0 & 1 & -1
\end{pmatrix}
\begin{pmatrix}
x_1\\x_2\\x_3
\end{pmatrix}
=
\begin{pmatrix}
0\\0\\0
\end{pmatrix}
\end{equation}
\begin{align}
\implies x_1 &= 0\\
x_2 &= x_3\\
\end{align}
\begin{gather}
S_3 = \left\{z\begin{pmatrix}0\\1\\1\end{pmatrix}\colon z \in
  \mathbb{R}\right\}\\
\implies \begin{pmatrix}0\\1\\1\end{pmatrix} \text{ is the eigenvector
  corresponding to } \lambda_2\\
\implies \beta = \left\{\begin{pmatrix}0\\-1\\1
  \end{pmatrix},\begin{pmatrix}-1\\1\\1\end{pmatrix},\begin{pmatrix}0\\1\\1\end{pmatrix}\right\}\\
\implies [\mathsf{T}]_\beta = \begin{pmatrix}
0 & 0 & 0\\
0 & 1 & 0\\
0 & 0 & 2
\end{pmatrix}
\end{gather}
\end{itemize}
\item $\mathsf{V} = \mathsf{C}^2$ and $\mathsf{T}$ is defined by
  $\mathsf{T}(z,w) = (z+iw,iz+2)$

Suppose $\alpha = \left\{(1,0),(-,1) \right\}$ is a basis for $\mathsf{C}^2$
\begin{gather}
\mathsf{T}(\alpha) = \left\{(1,i),(i,1)\right \}\\
\implies [\mathsf{T}]_\alpha = \begin{pmatrix}
1 & i\\
i & 1
\end{pmatrix}
\end{gather}
\begin{align}
\det{([\mathsf{T}]_\beta -\lambda I_2)} &= \det{\begin{pmatrix}
1-\lambda & i\\
i & 1-\lambda 
\end{pmatrix}
}\\
&= \lambda^2-2\lambda +2
\end{align}
$\mathbb{C}$ is algebraically closed so the characteristic polynomial
splits over $\mathbb{C}$
\begin{gather}
\lambda^2-2\lambda +2 =0 \\
\implies 1\pm i
\end{gather}
\begin{align}
\implies \lambda_1 &= 1+i, \text{ multiplicity } 1\\
\lambda_2 &= 1-i, \text{ multiplicity } 1 
\end{align}
\begin{itemize}
\item For $\lambda_1 = 1 +i$
\begin{gather}
[\mathsf{T}]_\beta -(i+1)I_2 = \begin{pmatrix}
-i & i\\
i & -i
\end{pmatrix}\\
\begin{pmatrix}
-i & i\\
i & -i
\end{pmatrix}
\leadsto
\begin{pmatrix}
0 & 1\\
0 & -1
\end{pmatrix}\\
\rank{[\mathsf{T}]_\beta -(i+1)I_2} =1\\
\text{multiplicity } \lambda_1 =i+1; \; 2-1 =1 \checkmark
\end{gather}
\item For $\lambda_2 = 1 -i$
\begin{gather}
[\mathsf{T}]_\beta - (1-i)I_2 = \begin{pmatrix} i & i \\ i &
  i\end{pmatrix}\\
\begin{pmatrix}
 i & i \\
 i & i
\end{pmatrix}
\leadsto
\begin{pmatrix}
1 & 0\\
1 & 0 
\end{pmatrix}\\
\implies \rank{[\mathsf{T}]_\beta-(1-i)I_2}=1\\
\text{multiplicity } \lambda_2 =1-i; \; 2-1 =1 \checkmark\\
\end{gather}
\end{itemize}
It follows that $\mathsf{T}$ is diagonalizable.
\begin{itemize}
\item For $\lambda_1 = 1 +i$
\begin{gather}
\begin{pmatrix}
-i & i\\
i & -1
\end{pmatrix}
\begin{pmatrix}
x_1\\x_2\\x_3
\end{pmatrix}
=
\begin{pmatrix}
0\\0\\0
\end{pmatrix}
=
\begin{pmatrix}
1 & -1\\
0 & 0
\end{pmatrix}
\begin{pmatrix}
x_1\\x_2\\x_3
\end{pmatrix}
=
\begin{pmatrix}
0\\0\\0
\end{pmatrix}\\
\implies x_1 = x_2\\
S_1 = \left\{z\begin{pmatrix}1\\1\end{pmatrix}\colon x \in
  \mathbb{C}\right\}\\
\notag \begin{pmatrix}1\\1\end{pmatrix} \text{ is the eigenvector
  corresponding to } \lambda_1
\end{gather}
\item For $\lambda_2 = 1 -i$
\begin{gather}
\begin{pmatrix}
i & i\\
i & i
\end{pmatrix}
\begin{pmatrix}
x_1\\x_2\\x_3
\end{pmatrix}
=
\begin{pmatrix}
0\\0\\0
\end{pmatrix}
\leadsto
\begin{pmatrix}
1 & 1\\
0 & 0
\end{pmatrix}
\begin{pmatrix}
x_1\\x_2\\x_3
\end{pmatrix}
=
\begin{pmatrix}
0\\0\\0
\end{pmatrix}\\
\implies x_1 = -x_2 \\
S_2 = \left\{z\begin{pmatrix}-1\\1\end{pmatrix}\colon z\in
  \mathbb{C}\right\}
\\\notag \begin{pmatrix}-1\\1\end{pmatrix} \text{ is the eigenvector
  corresponding to } \lambda_2\\
\beta
=\left\{\begin{pmatrix}1\\1\end{pmatrix},\begin{pmatrix}-1\\1\end{pmatrix}\right\}\\
\implies [\mathsf{T}]_\beta =
\begin{pmatrix}
1+i & 0\\
0 & 1-i
\end{pmatrix}
\end{gather}
\end{itemize}
\item $\mathsf{V} = \mathsf{M}_{n \times n}(\mathbb{R})$ and
  $\mathsf{T}$ is defined by $\mathsf{T}(A) = A^t$

Suppose $\alpha$ is the standard ordered basis of $\mathsf{M}_{n\times
  n}(\mathbb{R})$
\begin{gather}
\implies \mathsf{T}(\alpha) =\left\{\begin{pmatrix}1&0\\0 &
    0\end{pmatrix},\begin{pmatrix}0&1\\0& 0\end{pmatrix},\begin{pmatrix}0
    & 0 \\1&0\end{pmatrix},\begin{pmatrix}0& 0
    \\0&1\end{pmatrix}\right\}\\
\implies [\mathsf{T}]_\alpha = \begin{pmatrix}
1 & 0 & 0 & 0\\
0 & 0 & 1 & 0\\
0 & 1 & 0 & 0\\
0 & 0 & 0 & 1
\end{pmatrix}
\end{gather}
\begin{align}
\implies \det{([\mathsf{T}]_\beta-\lambda I_4)} &= \det{
\begin{pmatrix}
1-\lambda & 0 & 0 & 0\\
0 & -\lambda & 1 &  0\\
0 & 1 & -\lambda & 0\\
0 & 0 &0 & 1-\lambda
\end{pmatrix}
}\\
&=(1-\lambda^2)(\lambda^2-1)\\
\end{align}
\begin{align}
\implies \lambda_1 &= 1, \text{ multiplicity } 3\\
\implies \lambda_2 &= -1, \text{ multiplicity } 1
\end{align}
\begin{itemize}
\item For $\lambda_1 = 1$
\begin{gather}
[\mathsf{T}]_\beta -I_4 =
\begin{pmatrix}
0 & 0 & 0 & 0\\
0 & -1 & 1 & 0\\
0 & 1 & -1 & 0\\
0 & 0 & 0 & 0
\end{pmatrix}\\
\begin{pmatrix}
0 & 0 & 0 & 0\\
0 & -1 & 1 & 0\\
0 & 1 & -1 & 0\\
0 & 0 & 0 & 0
\end{pmatrix}\leadsto
\begin{pmatrix}
0 & 0 & 0 & 0\\
0 & 0 & 1 & 0\\
0 & 0 & -1 & 0\\
0 & 0 & 0 & 0
\end{pmatrix}\\
\implies \rank{[\mathsf{T}]_\beta -I_4} = 1\\
\text{multiplicity } \lambda_1 =3; \; 4-1 =3 \checkmark
\end{gather}
\item For $\lambda_2 = -1$
\begin{gather}
[\mathsf{T}]_\beta +I_4 = 
\begin{pmatrix}
2 & 0 & 0 & 0\\
0 & 1 & 1 & 0\\
0 & 1 & 1 & 0\\
0 & 0 & 0 & 2
\end{pmatrix}\\
\begin{pmatrix}
2 & 0 & 0 & 0\\
0 & 1 & 1 & 0\\
0 & 1 & 1 & 0\\
0 & 0 & 0 & 2
\end{pmatrix}
\leadsto
\begin{pmatrix}
2 & 0 & 0 & 0\\
0 & 0 & 1 & 0\\
0 & 0 & 1 & 0\\
0 & 0 & 0 & 2
\end{pmatrix}\\
\implies \rank{[\mathsf{T}]_\beta +I_4} =3\\
\text{multiplicity } \lambda_2 =1; \; 4-3 =1 \checkmark
\end{gather}
\end{itemize}
It follows that $\mathsf{T}$ is diagonalizable.
\begin{itemize}
\item For $\lambda_1 = 1$
\begin{equation}
\begin{pmatrix}
0 & 0 & 0 & 0\\
0 & -1 & 1 & 0\\
0 & 1 & -1 & 0\\
0 & 0 & 0 & 0
\end{pmatrix}
\begin{pmatrix}
x_1\\x_2\\x_3\\x_4
\end{pmatrix}
=\begin{pmatrix}
0\\0\\0\\0
\end{pmatrix}\\
\begin{pmatrix}
0 & 0 & 0 & 0\\
0 & 0 & 0 & 0\\
0 & 1 & -1 & 0\\
0 & 0 & 0 & 0
\end{pmatrix}
\begin{pmatrix}
x_1\\x_2\\x_3\\x_4
\end{pmatrix}
=\begin{pmatrix}
0\\0\\0\\0
\end{pmatrix}
\end{equation}
\begin{align}
\implies x_1 &= x_1\\
x_2 &=x_3\\
x_4 &= x_4
\end{align}
\begin{equation}
\implies S_1 =
\left\{z_1\begin{pmatrix}1&0\\0&0\end{pmatrix}+z_2\begin{pmatrix}0 &
    1\\1&0\end{pmatrix} +z_3\begin{pmatrix}0 & 0\\0 &
    1\end{pmatrix}\colon z_1,z_2,z_3 \in \mathbb{R}\right\}
\end{equation}
\begin{gather}
\implies \begin{pmatrix}1&0\\0&0\end{pmatrix},\begin{pmatrix}0 &
    1\\1&0\end{pmatrix},\begin{pmatrix}0 & 0\\0 &
    1\end{pmatrix} \text{ are eigenvectors corresponding to } \lambda_1\notag
\end{gather}
\item For $\lambda_2 = -1$
\begin{equation}
\begin{pmatrix}
2 &  0 & 0 & 0\\
0 & 1 & 1 &0\\
0 & 1 & 1 &0\\
0 & 0 & 0 & 2
\end{pmatrix}
\begin{pmatrix}
x_1\\x_2\\x_3\\x_4
\end{pmatrix}
=\begin{pmatrix}
0\\0\\0\\0
\end{pmatrix}
\leadsto
\begin{pmatrix}
1 & 0 & 0 & 0\\
0 & 1 & 1 & 0\\
0 & 0 & 0 & 0\\
0 & 0 & 0 & 1
\end{pmatrix}
\begin{pmatrix}
x_1\\x_2\\x_3\\x_4
\end{pmatrix}
=\begin{pmatrix}
0\\0\\0\\0
\end{pmatrix}
\end{equation}
\begin{align}
x_1 &= x_4 =0 \\
x_2 &= -x_3
\end{align}
\begin{equation}
\implies S_2 = \left\{z\begin{pmatrix}0 & -1\\1 & 0\end{pmatrix}\right\}
\end{equation}
\end{itemize}
\begin{gather}
\beta = \left\{\begin{pmatrix}1 & 0\\0 &
    0\end{pmatrix},\begin{pmatrix}0 & 1\\1
    &0\end{pmatrix},\begin{pmatrix}0 & 0 \\0 &
    1\end{pmatrix},\begin{pmatrix}0 & -1 \\1 &
    0\end{pmatrix}\right\}\\
\implies [\mathsf{T}]_\beta =
\begin{pmatrix}
1 & 0 & 0 & 0\\
0 &1 & 0 & 0\\
0 & 0 & 1 & 0\\
0 & 0 & 0 & -1
\end{pmatrix}
\end{gather}
\end{enumerate}

\setcounter{enumi}{6}
\item
Let $\mathsf{T}$ be a linear operator on a finite-dimensional vector
space $\mathsf{V}$. We define the {\bf determinant} of $\mathsf{T}$,
denoted $\det{(\mathsf{T})}$, as follows: Choose any ordered basis
$\beta$ for $\mathsf{V}$, and define $\det{(\mathsf{T})}=
\det{([\mathsf{T}]_\beta )}$.
\begin{enumerate}
\item Prove that he preceding definition is independent of the choice
  of an ordered basis for $\mathsf{V}$. That is, prove that if $\beta$
  and $\gamma$ are two ordered bases for $\mathsf{V}$, then $\det{([\mathsf{T}]_\beta)}=\det{([\mathsf{T}]_\gamma)}$.
\item Prove that $\mathsf{T}$ is invertible if and only if
  $\det{\mathsf{T}}\neq 0$.
\item Prove that if $\mathsf{T}$ is invertible, then
  $\det{(\mathsf{T}^{-1})}= [\det{(\mathsf{T})}]^{-1}$.
\item Prove that if $\mathsf{U}$ is also a linear operator on
  $\mathsf{V}$, then $\det{(\mathsf{TU})} = \det{(\mathsf{T})}\cdot\det{(\mathsf{U})}$.
\item Prove that $\det{(\mathsf{T}-\lambda \mathsf{I}_\mathsf{V})} =
    \det{[\mathsf{T}]_\beta -\lambda I)}$ for any scalar $\lambda$ and
      any ordered basis $\beta$ for $\mathsf{V}$.
\end{enumerate}
\begin{enumerate}
\item
Suppose $Q$ is the change of coordinates matrix from $\gamma$ to
$\beta$.
\begin{gather}
Q=[\mathsf{I}_\mathsf{V}]_\gamma^\beta \implies
Q^{-1}=[\mathsf{I}_\mathsf{V}]^\gamma_\beta \\
[\mathsf{I}_\mathsf{V}]_\gamma^\beta
[\mathsf{T}]_\gamma[\mathsf{I}_\mathsf{V}]^\gamma_\beta =
[\mathsf{T}]_\beta\\
\implies \det{([\mathsf{I}_\mathsf{V}]_\gamma^\beta
[\mathsf{T}]_\gamma[\mathsf{I}_\mathsf{V}]^\gamma_\beta)} =
\det{[\mathsf{T}]_\beta}\\
\implies\det{[\mathsf{I}_\mathsf{V}]_\gamma^\beta}
\det{[\mathsf{T}]_\gamma}\det{[\mathsf{I}_\mathsf{V}]^\gamma_\beta}\\
[\mathsf{I}_\mathsf{V}]_\gamma^\beta =
([\mathsf{I}_\mathsf{V}]^\gamma_\beta)^{-1}\\
\det{[\mathsf{I}_\mathsf{V}]_\gamma^\beta}=
\det{([\mathsf{I}_\mathsf{V}]^\gamma_\beta)^{-1}}
\end{gather}
\begin{align}
\implies \det{[\mathsf{I}_\mathsf{V}]_\gamma^\beta}
\det{[\mathsf{T}]_\gamma}\det{[\mathsf{I}_\mathsf{V}]^\gamma_\beta} &=
\det{([\mathsf{I}_\mathsf{V}]^\gamma_\beta)^{-1}}\det{[\mathsf{T}]_\gamma}\det{[\mathsf{I}_\mathsf{V}]^\gamma_\beta}\\
&= \det[\mathsf{T}]_\gamma = \det[\mathsf{T}]_\beta
\end{align}
\item 
($\implies$) 

Suppose $\mathsf{T}$ is invertible\\
Suppose $\beta$ is an ordered basis of $\mathsf{V}$.
\begin{gather}
[\mathsf{I}_\mathsf{V}]_\beta = [\mathsf{T}\cdot\mathsf{T}^{-1}]_\beta
= [\mathsf{T}]_\beta[\mathsf{T}^{-1}]_\beta\\
\implies \det{[\mathsf{I}_\mathsf{V}]_\beta} =
\det{[\mathsf{T}]_\beta}\det{[\mathsf{T}^{-1}]_\beta}\\
\det{[\mathsf{I}_\mathsf{V}]_\beta} =1 \\
\implies \det{[\mathsf{T}]_\beta}\det{[\mathsf{T}^{-1}]_\beta} =1\\
\implies \det{[\mathsf{T}]_\beta} \neq 0\\
\implies \det{\mathsf{T}} \neq 0
\end{gather}
($\Leftarrow$)

Suppose $\det{\mathsf{T}}\neq 0$ 
\begin{gather}
\implies \det[\mathsf{T}]_\beta \neq 0 \quad \text{for some ordered
  basis } \beta \text{ of } \mathsf{V}\\
%\det{[\mathsf{T}]_\beta} \neq  0\\
\implies [\mathsf{T}]_\beta \text{ is invertible, by corollary to Th. 4.7}\\ 
\implies \mathsf{T} \text{ is invertible, by corollary to Th. 2.18}
\end{gather}
\item Suppose $\mathsf{T}$ is invertible and $\beta$ is some ordered
  basis of $\mathsf{V}$.
\begin{align}
\det{\mathsf{I}_\mathsf{V}} &= \det{\mathsf{T}\cdot\mathsf{T}^{-1}}\\
&= \det{[\mathsf{T}\cdot\mathsf{T}^{-1}]_\beta}\\
&= \det{[\mathsf{T}]_\beta[\mathsf{T}^{-1}]_\beta}\\
&= \det{[\mathsf{T}]_\beta}\det{[\mathsf{T}^{-1}]_\beta}\\
&= \det{\mathsf{T}}\det{\mathsf{T}^{-1}}
\end{align}
\begin{gather}
\det{\mathsf{I}_\mathsf{V}} = \det{[\mathsf{I}_\mathsf{V}]_\beta} =
  \det{I_n} =1\\
\implies \det{\mathsf{T}} \det{\mathsf{T}^{-1}} = 1\\
\implies \det{\mathsf{T}^{-1}} = (\det{\mathsf{T}})^{-1}
\end{gather}
\item Suppose $\beta$ is an ordered basis of $\mathsf{V}$.
\begin{align}
\det{\mathsf{TU}} &= \det{[\mathsf{TU}]_\beta}\\
&= \det{[\mathsf{T}]_\beta[\mathsf{U}]_\beta} \text{ by Th. 2.11}\\
&= \det{[\mathsf{T}]_\beta}\det{[\mathsf{U}]_\beta} \text{ by Th. 4.7}\\
&= \det{\mathsf{T}}\det{\mathsf{U}}
\end{align}
\item 
\begin{align}
\det{([\mathsf{T}]_\beta -\lambda I)} &= \det{([\mathsf{T}]_\beta
  -\lambda[\mathsf{I}_\mathsf{V}]_\beta)}\\
&= \det{([\mathsf{T}]_\beta
  -[\lambda\mathsf{I}_\mathsf{V}]_\beta)} \text{ by Th. 2.8.b}\\
&= \det{[\mathsf{T}-\lambda \mathsf{I}_\mathsf{V}]_\beta} \text{ by
  Th. 2.8.a}\\
&= \det{(\mathsf{T}-\lambda \mathsf{I}_\mathsf{V})}
\end{align}
\end{enumerate}

\setcounter{enumi}{8}
\item
Let $\mathsf{T}$ be a linear operator on a finite-dimensional vector
space $\mathsf{V},$ and suppose there exists an ordered basis $\beta$
for $\mathsf{V}$ such that $[\mathsf{T}]_\beta$ is an upper triangular
matrix.
\begin{enumerate}
\item Prove that the characteristic polynomial for $\mathsf{T}$
  splits.
\item State and prove an analogous results for matrices.
\end{enumerate}
\begin{enumerate}
\item
\begin{equation}
[\mathsf{T}]_\beta \implies [\mathsf{T}]_\beta -\lambda I_n \text{ is
  upper triangular}
\end{equation}
\begin{align}
\det{([\mathsf{T}]_\beta -\lambda I_n)} &= \prod\limits_{i=1}^n
([\mathsf{T}]_\beta -\lambda I_n) \\
&= \prod\limits_{i=1}^n([\mathsf{T}]_\beta)_{ii} -\lambda\\
&= (a_{11}-\lambda)(a_{22}-\lambda)\dotsb(a_{nn}-\lambda) \text{ for }
a_{ii} \in F
\end{align}
It follows that $\mathsf{V}$ is $n$ dimensional, so the characteristic
polynomial splits over $F.$
\item 
Suppose $A\in \mathsf{M}_{n \times n}(F)$ is similar to an upper
triangular matrix. Prove the characteristic polynomial of $A$ splits.

Suppose $A = Q^{-1}UQ$ for some $U,Q \in \mathsf{M}_{n \times n}(F)$
such that $U$ is upper triangular and $Q$ is invertible. 
\begin{equation}
\implies \det{(A-\lambda I_n)} = \det{(U-\lambda I_n)}
\end{equation}
Because $U$ is upper triangular, $U0\lambda I_n$ is also upper
triangular.
\begin{align}
\det{(U-\lambda I_n)} &= \prod\limits^n_{i=1} (U-\lambda I_n)\\
&=\prod\limits^n_{i=1} (U_{ii} -\lambda)
\end{align}
It follows that the characteristic polynomial of $U$ splits. $U$ and
$A$ are similar, so $A$ and $U$ have the same characteristic
polynomial. Therefore the characteristic polynomial of $A$ splits.
\end{enumerate}

\setcounter{enumi}{10}
\item
Let $S =\left\{u_1,u_2,\dots,u_n\right\}$ be a linearly independent
  subset of a vector space $\mathsf{V}$ over the field $Z_2$. How many
  vectors are there in $\text{span}(S)$? Justify your answer.


\end{enumerate}
% \end{spacing}
\end{document}

