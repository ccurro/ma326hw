\documentclass[letterpaper,12pt]{article}
\usepackage{setspace}
\usepackage[utf8]{inputenc}
\usepackage[english]{babel}
\usepackage{enumerate}
\usepackage{graphicx}
\usepackage{amsmath}
\usepackage{amssymb}
\usepackage{amsthm}
\usepackage{amsfonts}
\usepackage{stmaryrd}
\usepackage{booktabs}
\usepackage{gauss}
\usepackage{mathtools}
\usepackage{multirow}
\usepackage{url}
\usepackage{siunitx}
\usepackage{microtype}
\usepackage{graphicx}
\usepackage{appendix}
\usepackage{verbatim}
\pagenumbering{Roman}
% Extra Commands
\newcommand{\tab}{\hspace*{5em}}
\newcommand{\gap}{\vspace*{0.25cm}}
\renewcommand{\implies}{\Rightarrow}
\newcommand{\T}{\mathsf{T}}
\newcommand{\U}{\mathsf{U}}
\newcommand{\V}{\mathsf{V}}
\newcommand{\W}{\mathsf{W}}
\newcommand{\tr}[1]{\text{tr}({#1})}
\newcommand{\rank}[1]{\text{rank}({#1})}
\newcommand{\dime}[1]{\text{dim}({#1})}
\newcommand{\dete}[1]{\text{det}({#1})}
\newcommand{\mfrac}[2]{^{#1}/_{#2}}
\newcommand{\M}[3]{\mathsf{M}_{#1 \times #2}(#3)}
\newmatrix{(}{|}{left}
\newmatrix{.}{)}{right}
\newcommand\bigzero{\makebox(0,0){\text{\huge0}}}
% Formatting
\usepackage[top=1in, bottom=1in, left=1.2in, right=1.2in]{geometry}
\usepackage{fancyhdr}

% Heading stuffs
% \pagestyle{plain}
\setlength{\headheight}{28pt}
\thispagestyle{fancy}
\fancyhf{}
\lhead{Chris Curro, John Biswakarma, Victor Chen \\ November 21, 2012} 
\chead{\hfill \\ MA326}
\rhead[RE,RO]{HW \#8 \\ Prof. Mintchev}
\cfoot[RO,RE]{\thepage}  %Page # for footer
% This should return the format
% Name         Course #        Assignment #
% Date         Course name     Prof. 
% -------------------------------------------
\begin{document}
%\begin{spacing}{1.2}
\section*{Assignment}
Appendix E: Prove theorems E.3, E.5, E.6, E.7; Section 5.2: 2(bde), 3(adef), 7, 9, 11
\section*{Work}
\subsection*{Appendix E}
\begin{enumerate}
\setcounter{enumi}{2}
\item
Let $f(x)$ be a polynomial with coefficients from a field $F,$ and let
$\mathsf{T}$ be a linear operator on a vector space $\mathsf{V}$ over
$F$. Then the following statements are true
\begin{enumerate}
\item $f(\mathsf{T})$ is a linear operator on $\mathsf{V}.$
\item If $\beta$ is a finite ordered basis for $\mathsf{V}$ and $A =
  [\mathsf{T}]_\beta,$ then $[f(\mathsf{T})]_\beta = f(A).$
\end{enumerate}

Suppose $f(x)$ is a polynomial with coefficients in $F$

Suppose $\mathsf{T} \in \mathcal{V}$ and $\mathsf{V}$ is a vector
space over $F.$
\begin{enumerate}
\item Claim: $f(\mathsf{T})$ is a linear operator on $\mathsf{V}.$
\begin{align}
f(x) &= a_nx^n + a_{n-1}x^{n-1} + \dotsb + a_1x + a_0 \\
f(\mathsf{T}) &= a_n\mathsf{T}^n + a_{n-1}\mathsf{T}^{n-1} + \dotsb +
a_1\mathsf{T} + a_0\mathsf{I}_\mathsf{V}
\end{align}
\paragraph{Lemma: } $\mathsf{T}^n \in \mathcal{L}(\mathsf{V})\; \forall
n \in \mathbb{Z}^+.$

Suppose $y,z \in \mathsf{V}.$

Suppose $n=1$
\begin{equation}
\mathsf{T}(az+y) = a\mathsf{T}(z) +\mathsf{T}(y)
\end{equation}
Suppose true for $1\leq n\leq k.$

Suppose $n=k+1$
\begin{align}
\mathsf{T}^{k+1}(az+y) &= \mathsf{T}(\mathsf{T}^k(az+y))\\
&= \mathsf{T}(a\mathsf{T}^k(z) + \mathsf{T}(y))\\
&= a\mathsf{T}^{k+1}(z) + \mathsf{T}^{k+1}(y) \\ & & \qedsymbol\notag
\end{align}
\begin{multline}
\therefore f(\mathsf{T}(az+y)) = a_n\mathsf{T}^n(az+y)
+a_{n-1}\mathsf{T}^{n-1}(az+y) +\dotsb + \\  {}+a_1\mathsf{T}(az +y) +
a_0(az+y)
\end{multline}
\begin{multline}
f(\mathsf{T}(az+y)) = a_n(a\mathsf{T}^n(z) + \mathsf{T}^n(y)) +
a_{n-1}(a\mathsf{T}^{n-1}(z) +\mathsf{T}^{n-1}(y)) + \dotsb + \\ {} +
a_1(a\mathsf{T}(z) +\mathsf{T}(y)) +a_0(az +y)
\end{multline}
\begin{multline}
f(\mathsf{T}(az+y)) = a(a_n\mathsf{T}^n(z)+a_{n-1}\mathsf{T}^{n-1}(z)
+ \dotsb + a_1\mathsf{T}(z) +a_oz) + \\
+ (a_n\mathsf{T}^n(y)+a_{n-1}\mathsf{T}^{n-1}(y)
+ \dotsb + a_1\mathsf{T}(y) +a_oy)
\end{multline}
\begin{equation}
f(\mathsf{T}(az+y)) = af(\mathsf{T})(z) + f(\mathsf{T})(y) \in \mathsf{V}
\end{equation}
\begin{equation}
\implies f(\mathsf{T}) \in \mathcal{L}(\mathsf{V})
\end{equation}
\item Claim: If $\beta$ is a finite ordered basis for $\mathsf{V}$ and
  $A=[\mathsf{T}]_\beta,$ then $\left[f(\mathsf{T})\right]_\beta = f(A).$
\begin{equation}
f(\mathsf{T}) = a_n\mathsf{T}^n +a_{n-1}\mathsf{T}^{n-1}+\dotsb
+a_1\mathsf{T} +a_0\mathsf{I}_\mathsf{V}
\end{equation}
\begin{align}
\implies\left[f(\mathsf{T})\right]_\beta &= \left[a_n\mathsf{T}^n +a_{n-1}\mathsf{T}^{n-1}+\dotsb
+a_1\mathsf{T} +a_0\mathsf{I}_\mathsf{V}\right]_\beta\\
&= a_n\left[\mathsf{T}^n\right]_\beta
a_{n-1}\left[\mathsf{T}^{n-1}\right]_\beta + \dotsb +
a_1\left[\mathsf{T}\right]_\beta
  +a_0\left[\mathsf{I}_\mathsf{V}\right]_\beta\\
&= a_n\left(\left[\mathsf{T}\right]_\beta\right)^n
+a_{n-1}\left(\left[\mathsf{T}\right]_\beta\right)^{n-1} + \dotsb +
a_1\left[\mathsf{T}\right]_\beta + a_0\mathsf{I}_\mathsf{V}\\
&= a_nA^n +a_{n-1}A^{n-1} +\dotsb + a_1A +a_0\\
&= f(A)
\end{align}
\end{enumerate}

\setcounter{enumi}{4}
\item
Let $\mathsf{T}$ be a linear operator on a vector space $\mathsf{V}$
over a field $F,$ and let $A$ be an $n\times n$ matrix with entries
from $F.$ If $f_1(x)$ and $f_2(x)$ are relatively prime polynomials
with entries from $F,$ then there exist polynomials $q_1(x)$ and
$q_2(x)$ with entries from $F$ such that
\begin{enumerate}
\item $q_1(\mathsf{T})f_1(\mathsf{T}) + q_2(\mathsf{T})f_2(\mathsf{T})
  = \mathsf{I}$
\item$q_1(A)f_1(A) + q_2(A)f_2(A) = I.$
\end{enumerate}
Suppose $\mathsf{T} \in \mathcal{L}(\mathsf{V}),$ such that
$\mathsf{V}$ is a vector space over $F,$ and $A\in
\mathsf{M}_{n\times n}(F)$

Suppose $f_1(x), f_2(x) \in \mathsf{P}(F)$ such that they are
relatively prime.
\begin{enumerate}
\item Claim: $ \exists q_1(x)$ and $q_2(x)$ such that
  $q_1(\mathsf{T})f_1(\mathsf{T}) + q_2(\mathsf{T})f_2(\mathsf{T}) =
  \mathsf{I}$ 
\paragraph{}
Because $f_1(x)$ and $f_2(x)$ are relatively prime there exists
polynomials $q_1(x)$ and $q_2(x)$ such that 
\begin{equation}
f_1(x)q_1(x) + f_2(x)q_2(x) = 1
\end{equation}
It follows that 
\begin{equation}
f_1(\mathsf{T})q_1(\mathsf{T}) + f_2(\mathsf{T})q_2(\mathsf{T}) = \mathsf{I}_\mathsf{V}
\end{equation}
\item Claim: $\exists q_1(x)$ and $q_2(x)$ such that $q_1(A)f_1(A) +
  q_2(A)f_2(A) = I_n$
\begin{gather}
f_1(x)q_1(x) + f_2(x)q_2(x) = 1\\
\implies f_1(A)q_1(A) + f_2(A)q_2(A) = I_n
\end{gather}
\end{enumerate}

\newpage{}
\item
Let $\phi(x)$ and $f(x)$ be polynomials. If $\phi(x)$ is irreducible
and $\phi(x)$ does not divide $f(x),$ then $\phi(x)$ and $f(x)$ are
relatively prime.

Claim: Let $\phi(x)$ and $f(x)$ be polynomials. If $\phi(x)$ is
irreducible, and $\phi(x)$ does not divide $f(x)$, then $\phi(x)$ and
$f(x)$ are relatively prime.

\paragraph{}
Because $\phi(x)$ is irreducible, $f(x)$ does not divide
$\phi(x)$. Since $\phi(x)$ does not divide $f(x)$ it follows that
$\phi(x)$ and $f(x)$ are relatively prime.

\item
Any two distinct irreducible monic polynomials are relatively prime.

\paragraph{Lemma: } All factors of an irreducible monic polynomial
$\phi(x)$ are either of the form $c \neq 0, c \in F$ of $d\phi(x), d
\neq 0, d \in F$

Suppose $f(x), \phi(x) \in \mathsf{P}(F)$ and $\phi(x)$ is an
irreducible polynomial.

Suppose $f(x)$ divides $\phi(x)$
\begin{equation}
\implies \phi(x) = f(x)g(x)\quad \text{for some } q(x) \in
\mathsf{P}(F)
\end{equation}
\begin{enumerate}[{\bf {Case} 1}]
\item $\text{deg}(f(x)) \notin \mathbb{Z}^+$
\begin{gather}
f(x)\neq 0 \because \phi(x) \neq 0\\
\implies \text{deg}(f(x)) = 0 \\
\implies f(x) = c \quad \text{for some } c \in F
\end{gather}
\item$\text{deg}(f(x)) \in \mathbb{Z}^+$
\begin{equation}
\phi(x) = f(x)q(x)
\end{equation}
Because $\phi(x)$ is irreducible, it cannot be expressed as a product
of polynomials both possessing positive degree.
\begin{gather}
\implies \text{deg}(q(x)) \leq 0 \\
\implies g(x) = \frac{1}{d} \quad \text{for some nonzero } d \in F\\
\implies \phi(x) = \frac{f(x)}{d}\\
\implies d\phi(x) = f(x)
\end{gather}
\begin{align}
\notag & & & & \qedsymbol
\end{align}
\newpage{}
By lemma, suppose the factors of $\phi_1$ are $c$ and $d\phi_1$ where
$c,d \in F$ and $c,d\neq 0.$ suppose the factors of $\phi_2$ are $e$
and $g\phi_2$ where $e,g \in F, e,g \neq 0$

Claim: $d\phi_1 \neq g\phi_2$

Suppose $g\phi_2 | \phi_1$
\begin{gather}
\implies g\phi_2 = d\phi_1\\
\implies \phi_2 = \frac{d}{g}\phi_1
\implies \phi_2 = \frac{d}{g}(x^n +a_{n-1}x^{n-1} + \dotsb +a_1x + a_0)\\
\implies \phi_2 = \frac{d}{g}x^n +\frac{d}{g}a_{n-1}x^{n-1} + \dotsb +
\frac{d}{g}a_1x + a_0\\
\implies \frac{d}{g} = 1\text{ because } \phi_2 \text{ is monic.} \\
\implies \phi_2 = \phi_1\; \lightning \text{ Contradiction!} \\
\text{Theorem states } \phi_1 \text{ and } \phi_2 \text{ are distinct
  polynomials.} \notag
\end{gather}
\end{enumerate}


\end{enumerate}
\newpage{}
\subsection*{5.2}
\begin{enumerate}
\setcounter{enumi}{1}
\item
In each part, apply the Gram-Schmidt process to the given subset $S$
of the inner product space $\mathsf{V}$ to obtain an orthogonal basis
for $\text{span}(S).$ Then normalize the vectors in this basis to
obtain an orthonormal basis $\beta$ for $\text{span}(S),$ and compute
the Fourier coefficients of the given vector relative to $\beta.$
Finally, use Theorem 6.5 to verify your results.
\begin{enumerate}
\item $\mathsf{V} = \mathsf{R}^3, S= \{(1,0,1),(0,1,1),(1,3,3)\},$ and
  $x = (1,1,2)$
\begin{align}
w_1 = (1,0,1) & & w_2 = (0,1,1) & & w_3 = (1,3,3)
\end{align}
\begin{equation}
v_1 = (1,0,1)
\end{equation}
\begin{equation}
v_2 = (0,1,1) - \frac{\langle w_2,v_1 \rangle}{\norm{v_1}^2}v_1 = \left(-\frac{1}{2},1,\frac{1}{2}\right)
\end{equation}
\begin{equation}
v_3 = (1,3,3) - \left( \frac{\langle w_3,v_1\rangle}{\norm{v_1}^2}v_1
  +  \frac{\langle w_3,v_2\rangle}{\norm{v_2}^2}v_2\right) = \left(\frac{1}{3},\frac{1}{3},\frac{1}{3}\right)
\end{equation}
An orthogonal basis is
\begin{equation}
\notag \left\{(1,3,3),\left(-\frac{1}{2},1,\frac{1}{2}\right),\left(\frac{1}{3},\frac{1}{3},\frac{1}{3}\right)\right\}
\end{equation}
The corresponding orthonormal basis is
\begin{equation}
\beta = \left\{\left(\frac{1}{\sqrt{2}}\right),\left(-\frac{1}{\sqrt{6}},\sqrt{\frac{2}{3}},\frac{1}{\sqrt{6}}\right),\left(\frac{1}{\sqrt{3}},\frac{1}{\sqrt{3}},-\frac{1}{\sqrt{3}}\right)\right\}
\end{equation}
\begin{align}
\langle x,v_1\rangle = \frac{3}{\sqrt{2}} & & \langle x,v_2\rangle =
\sqrt{\frac{3}{2}} && \langle x,v_3\rangle = 0
\end{align}
The Fourier coefficients relative to $\beta$ are
\begin{equation}
\notag \left\{\frac{3}{\sqrt{2}},\sqrt{\frac{3}{3}},0\right\}
\end{equation}
\begin{equation}
\left(\frac{3}{\sqrt{2}}\right)\left(\frac{1}{\sqrt{2}}\right) +
\left(\sqrt{\frac{3}{2}}\right)\left(-\frac{1}{\sqrt{6}},\sqrt{\frac{2}{3}},\frac{1}{\sqrt{6}}\right)
+\left(0\right)\left(\frac{1}{\sqrt{3}},\frac{1}{\sqrt{3}},-\frac{1}{\sqrt{3}}\right)
= (1,1,2) \; \checkmark
\end{equation}
\setcounter{enumii}{2}
\item $\mathsf{V} = \mathsf{P}_2(\mathbb{R})$ with the inner product
  $\langle f(x),g(x)\rangle = \int_0^1f(t)g(t)\; \mathrm{d}t,$
$S=\{1,x,x^2\},$ and $h(x) = 1+x$
\begin{align}
w_1 = 1 & & w_2 = x & & w_3 = x^2
\end{align}
\begin{equation}
v_1 = w_1 = 1
\end{equation}
\begin{equation}
v_2 = w_2 - \frac{\langle w_2,v_1\rangle}{\norm{v_1}^2}v_1 = x - \frac{1}{2}
\end{equation}
\begin{equation}
v_3 = w_3 -\left(
 \frac{\langle w_3,v_1\rangle}{\norm{v_1}^2}v_1 +  \frac{\langle
   w_3,v_2\rangle}{\norm{v_2}^2}v_2\right) = x^2 -x + \frac{1}{6}
\end{equation}
An orthogonal basis is 
\begin{equation}
\notag \left\{1,x-\frac{1}{2},x^2-x+\frac{1}{6}\right\}
\end{equation}
The corresponding orthonormal basis is
\begin{equation}
\beta = \left\{1,\sqrt{3}(2x-1),\sqrt{3}(6x^2-6x+1)\right\}
\end{equation}
\begin{align}
\langle h(x),1\rangle &= \frac{3}{2} \\  \langle
h(x),\sqrt{3}(2x-1)\rangle &= \frac{\sqrt{3}}{6} \\ \langle
h(x),\sqrt{3}(6x^2-6x+1)\rangle &= 0
\end{align}
The Fourier coefficients relative to $\beta$ are 
\begin{equation}
\notag \left\{\frac{3}{2},\frac{\sqrt{3}}{6},0\right\}
\end{equation}
\begin{equation}
\left(\frac{3}{2}\right)(1)
+\left(\frac{\sqrt{3}}{6}\right)(\sqrt{3})(2x-1) + \frac{3}{2} +
\left(\frac{3}{6}\right)(2x-1) = x +1 \; \checkmark
\end{equation}
\setcounter{enumii}{6}
\item $\mathsf{V} = \mathsf{M}_{2\times 2}(\mathbb{R}),$ $S
  =\left\{\begin{pmatrix}3 & 5 \\-1 & 1\end{pmatrix},\begin{pmatrix}-1
      & 9\\5 & -1\end{pmatrix},\begin{pmatrix}7 & -17\\2 &
      -6\end{pmatrix}\right\},$ and $A = \begin{pmatrix}-1 & 27\\-4 &
    8\end{pmatrix}$
\begin{align}
w_1 = \begin{pmatrix}3 & 5 \\-1 & 1\end{pmatrix} & & w_2 = \begin{pmatrix}-1
      & 9\\5 & -1\end{pmatrix} &  & w_3 = \begin{pmatrix}7 & -17\\2 &
      -6\end{pmatrix}
\end{align}
\begin{equation}
v_1 = w_1 = \begin{pmatrix}3 & 5 \\-1 & 1\end{pmatrix}
\end{equation}
\begin{equation}
v_2 = w_2 \frac{\langle w_2,v_1\rangle}{\norm{v_1}^2}v_1
= \begin{pmatrix}-4 & 4\\6 & -2\end{pmatrix}
\end{equation}
\begin{equation}
v_3 = w_3 -\left(
 \frac{\langle w_3,v_1\rangle}{\norm{v_1}^2}v_1 +  \frac{\langle
   w_3,v_2\rangle}{\norm{v_2}^2}v_2\right) = \begin{pmatrix}9 & -3\\6
 &  -6 \end{pmatrix}
\end{equation}
An orthogonal basis is
\begin{equation}
\notag \left\{\begin{pmatrix}3 & 5 \\-1 & 1\end{pmatrix},\begin{pmatrix}-4 & 4\\6 & -2\end{pmatrix},\begin{pmatrix}9 & -3\\6
 &  -6 \end{pmatrix}\right\}
\end{equation}
The corresponding orthonormal basis is 
\begin{equation}
\beta = \left\{
\begin{pmatrix}
\mfrac{1}{2} & \mfrac{5}{6}\\
\mfrac{-1}{6} & \mfrac{1}{6}
\end{pmatrix},
\begin{pmatrix}
\mfrac{-\sqrt{2}}{3} & \mfrac{\sqrt{2}}{3}\\
\mfrac{1}{\sqrt{2}} & \mfrac{-1}{3\sqrt{2}}
\end{pmatrix},
\begin{pmatrix}
\mfrac{1}{\sqrt{2}} & \mfrac{-1}{3\sqrt{2}}\\
\mfrac{\sqrt{2}}{3} & \mfrac{-\sqrt{2}}{3}
\end{pmatrix}
\right\}
\end{equation}
The Fourier coefficients relative to $\beta$ are
\begin{equation}
\notag \left\{24,6\sqrt{2},-9\sqrt{2}\right\}
\end{equation}
\begin{equation}
24
\begin{pmatrix}
\mfrac{1}{2} & \mfrac{5}{6}\\
\mfrac{-1}{6} & \mfrac{1}{6}
\end{pmatrix}+
6\sqrt{2}
\begin{pmatrix}
\mfrac{-\sqrt{2}}{3} & \mfrac{\sqrt{2}}{3}\\
\mfrac{1}{\sqrt{2}} & \mfrac{-1}{3\sqrt{2}}
\end{pmatrix}
-9\sqrt{2}
\begin{pmatrix}
\mfrac{1}{\sqrt{2}} & \mfrac{-1}{3\sqrt{2}}\\
\mfrac{\sqrt{2}}{3} & \mfrac{-\sqrt{2}}{3}
\end{pmatrix}= \begin{pmatrix}-1 & 27\\-4 &
    8\end{pmatrix} \; \checkmark
\end{equation}
\setcounter{enumii}{9}
\item $\mathsf{V} = \mathsf{C}^4,$ $S =
  \{(1,i,2-i,-1),(2+3i,2i,1-i,2i),(-1+7i,6+10i,11-4i,3+4i)\}$ and $x =
  (-2+7i,6+9i,9-3i,4+4i)$
\begin{align}
w_1  = \begin{pmatrix}1\\i\\2-i\\-1\end{pmatrix} & & w_2
= \begin{pmatrix}2+3i\\2i\\1-i\\2i\end{pmatrix} & & w_3 = \begin{pmatrix}-2+7i\\6+9i\\9-3i\\4+4i\end{pmatrix}
\end{align}
\begin{equation}
v_1 = w_1 = \begin{pmatrix}1\\i\\2-i\\-1\end{pmatrix}
\end{equation}
\begin{equation}
v_2 = w_2 \frac{\langle w_2,v_1\rangle}{\norm{v_1}^2}v_1 = \begin{pmatrix}1+3i\\2i\\-1\\1+2i\end{pmatrix}
\end{equation}
\begin{equation}
v_3 = w_3 -\left(
 \frac{\langle w_3,v_1\rangle}{\norm{v_1}^2}v_1 +  \frac{\langle
   w_3,v_2\rangle}{\norm{v_2}^2}v_2\right) = \begin{pmatrix}-7+i \\6+2i\\5\\5\end{pmatrix}
\end{equation}
An orthogonal basis is
\begin{equation}
\notag \left\{
\begin{pmatrix}1\\i\\2-i\\-1\end{pmatrix},\begin{pmatrix}1+3i\\2i\\-1\\1+2i\end{pmatrix},\begin{pmatrix}-7+i \\6+2i\\5\\5\end{pmatrix}
\right\}
\end{equation}
The corresponding orthonormal basis is
\begin{equation}
\beta = \left\{
\frac{1}{\sqrt{8}}\begin{pmatrix}1\\i\\2-i\\-1\end{pmatrix},\frac{1}{2\sqrt{5}}\begin{pmatrix}1+3i\\2i\\-1\\1+2i\end{pmatrix},\frac{1}{2\sqrt{35}}\begin{pmatrix}-7+i \\6+2i\\5\\5\end{pmatrix}
\right\}
\end{equation}
The Fourier coefficients are
\begin{equation}
\notag \left\{
6\sqrt{2},4\sqrt{5},2\sqrt{35}
\right\}
\end{equation}
\begin{equation}
\frac{6\sqrt{2}}{\sqrt{8}}\begin{pmatrix}1\\i\\2-i\\-1\end{pmatrix}+
\frac{4\sqrt{5}}{2\sqrt{5}}\begin{pmatrix}1+3i\\2i\\-1\\1+2i\end{pmatrix}+
\frac{2\sqrt{35}}{2\sqrt{35}}\begin{pmatrix}-7+i
  \\6+2i\\5\\5\end{pmatrix}
=\begin{pmatrix}-2+7i\\6+9i\\9-3i\\4+4i
\end{pmatrix}\;\checkmark
\end{equation}
\end{enumerate}

\newpage{}
\item
Using the results of Exercise 2, find all solutions to the following
systems.
\begin{enumerate}
\item \begin{align}
x_1 +3x_2 = 5 & & 2x_1 + 6x_2 = 10
  \end{align}
\begin{equation}
\begin{gmatrix}[left]
1 & 3&\\
2 & 6&
\end{gmatrix}
\begin{gmatrix}[right]
5\\
10
\rowops
\add[-2]{0}{1}
\end{gmatrix}
\leadsto
\begin{gmatrix}[left]
1 & 3&\\
0 & 0&
\end{gmatrix}
\begin{gmatrix}[right]
5 \\ 0
\end{gmatrix}
\end{equation}
\begin{align}
\implies x_2 = t && x_1 = 5-3t
\end{align}
\begin{equation}
x = \left\{\begin{pmatrix}5\\0\end{pmatrix}
  +t \begin{pmatrix}-3\\1\end{pmatrix}\colon t \in \mathbb{R}\right\}
\end{equation}
\setcounter{enumii}{3}
\item\begin{align}
2x_1 + x_2 - x_3 =5 & & x_1 - x_2 +x_3 =1 & & x_1 +2x_2 -2x_3 =4
  \end{align}
\begin{equation}
\begin{gmatrix}[left]
2 & 1 & -1 &\\
1 & -1 & 1 &\\
1 & 2 & -2 &
\end{gmatrix}
\begin{gmatrix}[right]
5\\
1\\
4
\rowops
\swap{0}{1}
\add[-2]{0}{1}
\add[-1]{0}{2}
\add[-1]{1}{2}
\mult{1}{\cdot \frac{1}{3}}
\end{gmatrix}
\leadsto
\begin{gmatrix}[left]
1 & -1 &1&\\
0 & 1& -1&\\
0 & 0 & 0&
\end{gmatrix}
\begin{gmatrix}[right]
1\\
1\\
0
\end{gmatrix}
\end{equation}
\begin{align}
\implies x_3 =t && x_2 1 + t && x_1 =2
\end{align}
\begin{equation}
\implies x = \left\{\begin{pmatrix}2\\1\\0\end{pmatrix} +
  t \begin{pmatrix}0\\1\\1\end{pmatrix}\colon t \in \mathbb{R}\right\}
\end{equation}
\end{enumerate}

\setcounter{enumi}{6}
\item
Let $\mathsf{T}$ be a linear operator on a finite-dimensional vector
space $\mathsf{V}$. We define the {\bf determinant} of $\mathsf{T}$,
denoted $\det{(\mathsf{T})}$, as follows: Choose any ordered basis
$\beta$ for $\mathsf{V}$, and define $\det{(\mathsf{T})}=
\det{([\mathsf{T}]_\beta )}$.
\begin{enumerate}
\item Prove that he preceding definition is independent of the choice
  of an ordered basis for $\mathsf{V}$. That is, prove that if $\beta$
  and $\gamma$ are two ordered bases for $\mathsf{V}$, then $\det{([\mathsf{T}]_\beta)}=\det{([\mathsf{T}]_\gamma)}$.
\item Prove that $\mathsf{T}$ is invertible if and only if
  $\det{\mathsf{T}}\neq 0$.
\item Prove that if $\mathsf{T}$ is invertible, then
  $\det{(\mathsf{T}^{-12})}= [\det{(\mathsf{T})}]^{-1}$.
\item Prove that if $\mathsf{U}$ is also a linear operator on
  $\mathsf{V}$, then $\det{(\mathsf{TU})} = \det{(\mathsf{T})}\cdot\det{(\mathsf{U})}$.
\item Prove that $\det{(\mathsf{T}-\lambda \mathsf{I}_\mathsf{V})} =
    \det{[\mathsf{T}]_\beta -\lambda I)}$ for any scalar $\lambda$ and
      any ordered basis $\beta$ for $\mathsf{V}$.
\end{enumerate}
\begin{enumerate}
\item
Suppose $Q$ is the change of coordinates matrix from $\gamma$ to
$\beta$.
\begin{gather}
Q=[\mathsf{I}_\mathsf{V}]_\gamma^\beta \implies
Q^{-1}=[\mathsf{I}_\mathsf{V}]^\gamma_\beta \\
[\mathsf{I}_\mathsf{V}]_\gamma^\beta
[\mathsf{T}]_\gamma[\mathsf{I}_\mathsf{V}]^\gamma_\beta =
[\mathsf{T}]_\beta\\
\implies \det{([\mathsf{I}_\mathsf{V}]_\gamma^\beta
[\mathsf{T}]_\gamma[\mathsf{I}_\mathsf{V}]^\gamma_\beta)} =
\det{[\mathsf{T}]_\beta}\\
\implies\det{[\mathsf{I}_\mathsf{V}]_\gamma^\beta}
\det{[\mathsf{T}]_\gamma}\det{[\mathsf{I}_\mathsf{V}]^\gamma_\beta}\\
[\mathsf{I}_\mathsf{V}]_\gamma^\beta =
([\mathsf{I}_\mathsf{V}]^\gamma_\beta)^{-1}\\
\det{[\mathsf{I}_\mathsf{V}]_\gamma^\beta}=
\det{([\mathsf{I}_\mathsf{V}]^\gamma_\beta)^{-1}}
\end{gather}
\begin{align}
\implies \det{[\mathsf{I}_\mathsf{V}]_\gamma^\beta}
\det{[\mathsf{T}]_\gamma}\det{[\mathsf{I}_\mathsf{V}]^\gamma_\beta} &=
\det{([\mathsf{I}_\mathsf{V}]^\gamma_\beta)^{-1}}\det{[\mathsf{T}]_\gamma}\det{[\mathsf{I}_\mathsf{V}]^\gamma_\beta}\\
&= \det[\mathsf{T}]_\gamma = \det[\mathsf{T}]_\beta
\end{align}
\item 
($\implies$) 

Suppose $\mathsf{T}$ is invertible\\
Suppose $\beta$ is an ordered basis of $\mathsf{V}$.
\begin{gather}
[\mathsf{I}_\mathsf{V}]_\beta = [\mathsf{T}\cdot\mathsf{T}^{-1}]_\beta
= [\mathsf{T}]_\beta[\mathsf{T}^{-1}]_\beta\\
\implies \det{[\mathsf{I}_\mathsf{V}]_\beta} =
\det{[\mathsf{T}]_\beta}\det{[\mathsf{T}^{-1}]_\beta}\\
\det{[\mathsf{I}_\mathsf{V}]_\beta} =1 \\
\implies \det{[\mathsf{T}]_\beta}\det{[\mathsf{T}^{-1}]_\beta} =1\\
\implies \det{[\mathsf{T}]_\beta} \neq 0\\
\implies \det{\mathsf{T}} \neq 0
\end{gather}
($\Leftarrow$)

Suppose $\det{\mathsf{T}}\neq 0$ 
\begin{gather}
\implies \det[\mathsf{T}]_\beta \neq 0 \quad \text{for some ordered
  basis } \beta \text{ of } \mathsf{V}\\
\det{[\mathsf{T}]_\beta} \neq  0\\
\implies \mathsf{T} \text{ is invertible, by corollary to Th. 2.18}
\end{gather}
\item Suppose $\mathsf{T}$ is invertible and $\beta$ is some ordered
  basis of $\mathsf{V}$.
\begin{align}
\det{\mathsf{I}_\mathsf{V}} &= \det{\mathsf{T}\cdot\mathsf{T}^{-1}}\\
&= \det{[\mathsf{T}\cdot\mathsf{T}^{-1}]_\beta}\\
&= \det{[\mathsf{T}]_\beta[\mathsf{T}^{-1}]_\beta}\\
&= \det{[\mathsf{T}]_\beta}\det{[\mathsf{T}^{-1}]_\beta}\\
&= \det{\mathsf{T}}\det{\mathsf{T}^{-1}}
\end{align}
\begin{gather}
\det{\mathsf{I}_\mathsf{V}} = \det{[\mathsf{I}_\mathsf{V}]_\beta} =
  \det{I_n} =1\\
\implies \det{\mathsf{T}} \det{\mathsf{T}^{-1}} = 1\\
\implies \det{\mathsf{T}^{-1}} = (\det{\mathsf{T}})^{-1}
\end{gather}
\item Suppose $\beta$ is an ordered basis of $\mathsf{V}$.
\begin{align}
\det{\mathsf{TU}} &= \det{[\mathsf{TU}]_\beta}\\
&= \det{[\mathsf{T}]_\beta[\mathsf{U}]_\beta}\\
&= \det{[\mathsf{T}]_\beta}\det{[\mathsf{U}]_\beta}\\
&= \det{\mathsf{T}}\det{\mathsf{U}}
\end{align}
\item 
\begin{align}
\det{([\mathsf{T}]_\beta -\lambda I)} &= \det{([\mathsf{T}]_\beta
  -\lambda[\mathsf{I}_\mathsf{V}]_\beta)}\\
&= \det{([\mathsf{T}]_\beta
  -[\lambda\mathsf{I}_\mathsf{V}]_\beta)}\\
&= \det{[\mathsf{T}_\beta -\lambda \mathsf{I}_\mathsf{V}]_\beta}\\
&= \det{(\mathsf{T}-\lambda \mathsf{I}_\mathsf{V})}
\end{align}
\end{enumerate}

\setcounter{enumi}{8}
\item
Prove that the system of linear equations $Ax=b$ has a solution if and
only if $b\in \mathsf{R}(\mathsf{L}_A)$.
\\$(\implies)$
\\Suppose $Ax=b$ has a solution
\begin{gather}
\implies \exists x \colon \mathsf{L}_A(x) =b\\
\implies b \in \mathsf{R}(\mathsf{L}_A)
\end{gather}
$(\Leftarrow)$
\\Suppose $b \in \mathsf{R}(\mathsf{L}_A)$
\begin{gather}
\implies \exists x \colon Ax =b
\end{gather}

\setcounter{enumi}{10}
\item
Let $\V$ be as in Exercise 10.
\begin{enumerate}
\item Show that $S=\{(1,2,1,0,0)\}$ is a linearly independent subset
  of $\V$.
\item Extend $S$ to a basis for $\V$.
\end{enumerate}
\begin{enumerate}
\item Claim: $S$ is a linearly independent subset of $\V$.
\\For $x \in S, x = (1,2,1,0,0)$
\begin{gather}
1+ (-2)(2) + 3(1) +(-1)(0) +(2)(0) = 0\\
\implies x \in \V\\
\implies S \subseteq \V
\end{gather}
Suppose $cx -0$ 
\begin{gather}
c(1,2,1,0,0) = 0\\
\implies (c,2c,c,0,0) = 0\\
\implies c =0
\end{gather}
It follows that $S$ is linearly independent.
\item Suppose $x_1,x_2,\dotsc,x_5 \in F \colon x_1 = 2x_2 -3x_3 +x_4
  -2x_5$ 
\begin{align}
\begin{pmatrix}x_1\\x_2\\x_3\\x_4\\x_5\end{pmatrix} &= \begin{pmatrix}2x_2 -3x_3 +x_4 -2x_5\\x_2\\x_3\\x_4\\x_5\end{pmatrix}\\
&= x_2\begin{pmatrix}2\\1\\0\\0\\0\end{pmatrix}
+x_3\begin{pmatrix}-3\\0\\1\\0\\0\end{pmatrix}
+x_4\begin{pmatrix}1\\0\\0\\1\\0\end{pmatrix}
+x_5\begin{pmatrix}-2\\0\\0\\0\\1\end{pmatrix}
\end{align}
\begin{equation}
\beta =\left\{\begin{pmatrix}2\\1\\0\\0\\0\end{pmatrix},
\begin{pmatrix}-3\\0\\1\\0\\0\end{pmatrix},
\begin{pmatrix}1\\0\\0\\1\\0\end{pmatrix},
\begin{pmatrix}-2\\0\\0\\0\\1\end{pmatrix}\right\}
\end{equation}
\paragraph{} Where $\beta$ is a basis $\V$.
\begin{equation}
(S\, | B )
=
\begin{pmatrix}
1 & 2 & -3 & 1 & -2 \\
2 & 1 & 0 & 0 & 0 \\
1 & 0 & 1 & 0 & 0 \\
0 & 0 & 0 & 1 & 0 \\
0 & 0 & 0 & 0 & 1
\end{pmatrix}
\leadsto
\begin{pmatrix}
1 & 0 & 1 & 0 & 0\\
0 & 1 & 2 & 0 & 0\\
0 & 0 & 0 & 1 & 0\\
0 & 0 & 0 & 0 & 1\\
0 & 0 & 0 & 0 & 0
\end{pmatrix}
\end{equation}
\begin{equation}
S_\beta\footnote{We need say what exaclty thsi is} = \left\{\begin{pmatrix}1\\2\\1\\0\\0\end{pmatrix},\begin{pmatrix}2\\1\\0\\0\\0\end{pmatrix},\begin{pmatrix}1\\0\\0\\1\\0\end{pmatrix},\begin{pmatrix}-2\\0\\0\\0\\1\end{pmatrix}\right\}
\end{equation}
\end{enumerate}

\end{enumerate}
% \end{spacing}
\end{document}

