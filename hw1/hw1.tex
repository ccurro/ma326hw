\documentclass[letterpaper,12pt]{article}
\usepackage{setspace}
\usepackage[utf8]{inputenc}
\usepackage[english]{babel}
\usepackage{enumerate}
\usepackage{graphicx}
\usepackage{amsmath}
\usepackage{gauss}
\usepackage{amsfonts}
\usepackage{stmaryrd}
\usepackage{mathtools}
\usepackage{multirow}
\usepackage{url}
\usepackage{amssymb}
\usepackage{siunitx}
\usepackage{url}
\usepackage{microtype}
\usepackage{graphicx}
\usepackage{appendix}
\usepackage{verbatim}
\pagenumbering{Roman}
% Extra Commands
\newcommand{\tab}{\hspace*{1.5em}}
\newcommand{\gap}{\vspace*{0.01cm}}
\newcommand{\?}{\stackrel{?}{=}}

% Formatting
\usepackage[top=1in, bottom=1in, left=1in, right=1in]{geometry}
\usepackage{fancyhdr}

% Heading stuffs
% \pagestyle{plain}
\thispagestyle{fancy}
\fancyhf{}
\lhead{Chris Curro, John Biswakarma, Yu-An Chen \\ September  21, 2012} 
\chead{ \\ MA 326 Linear Algebra}
\rhead[RE,RO]{HW \# 2 \\ Prof. Mintchev }
\cfoot[RO,RE]{\thepage}  %Page # for footer
% This should return the format
% Name         Course #        Assignment #
% Date         Course name     Prof. 
% -------------------------------------------
\begin{document}
%\hfill\\
\gap
\begin{spacing}{1.5}
\section{Assignment}
Section 1.2: 7, 20, 21; Section 1.3: 5, 10, 20 ,23, 25; 
Section 1.4: 2, 4, 11, 13
\section{Work}
\setcounter{section}{1}
\setcounter{subsection}{1}
\subsection{}
\begin{enumerate}
\setcounter{enumi}{6}
\item Let $S = \left\{0,1\right\}$ and $F = \mathbb{R}$. In
  $\mathcal{F}\left(S,\mathbb{R}\right)$, show that $f = g$ and $f + g = h$,
  where $f(t) = 2t + 1, g(t) = 1 + 4t -2t^2,$ and $ h(t) = 5^t +1$.
\\
Claim: $\forall\; t \in S, f = g$
\begin{itemize}
\item For $t=0$
\begin{equation}
f(0) = \left(2 \times 0\right) + 1 = g(0) = 1 + \left(4 \times 0\right) - \left(2\times 0^2\right)
\end{equation}
\begin{equation}
f(0) = g(0) = 0
\end{equation}
\item For $t=1$
\begin{equation}
f(1) = \left(2 \times 1\right) + 1 = g(1) = 1 + \left(4 \times 1\right) - \left(2\times 1^2\right)
\end{equation}
\begin{equation}
f(1) = g(1) = 3
\end{equation}
\end{itemize}
Claim: $\forall\; t \in S, f + g = h$
\begin{itemize}
\item For $t=0$
\begin{equation}
f(0) + g(0) = h(0) = 5^0 + 1
\end{equation}
\begin{equation}
f(0) + g(0) = h(0) = 2
\end{equation}
\item For $t=1$
\begin{equation}
f(1) + g(1) = h(1) = 5^1 + 1
\end{equation}
\begin{equation}
f(1) + g(1) = h(1) = 6 
\end{equation}
\end{itemize}
\setcounter{enumi}{19}
\item Let $\mathsf{V}$ be the set of sequences $\left\{a_n\right\}$ of
  real numbers. For $\left\{a_n\right\}$, $\left\{b_n\right\} \in
    \mathsf{V}$ and real number $t$ define 
\[
\left\{a_n\right\}  + \left\{b_n\right\} = \left\{a_n + b_n\right\}
\]
\[
t\left\{a_n\right\} = \left\{ta_n\right\}
\]
Prove that, with these operations, $\mathsf{V}$ is a vector space over
$\mathbb{R}$.
\\
\textbf{Definition}: $\left\{a_n\right\}$ is any sequence where $\sigma \colon
\mathbb{Z}^+ \to \mathbb{R}$ given by $\sigma\left(n\right) = a_n$
\\
Claim: $\left\{a_n + b_n\right\} \in \mathsf{V}$
\\
\tab Since the sequences $\{a_n\}$ and $\{b_n\}$ are defined as sequences
whose elements exist in $\mathbb{R}$ and addition is a closed binary operation on
$\mathbb{R}$, the sum of any two elements in $\{a_n\}$ and $\{b_n\}$ must also
exist in $\mathbb{R}$.
\\
Claim: $\left\{ta_n\right\} \in \mathsf{V}$
\\
\tab Since the sequence $\{a_n\}$ is defined as a sequence whose
elements exist in $\mathbb{R}$ and multiplication is a closed binary operation on
$\mathbb{R}$, the product of an element of $\{a_n\}$ and $t$ an
element of $\mathbb{R}$ must also exist in $\mathbb{R}$.
\begin{enumerate}[(VS 1)]
\item $\forall\; x,y \in \mathsf{V}, \; x+y = y + x$
\\
Claim: $\left\{a_n\right\} + \left\{b_n\right\} = \left\{b_n\right\} +
\left\{a_n\right\}$
\\
\begin{equation}
\left\{a_n\right\} + \left\{b_n\right\} =
\left\{a_n +_\mathbb{R} b_n\right\} =
\left\{b_n +_\mathbb{R} a_n\right\} = 
\left\{b_n\right\} +
\left\{a_n\right\}
\end{equation}
\item $\forall\; x,y,z \in \mathsf{V}, \; \left(x + y\right) + z = x +
  \left(y + z\right)$
\\
Claim: $\big(\left\{a_n\right\} + \left\{b_n\right\}\big) +
\left\{c_n\right\} = \left\{a_n\right\} + \big(\left\{b_n\right\} +
\left\{c_n\right\}\big)$
\begin{align}
\big(\left\{a_n\right\} + \left\{b_n\right\}\big) +
\left\{c_n\right\} &= 
\left\{a_n +_\mathbb{R} b_n\right\} +
\left\{c_n\right\}\\
%\end{equation}
%\begin{equation}
 &= \left\{a_n +_\mathbb{R} b_n +_\mathbb{R} c_n\right\}\\
%\end{equation}
%\begin{equation}
&= \left\{a_n\right\} + \left\{ b_n +_\mathbb{R} c_n\right\}\\
%\end{equation}
%\begin{equation}
&= 
\left\{a_n\right\} + \big(\left\{b_n\right\} + \left\{c_n\right\}\big)
\end{align}
\newpage{}
\item $\exists\; 0 \in \mathsf{V}$ such that $\forall\; x \in
  \mathsf{V}, \;x + 0 = x$
\\
Suppose $\left\{b_n\right\} \in \mathsf{V}$ such that $\forall\; n \in
\mathbb{ Z}^+,\; \sigma(n) = b_n = 0$

\\
Claim: $\left\{a_n\right\} + 0_\mathsf{V} = \left\{a_n\right\}$ 
\begin{align}
\left\{a_n\right\} + \left\{b_n\right\} = \left\{a_n +_\mathbb{R}
  b_n\right\} &= \left\{a_n +_\mathbb{R} 0\right\} = \left\{a_n\right\}\\
\implies \left\{b_n\right\} &= 0_\mathsf{V}
\end{align}

\item $\forall\; x \in \mathsf{V}\; \exists\; y \in \mathsf{V}$ such that
  $x + y =0$
\\
Suppose $ \left\{b_n\right\} \in \mathsf{V}$ such that $\sigma(n) =
b_n = -a_n$
\\
Claim: $\left\{a_n\right\} + \left\{b_n\right\} = 0$
\begin{align}
\left\{a_n\right\} + \left\{b_n\right\} &= \left\{a_n +_\mathbb{R}
  b_n\right\}\\
\left\{a_n +_\mathbb{R} \left(-a_n\right)\right\} &= 0_\mathsf{V}
\end{align}

%%%%%%%%%%%%%%%%%%%%%%%%%%%%%%%%%%%%%%%%%%%
\item $\forall\; x \in \mathsf{V}, 1\times x = x$
\\
Claim: $1\times\left\{a_n\right\} = \left\{a_n\right\}$
\begin{equation}
1\times\left\{a_n\right\} =
\left\{1\times_\mathbb{R}a_n\right\} = 
\left\{a_n\right\}
\end{equation}

%%%%%%%%%%%%%%%%%%%%%%%%%%%%%%%%%%%%%%%%%%%%
\item $\forall\; a,b \in F, \forall\; x \in \mathsf{V}, \;
  \left(ab\right)x = a\left(bx\right)$
\\
Suppose $s,t \in \mathbb{R}$ 
\\
Claim: $ \left(s\times_\mathbb{R}t\right) \left\{a_n\right\} = s \left(t 
  \left\{a_n\right\}\right)$
\begin{align}
\left(s\times_\mathbb{R}t\right) \left\{a_n\right\} &=
\left\{\left(s\times_\mathbb{R}t\right)\times_\mathbb{R} a_n\}\\
&=
\left\{s\times_\mathbb{R}\left(t\times_\mathbb{R}a_n\right)\right\}\\
&= s\left\{t\times_\mathbb{R}a_n\right\}\\
&= s\left(t\left\{a_n\right\}\right)
\end{align}
\newpage{}
%%%%%%%%%%%%%%%%%%%%%%%%%%%%%%%%%%%%%%%%%%%%%%%%%%%
\item $\forall\; a \in F, \forall\; x,y \in \mathsf{V}, \;
  a\left(x+y\right) = ax + ay$
\\
Suppose $t\in\mathbb{R}$
\\
Claim: $ t\big(\left\{a_n\right\} + \left\{b_n\right\}\big) =
  t\left\{a_n\right\} + t\left\{b_n\right\}$
\begin{align}
t\left(\left\{a_n\right\} + \left\{b_n\right\}\right) &= t\left\{a_n
  +_\mathbb{R} b_n\right\} \\
&= \left\{t\times_\mathbb{R} \left(a_n +_\mathbb{R}
    b_n\right)\right\}\\
&= \left\{\left(t\times_\mathbb{R}a_n\right) +_\mathbb{R}
  \left(t\times_\mathbb{R}b_n\right)\right\}\\
&= \left\{t\times_\mathbb{R}a_n\right\} +
  \left\{t\times_\mathbb{R}b_n\right\}\\
 &= t\left\{a_n\right\} + t\left\{b_n\right\}
\end{align}

%%%%%%%%%%%%%%%%%%%%%%%%%%%%%%%%%%%%%%%%%%%%%%%%%%%%%
\item$\forall\; a,b \in F, \forall\; x \in \mathsf{V}, \;
  \left(a+b\right)x = ax + bx$
\\
Suppose $s,t \in \mathbb{R}$ 
\\
Claim: $\left(s +_\mathbb{R} t\right)\left\{a_n\right\} =
s\left\{a_n\right\} + t\left\{a_n\right\}$
\begin{align}
\left(s+_\mathbb{R}t\right)\left\{a_n\right\} &= 
\left\{\left(s+_\mathbb{R}t\right)a_n\right\}\\
&= \left\{s\times_\mathbb{R}a_n+t\times_\mathbb{R}a_n\right\}\\
&= \left\{s\times_\mathbb{R}a_n\right\} +
\left\{t\times_\mathbb{R}a_n\right\}\\
&= s\left\{a_n\right\} + t\left\{a_n\right\}
\end{align}
\end{enumerate}
%%%%%%%%%%%%%%%%%%%%%%%%%%%%%%%%%%%%%%%%%
\newpage{}
\item Let $\mathsf{V}$ and $\mathsf{W}$ be vector spaces over a field
  $F$. Let 
\[
\mathsf{Z} = \left\{\left(v,w\right)\colon v \in \mathsf{V} \,\mathrm{and}\; w
  \in \mathsf{W}\right\}
\]
Prove that $\mathsf{Z}$ is a vector space over $F$ with the operations 
\[
\left(v_1, w_1\right) + \left(v_2, w_2\right) = \left(v_1 + v_2, w_1 +
  w_2\right) \, \mathrm{and}\; c\left(v_1, w_1\right) = \left(cv_1, cw_1\right)
\]
\\
Claim: $\left(v_1 + v_2, w_1 + w_2\right) \in \mathsf{Z}$
\begin{align}
v_1,v_2 &\in \mathsf{V} & w_1, w_2 &\in \mathsf{W}\\
%\end{equation}
%\begin{equation}
\implies \left(v_1 + v_2\right) &\in \mathsf{V} & \implies
\left(w_1+w_2\right) &\in \mathsf{W}
\end{align}
\begin{equation}
\implies \left(v_1 + v_2, w_1 + w_2\right) \in \mathsf{Z}
\end{equation}
\begin{equation}
\therefore\; + \colon \mathsf{Z} \times \mathsf{Z} \to \mathsf{Z}
\end{equation}
\\
Claim: $\left(cv_1, cw_1\right) \in \mathsf{Z}$
\begin{align}
v_1 &\in \mathsf{V} & w_1 &\in \mathsf{W}\\
\implies \left(c \times_\mathsf{V} v_1\right) &\in \mathsf{V} & \implies
\left(c\times_\mathsf{V}w_1\right) &\in \mathsf{W}
\end{align}
\begin{align}
\implies \left(c\times_\mathsf{V} v_1, c\times_\mathsf{W}w_1\right)
&\in \mathsf{Z}\\
\left(cv_1, cw_1\right) &\in \mathsf{Z}
\end{align}
\begin{equation}
\therefore\; \times_\mathsf{Z} \colon F \times \mathsf{Z} \to \mathsf{Z}
\end{equation}
\newpage{}
\begin{enumerate}[(VS 1)]
\item $\forall\; x,y \in \mathsf{V}, \; x+y = y + x$
\\
Suppose $z_1,z_2 \in \mathsf{Z} $ where $z_1 = \left(v_1,w_1\right)$
and $z_2 = \left(v_2,w_2\right)$\\
Claim: $z_1 + z_2 = z_2 + z_1$
\begin{align}
z_1 +_\mathsf{Z} z_2 &= \left(v_1,w_1\right) +_\mathsf{Z} \left(v_1,w_2\right)\\
&= \left(v_1 +_\mathsf{V} v_2, w_1 +_\mathsf{W} w_2 \right)\\
&= \left(v_2 +_\mathsf{V} v_1, w_2 +_\mathsf{W} w_1 \right)\\
&= \left(v_2,w_2\right) +_\mathsf{Z} \left(v_1,w_1\right)\\
&= z_2 +_\mathsf{Z} z_1
\end{align}

%%%%%%%%%%%%%%%%%%%%%%%%%%%%%%%%%%%%%%%%%%%%%%%%%%5
\item $\forall\; x,y,z \in \mathsf{V}, \; \left(x + y\right) + z = x +
  \left(y + z\right)$\\
Suppose $z_1,z_2,z_3 \in \mathsf{Z} $ where $z_1 = \left(v_1,w_1\right)$
, $z_2 = \left(v_2,w_2\right)$ and $z_3 = \left(v_3,w_3\right)$   \\
Claim: $\left(z_1 +_\mathsf{Z} z_2\right) +_\mathsf{Z} z_3 =z_1
+_\mathsf{Z} \left(z_2 +_\mathsf{Z} z_3\right)$
\begin{align}
\left(z_1 +_\mathsf{Z} z_2\right) +_\mathsf{Z} z_3 & = \left(v_1
  +_\mathsf{V} v_2, w_1 +_\mathsf{W} w_2\right) +_\mathsf{V}
\left(v_3,w_3\right)\\
&= \left(v_1 +_\mathsf{V} v_2 +_\mathsf{V} v_3, w_1 +_\mathsf{W} w_2 +_\mathsf{W} w_3\right)\\
&= \left(v_1,w_1\right) +_\mathsf{Z}
\left(v_2+_\mathsf{V}v_3,w_2+_\mathsf{W}w_3\right)\\
&= z_1 +_\mathsf{Z} \left(z_2 +_\mathsf{Z} z_3\right)
\end{align}

%%%%%%%%%%%%%%%%%%%%%%%%%%%%%%%%%%%%%%%%%%%%%%%%%%% 
\item $\exists\; 0 \in \mathsf{V}$ such that $\forall\; x \in
  \mathsf{V}, \;x + 0 = x$\\
Suppose $z \in \mathsf{Z}$ where $z = \left(v,w\right)$ and $0_\mathsf{Z} = \left(0_\mathsf{V},0_\mathsf{W}\right)$\\
Claim: $z +_\mathsf{Z} 0_\mathsf{Z} = z$
\begin{align}
z + 0_\mathsf{Z} &= \left(v,w\right) +_\mathsf{Z}
\left(0_\mathsf{V},0_\mathsf{W}\right)\\
&= \left(v+_\mathsf{V}0_\mathsf{V},w+_\mathsf{W}0_\mathsf{W}\right)\\
&= \left(v,w\right)\\
&= z
\end{align}
\newpage{}

%%%%%%%%%%%%%%%%%%%%%%%%%%%%%%%%%%%%%%%%%%%%%%%
\item $\forall\; x \in \mathsf{V}\; \exists\; y \in \mathsf{V}$ such that
  $x + y =0$\\
Suppose $z_1,z_2 \in \mathsf{Z}$ where $z_1 = \left(v,w\right)$ and
$z_2 = \left(-v,-w\right)$\\
Claim: $z_1 +_\mathsf{Z} z_2 = 0_\mathsf{Z}$
\begin{align}
z_1 +_\mathsf{Z} z_2 &= \left(v,w\right) +_\mathsf{Z}
\left(-v,-w\right)\\
&= \left(v +_\mathsf{V} \left(-v\right), w +_\mathsf{W}
  \left(-w\right)\right)\\
&=\left(0_\mathsf{V},0_\mathsf{W}\right)\\
&=0_\mathsf{Z}
\end{align}

%%%%%%%%%%%%%%%%%%%%%%%%%%%%%%%%%%%%%%%%%%%
\item $\forall\; x \in \mathsf{V}, 1\times x = x$
Suppose $z \in \mathsf{Z}$ where $z = \left(v,w\right)$\\
Claim: $1 \times_\mathsf{Z} z = z$
\begin{align}
1 \times_\mathsf{Z} z &= 1 \times_\mathsf{Z} \left(v,w\right)\\
&= \left(1\times_\mathsf{V}v,1\times_\mathsf{W}w\right)\\
&= \left(v,w\right)\\
&= z
\end{align}

%%%%%%%%%%%%%%%%%%%%%%%%%%%%%%%%%%%%%%%%%%%%
\item $\forall\; a,b \in F, \forall\; x \in \mathsf{V}, \;
  \left(ab\right)x = a\left(bx\right)$\\
Suppose $z\in \mathsf{Z}$ where $z = \left(v,w\right)$ and $a,b \in
F$\\
Claim: $\left(a\times_Fb\right)\times_\mathsf{Z}z = a\times_\mathsf{Z}\left(b\times_\mathsf{Z}z\right)$
\begin{align}
\left(a\times_Fb\right)\times_\mathsf{Z}z &= \left(ab\times_\mathsf{V}v,ab\times_\mathsf{W}w\right)\\
&= a\times_\mathsf{Z}\left(b\times_\mathsf{V}v,b\times_\mathsf{W}w\right)\\
&= a\times_\mathsf{Z}\left(b\times_\mathsf{Z}z\right)
\end{align}
\newpage{}
%%%%%%%%%%%%%%%%%%%%%%%%%%%%%%%%%%%%%%%%%%%%%%%%%%%
\item $\forall\; a \in F, \forall\; x,y \in \mathsf{V}, \;
  a\left(x+y\right) = ax + ay$\\
Suppose $z_1,z_2 \in \mathsf{Z}$ where $z_1=\left(v_1,w_1\right),\;
z_2=\left(v_2,w_2\right)$ and $a \in F\right)$\\
Claim: $a\times_\mathsf{Z}\left(z_1 +_\mathsf{Z} z_2\right) =
a\times_\mathsf{Z}z_1 +_\mathsf{Z} a\times_\mathsf{Z}z_2$
\begin{align}
a\times_\mathsf{Z}\left(z_1 +_\mathsf{Z} z_2\right) &=
a\times_\mathsf{Z}\left(v_1+v_2,w_1+w_q\right)\\
&=
\big(a\times_\mathsf{V}\left(v_1+v_2\right),a\times_\mathsf{W}\left(w_1+w_2\right)\big)\\
&= \big(a\times_\mathsf{W}v_1 +_\mathsf{W}
a\times_\mathsf{W}v_2,a\times_\mathsf{W}w_1 +_\mathsf{W}
a\times_\mathsf{W}w_2\big)\\
&= \left(a\times_\mathsf{V}v_1,a\times_\mathsf{W}w_1\right)
+_\mathsf{Z}
\left(a\times_\mathsf{V}v_2,a\times_\mathsf{W}w_2\right)\\
&= a\times_\mathsf{Z}z_1 +_\mathsf{Z} a\times_\mathsf{Z}z_2
\end{align}
%%%%%%%%%%%%%%%%%%%%%%%%%%%%%%%%%%%%%%%%%%%%%%%%%%%%% 
\item$\forall\; a,b \in F, \forall\; x \in \mathsf{V}, \;
  \left(a+b\right)x = ax + bx$\\
Suppose $z\in \mathsf{Z}$ where $z=\left(v,w\right)$ and $a,b\in F$\\
Claim: $\left(a +_F b\right)\times_\mathsf{Z} z = a\times_\mathsf{Z}z
+_\mathsf{Z} b\times_\mathsf{Z}z$
\begin{align}
\left(a +_F b\right)\times_\mathsf{Z} z &= \left(a +_F
  b\right)\times_\mathsf{Z} \left(v,w\right)\\
&=
\left(\left(a+_Fb\right)\times_\mathsf{V}v,\left(a+_Fb\right)\times_\mathsf{W}w\right)\\
&=\left(a\times_\mathsf{V}v +_\mathsf{V}
  b\times_\mathsf{V}v,a\times_\mathsf{W}w +_\mathsf{W}
  b\times_\mathsf{W}w\right)\\
&= \left(a\times_\mathsf{V}v,a\times_\mathsf{W}w\right) +_\mathsf{Z}
\left(b\times_\mathsf{V}v,b\times_\mathsf{W}w\right)\\
&= a\times_\mathsf{Z}z +_\mathsf{Z} b\times_\mathsf{Z}z
\end{align}








\end{enumerate}
\end{enumerate}
\newpage{}
\subsection{}
\begin{enumerate}
\setcounter{enumi}{4}
\item Prove that $A + A^t$ is symmetric for any square matrix A. 
\\
Claim: $A+A^t = \left(A + A^t)^t$
\begin{align*}
%\[
 A_{n,n} &=
 \begin{pmatrix}
  a_{1,1} & a_{1,2} & \cdots & a_{1,n} \\
  a_{2,1} & a_{2,2} & \cdots & a_{2,n} \\
  \vdots  & \vdots  & \ddots & \vdots  \\
  a_{n,1} & a_{n,2} & \cdots & a_{n,n}
 \end{pmatrix}
%\]
\\
%\[
\\ A^t_{n,n} &=
 \begin{pmatrix}
  a_{1,1} & a_{2,1} & \cdots & a_{n,1} \\
  a_{1,2} & a_{2,2} & \cdots & a_{n,2} \\
  \vdots  & \vdots  & \ddots & \vdots  \\
  a_{1,n} & a_{2,n} & \cdots & a_{n,n}
 \end{pmatrix}
%\]
\\
%\[
 \\A_{n,n} + A^t_{n,n} &=
 \begin{pmatrix}
  a_{1,1}+a_{1,1} & a_{1,2} + a_{2,1} & \cdots & a_{1,n} + a_{n,1} \\
  a_{2,1} + a_{1,2} & a_{2,2} + a_{2,2} & \cdots & a_{2,n} + a_{n,2}  \\
  \vdots  & \vdots  & \ddots & \vdots  \\
  a_{n,1} + a_{1,n}  & a_{n,2} + a_{2,n} & \cdots & a_{n,n} + a_{n,n}
 \end{pmatrix}
%\]
\\
%\[
\\ \left(A_{n,n} + A^t_{n,n}\right)^t &=
 \begin{pmatrix}
  a_{1,1}+a_{1,1} & a_{2,1} + a_{1,2} & \cdots & a_{n,1} + a_{1,n} \\
  a_{1,2} + a_{2,1} & a_{2,2} + a_{2,2} & \cdots & a_{n,2} + a_{2,n}  \\
  \vdots  & \vdots  & \ddots & \vdots  \\
  a_{1,n} + a_{n,1}  & a_{2,n} + a_{n,2} & \cdots & a_{n,n} + a_{n,n}
 \end{pmatrix}
%\]
\end{align*}
\newpage{}
\setcounter{enumi}{9}
\item Prove that $\mathsf{W}_1 = \big\{\left(a_1,a_2,\dots,a_n\right)
  \in \mathsf{F}^n \colon a_1 + a_2 + \cdots + a_n = 0 \big\}$ is a
  subspace of $\mathsf{F}^n$, but $\mathsf{W}_2 =
  \big\{\left(a_1,a_2,\dots,a_n\right) \in \mathsf{F}^n \colon a_1 +
  a_2 + \cdots + a_n = 1\big\}$ is not.
\begin{enumerate}[(a)]
\item $0_\mathsf{V} \in \mathsf{W}$
\\
Suppose $\left(a_1,a_2,\dots, a_n\right) \in \mathsf{W}_1$\\
Suppose $\left(b_1,b_2,\dots, b_n\right) \in \mathsf{W}_1$ such that
$b_i=0$ for integer $i\in [1,n]$
\\
Claim: $\left(a_1,a_2,\dots, a_n\right) +  \left(b_1,b_2,\dots,
  b_n\right) = \left(a_1,a_2,\dots, a_n\right)$
\begin{align}
\left(a_1,a_2,\dots,a_n\right) + \left(b_1,b_2,\dots,a_n\right) &=
\left(a_1+b_1,a_2+b_2,\dots,a_n+b_n\right)\\
&= \left(a_1+0,a_2+0,\dots,a_n+0\right)\\
&= \left(a_1,a_2,\dots,a_n\right)
\end{align}
\item $\forall\; x,y \in \mathsf{W}, \; x + y \in \mathsf{W}$
\\
Suppose $\left(a_1,a_2,\dots, a_n\right), \left(b_1,b_2,\dots, b_n\right) \in \mathsf{W}_1$ 
\\
Claim: $\left(a_1+b_1,a_2+b_2,\dots,a_3+b_3\right) \in \mathsf{W}_1$
\begin{equation}
\left(a_1,a_2,\dots,a_n\right) + \left(b_1,b_2,\dots,b_n\right) &=
\left(a_1 + b_1,a_2+b_2,\dots,a_n+b_n\right)\\
\end{equation}
\begin{equation}
\sum\limits_{k=1}^n \left(a_k\right) + \sum\limits_{k=1}^n
\left(b_k\right) = \sum\limits_{k=1}^n \left(a_k+b_k\right) = 0
\end{equation}
\begin{equation}
\therefore \left(a_1+b_1,a_2+b_2,\dots,a_n+b_n\right) \in \mathsf{W}_1
\end{equation}
\item $\forall\; a \in F,\, x \in \mathsf{W}\; ax \in \mathsf{W}$
\\
Suppose $c\in F,\; \left(a_1,a_2,\dots,a_n\right) \in \mathsf{W}_1$
\\
Claim: $\left(ca_1,ca_2,\dots,ca_n\right) \in \mathsf{W}_2$
\begin{equation}
c\left(a_1,a_2,\dots,a_n\right) = \left(ca_1,ca_2,\dots,ca_n\right)
\end{equation}
\begin{equation}
\sum\limits_{k=1}^n \left(ca_k\right) = c\sum\limits_{k=1}^n
\left(a_k\right) = 0
\end{equation}
\begin{equation}
\therefore \left(ca_1,ca_2,\dots,ca_n+\right) \in \mathsf{W}_1
\end{equation}
\end{enumerate}
\newpage{}
Suppose
$\left(a_1,a_2,\dots,a_n\right),\left(b_1,b_2,\dots,b_n\right) \in \mathsf{W}_2$
\\
Claim: $\left(a_1+b_1,a_1+b_2,\dots,a_n+b_n\right) \notin
\mathsf{W}_2\right)$
\begin{equation}
\left(a_1,a_2,\dots,a_n\right + \left(b_1,b_2,\dots,b_n\right) = \left(a_1+b_1,a_2+b_2,\dots,a_n+b_n\right)
\end{equation}
\begin{equation}
\sum\limits_{k=1}^n\left(a_k\right) +
\sum\limits_{k=1}^n\left(b_k\right) = \sum\limits_{k=1}^n\left(a_k +
  b_k\right) = 2
\end{equation}
\begin{equation}
\therefore \left(a_1+b_1,a_2+b_2,\dots,a_n+b_n\right) \notin \mathsf{W}_2
\end{equation}
%%%%%%%%%%%%%%%%%%%%%%%%%%%%%%%%%%%%%%%%%%%%%%
\setcounter{enumi}{19}
\item Prove that if $\mathsf{W}$ is a subspace of a vector space $\mathsf{V}$ and
  $w_1, w_2,\dots,w_n$ are in $\mathsf{W}$, then $a_1w_1 + a_2w_2 +
  \cdots + a_nw_n \in \mathsf{W}$ for any scalars $a_1,a_2,\dots,a_n$.
\\
Suppose $a_1,a_2,\dots,a_n \in F,\; w_1,w_2,\dots,w_n \in \mathsf{W}$
\\
Claim: $\forall a_1,a_2,\dots,a_n \in F$ and $\forall
w_1,w_2,\dots,w_n \in \mathsf{W}, a_1w_1 + a_2w_2+\cdots+a_nw_n\in
\mathsf{W}$.
For every integer $i \in \left[i,n\right], a_iw_i \in \mathsf{W}$ by
theorem 1.3.c
\begin{equation}
\implies a_1w_1 + a_2w_2 +\cdots+a_nw_n\; \text{(by theorem 1.2.c)}
\end{equation}

%%%%%%%%%%%%%%%%%%%%%%%%%%%%%%%%%%%%%%%%555555555555555555555555555555555
\newpage{}
\setcounter{enumi}{22}
\item Let $\mathsf{W}_1$ and $\mathsf{W}_2$ be subspaces of a vector
  space $\mathsf{V}$.
\begin{enumerate}[(a)]
\item Prove that $\mathsf{W}_1 + \mathsf{W}_2$ is a subspace of
  $\mathsf{V}$ that contains both $\mathsf{W}_1$ and $\mathsf{W}_2$.
\item Prove that any subspace of $\mathsf{V}$ that contains both
  $\mathsf{W}_1$ and $\mathsf{W}_2$ must also contain $\mathsf{W}_1 +
  \mathsf{W}_2$ 
\end{enumerate}
\\
Claim: $\mathsf{W}_1 + \mathsf{W}_2 \subseteq \mathsf{V}$
\\
Suppose $x \in \mathsf{W}_1 + \mathsf{W}_2$ such that $z = x + y $ and
$x \in \mathsf{W}_1, \; y  \in \mathsf{W}_2$
\begin{align}
\mathsf{W}_1 \in \mathsf{V}\\
\mathsf{W}_2 \in \mathsf{V}\\
\implies x \in \mathsf{V} \\
 \implies y \in \mathsf{V}\\
\implies x + y \in \mathsf{V}\\
\implies \mathsf{W}_1 + \mathsf{W}_2 \subseteq \mathsf{V}
\end{align}
\begin{enumerate}[(i)]
\item $0 \in \mathsf{W}_1$
\begin{align}
0 \in \mathsf{W}_1\\
0 \in \mathsf{W}_2
\end{align}
Suppose $z \in \mathsf{W}_1 + \mathsf{W}_2$ such that $z = x+y$ and
$x=y=0$ 
\begin{align}
\implies z = 0 + 0 =0
\end{align}
\item $x + y \in \mathsf{W}$ when $ x \in \mathsf{W}$ and $y \in
  \mathsf{W}$
Suppose $z_1,z_2 \in \mathsf{W}$ \\such that $z_1 = x +  y,z_2  =  a +
b$\\
Claim: $z1_1 + z_2 \in \mathsf{W}_1  + \mathsf{W}_2$
\begin{align}
z_1 + z_2 &= x+ y + a+b\\
&= \left(x+a\right) + \left(y+b\right)
\end{align}
\begin{align}
x+a\in \mathsf{W}_1\\
y+b\in \mathsf{W}_2
\end{align}
\begin{equation}
\implies z_1 + z_2 \in \mathsf{W}_2
\end{equation}
\item $cx \in \mathsf{W}$ when $z \in \mathsf{W}_1 + \mathsf{W}_2$
  such that $z=x+y$
\\
Suppose $c \in F$ and $z \in \mathfs{W}_1 + \mathsf{W}_2$ such that $z
= x+y$
\begin{align}
cz &= c\left(x+y\right)\\
&= cx +cy 
\end{align}
\begin{equation}
cx \in \mathsf{W}_1
\end{equation}
\begin{equation}
cy \in \mathsf{W}_2
\end{equation}
\begin{equation}
\implies cz \in \mathsf{W}_1 + \mathsf{W}_2
\end{equation}
%%%% ]b
Suppose $\mathsf{X}$ is a subspace of $\mathsf{V}$ and $\mathsf{W}_1
\subseteq \mathsf{X}$, and $\mathsf{W}_2 \subseteq \mathsf{X}$
\\
Claim: $\mathsf{W}_1 + \mathsf{W}_2 \subseteq \mathsf{X}$
\\
Suppose $z \in \mathsf{W}_1 + \mathsf{W}_2$ such that $z = x+y$ for $x
\in \mathsf{W}_1 $ and $y \in \mathsf{W}_2$
\begin{equation}
x \in \mathsf{W}_1 \implies x \in \mathsf{X}
\end{equation}
\begin{equation}
y \in \mathsf{W}_2 \implies y \in \mathsf{X}
\end{equation}
\begin{equation}
\implies x + y \in \mathsf{X}
\end{equation}
\end{enumerate}
\newpage{}
\setcounter{enumi}{24}
\item Let $\mathsf{W}_1$ denote the set of all polynomials $f(x)$ in
  $\mathsf{P}(F)$ such that in the representation
\[
f(x) = a_nx^n +a_{n-1}x^{n-1} + \cdots + a_1x+a_0,
\]
we have $a_i = 0$ whenever $i$ is even. Likewise let $\mathsf{W}_2$
denote the set of all polynomials $g(x)$ in $\mathsf{P}(F)$ such that
in the representation 
\[
g(x) = b_nx^n +b_{n-1}x^{n-1} + \cdots + b_1x+b_0,
\]
we have $b_i = 0$ whenever $i$ is odd. Prove that $\mathsf{P}(F) =
\mathsf{W_1} \oplus \mathsf{W}_2$.
\begin{equation}
\mathsf{W}_1 = \left\{x\in \mathsf{P}(F) \colon c_k = 0\; \forall\;
  \text{integers}\; k\in\left[0,2\right] \text{such that}\; 2\mid\left(k+1\right)\right\}
\end{equation}
\begin{equation}
\mathsf{W}_2 = \left\{x\in \mathsf{P}(F) \colon c_k = 0\; \forall\;
  \text{integers}\; k\in\left[0,2\right] \text{such that}\; 2\mid k\right\}
\end{equation}
\begin{enumerate}[(a)]
\item $0_\mathsf{V} \in \mathsf{W}$
\\
Suppose $\left(a_nx^n + a_{n-2}x^{n-2}+\cdots+ a_2x^2 + a_0\right) \in
\mathsf{W}_1,\\ \left(b_mx^m + b_{m-2}x^{m-2}+\cdots+ b_3x^3 + b_1x\right) \in
\mathsf{W}_1$ such that $b_i = 0$ for every integer
$i\in\left[1,m\right]$
\\
Claim: $\left(a_nx^n + a_{n-2}x^{n-2}+\cdots+ a_2x^2 + a_0\right) +
\left(b_mx^m + b_{m-2}x^{m-2}+\cdots+ b_3x^3 + b_1x\right) =
\left(a_nx^n + a_{n-2}x^{n-2}+\cdots+ a_2x^2 + a_0\right)$
\begin{align}
\left(a_nx^n +\cdots+ a_2x^2 + a_0\right) &+
\left(b_mx^m +\cdots+ b_3x^3 + b_1x\right)\\
\left(a_nx^n + a_{n-2}x^{n-2}+\cdots+ a_2x^2 + a_0\right) &+ \left(0x^m
  + 0x^{m-2}+\cdots+ 0x^3 + 0x\right)\\
\left(a_nx^n + a_{n-2}x^{n-2}+\cdots+ a_2x^2 + a_0\right) &+ \left(0
  + 0 +\cdots+ 0 + 0\right)
\end{align}
\begin{equation}
\left(a_nx^n +\cdots+ a_0\right) +
0_{\mathsf{W}_1} = \left(a_nx^n + a_{n-2}x^{n-2}+\cdots+ a_2x^2 + a_0\right)
\end{equation}
\newpage{}
%%%%%%%%%%%%%%%%%%%%%%%%%%%%
\item $ \forall\; x,y \in \mathsf{W},\; x + y \in \mathsf{W}$
\\
Suppose $\left(a_nx^n + a_{n-2}x^{n-2}+\cdots+ a_3x^3 + a_1x\right),$ \\
\tab \tab \tab $\left(b_mx^m + b_{m-2}x^{m-2}+\cdots+ b_3x^3 + b_1x\right) \in
\mathsf{W}_1$
\\
Claim: $\left(a_nx^n + a_{n-2}x^{n-2}+\cdots+ a_3x^3 + a_1x\right) +$
\\ \tab \tab \tab $\left(b_mx^m + b_{m-2}x^{m-2}+\cdots+ b_3x^3 +
  b_1x\right) \in \mathsf{W}_1$
\paragraph{} 
Without loss of generality assume $n\leq m \implies \exists\; k \in
\left[0,\frac{n-1}{2}\right]$ such that $m= n -2k$
\begin{multline}
\left(a_nx^n + a_{n-2}x^{n-2}+\cdots+ a_3x^3 + a_1x\right) + \\
\left(b_mx^m + b_{m-2}x^{m-2}+\cdots+ b_3x^3 +
  b_1x\right)
\end{multline}
\begin{multline}
\left(a_nx^n + a_{n-2}x^{n-2}+\cdots+\\
  \left(a_{n-2k}+b_{n-2k}\right)\left(x^{n-2k}\right)+ 
  \left(a_{n-2k-2}+b_{n-2k-2}\right)\left(x^{n-2k-2}\right) +\cdots+ \\
\left(a_3+b_3\right)x^3 + \left(a_1+b_1\right)x\right)
\end{multline}
\begin{multline}
\implies \left(a_nx^n + a_{n-2}x^{n-2}+\cdots+ a_3x^3 + a_1x\right) +\\
\left(b_mx^m + b_{m-2}x^{m-2}+\cdots+ b_3x^3 +b_1x\right) \in \mathsf{W}_1
\end{multline}
%%%%%%%%%%%%%%%%%%%%%%%%%%%%
\item $ \forall\; a\in F, x \in \mathsf{W}, \; ax \in \mathsf{W}$
\\
Suppose $c\in F$ and $\left(a_nx^n + a_{n-2}x^{n-2}+\cdots+ a_3x^3 +
  a_1x\right)\in \mathsf{W}_1$\\
Claim: $ c\left(a_nx^n + a_{n-2}x^{n-2}+\cdots+ a_3x^3 + a_1x\right)
\in \mathsf{W}_1$
\begin{multline}
c\left(a_nx^n + a_{n-2}x^{n-2}+\cdots+ a_3x^3 + a_1x\right)\\
= \left(ca_nx^n + ca_{n-2}x^{n-2}+\cdots+ ca_3x^3 + ca_1x\right)
\end{multline}
\begin{equation}
ca_n,ca_{n-2},\dots,ca_1 \in F
\end{equation}
\begin{equation}
\implies c\left(a_nx^n + a_{n-2}x^{n-2}+\cdots+ a_3x^3 + a_1x\right) \in \mathsf{W}_1
\end{equation}
%%%%%%%%%%%%%%%%%%%%%%%%%%%%%%%%% 
\end{enumerate}

%%%%%%%%%%%%%%%%%%%%%%%%5
%%%%%%%%%%%%%%%%%%%%%%%%%
\begin{enumerate}[(a)]
\item $0_\mathsf{V} \in \mathsf{W}$
\\
Suppose $\left(a_nx^n + a_{n-2}x^{n-2}+\cdots+ a_2x^2 + a_0\right) \in
\mathsf{W}_2,\\ \left(b_mx^m + b_{m-2}x^{m-2}+\cdots+ b_2^2 + b_0\right) \in
\mathsf{W}_2$ such that $b_i = 0$ for every integer
$i\in\left[1,m\right]$
\\
Claim: $\left(a_nx^n + a_{n-2}x^{n-2}+\cdots+ a_2x^2 + a_0\right) +
\left(b_mx^m + b_{m-2}x^{m-2}+\cdots+ b_2x^2 + b_0\right) =
\left(a_nx^n + a_{n-2}x^{n-2}+\cdots+ a_2x^2 + a_0\right)$
\begin{align}
\left(a_nx^n +\cdots+ a_2x^2 + a_0\right) &+
\left(b_mx^m +\cdots+ b_2x^2 + b_0\right)\\
\left(a_nx^n + a_{n-2}x^{n-2}+\cdots+ a_2x^2 + a_0\right) &+ \left(0x^m
  + 0x^{m-2}+\cdots+ 0x^2 + 0\right)\\
\left(a_nx^n + a_{n-2}x^{n-2}+\cdots+ a_2x^2 + a_0\right) &+ \left(0
  + 0 +\cdots+ 0 + 0\right)
\end{align}
\begin{equation}
\left(a_nx^n +\cdots+ a_0\right) +
0_{\mathsf{W}_1} = \left(a_nx^n + a_{n-2}x^{n-2}+\cdots+ a_2x^2 + a_0\right)
\end{equation}

%%%%%%%%%%%%%%%%%%%%%%%%%%%%
\item $ \forall\; x,y \in \mathsf{W},\; x + y \in \mathsf{W}$
\\
Suppose $\left(a_nx^n + a_{n-2}x^{n-2}+\cdots+ a_3x^3 + a_1x\right),$ \\
\tab \tab \tab $\left(b_mx^m + b_{m-2}x^{m-2}+\cdots+ b_3x^3 + b_1x\right) \in
\mathsf{W}_2$
\\
Claim: $\left(a_nx^n + a_{n-2}x^{n-2}+\cdots+ a_2x^2 + a_0\right) +$
\\ \tab \tab \tab $\left(b_mx^m + b_{m-2}x^{m-2}+\cdots+ b_2x^2 +
  b_0\right) \in \mathsf{W}_2$
\paragraph{} 
Without loss of generality assume $n\leq m \implies \exists\; k \in
\left[0,\frac{n}{2}\right]$ such that $m= n -2k$
\begin{multline}
\left(a_nx^n + a_{n-2}x^{n-2}+\cdots+ a_2x^2 + a_0\right) + \\
\left(b_mx^m + b_{m-2}x^{m-2}+\cdots+ b_2x^2 +
  b_0\right)
\end{multline}
\begin{multline}
\left(a_nx^n + a_{n-2}x^{n-2}+\cdots+\\
  \left(a_{n-2k}+b_{n-2k}\right)\left(x^{n-2k}\right)+ 
  \left(a_{n-2k-2}+b_{n-2k-2}\right)\left(x^{n-2k-2}\right) +\cdots+ \\
\left(a_2+b_2\right)x^2 + \left(a_0+b_0\right)\right)
\end{multline}
\begin{multline}
\implies \left(a_nx^n + a_{n-2}x^{n-2}+\cdots+ a_2x^2 + a_0\right) +\\
\left(b_mx^m + b_{m-2}x^{m-2}+\cdots+ b_2x^2 +b_0\right) \in \mathsf{W}_2
\end{multline}
%%%%%%%%%%%%%%%%%%%%%%%%%%%%
\item $ \forall\; a\in F, x \in \mathsf{W}, \; ax \in \mathsf{W}$
\\
Suppose $c\in F$ and $\left(a_nx^n + a_{n-2}x^{n-2}+\cdots+ a_2x^2 +
  a_0\right)\in \mathsf{W}_2$\\
Claim: $ c\left(a_nx^n + a_{n-2}x^{n-2}+\cdots+ a_2x^2 + a_0\right)
\in \mathsf{W}_2$
\begin{multline}
c\left(a_nx^n + a_{n-2}x^{n-2}+\cdots+ a_2x^2 + a_0\right)\\
= \left(ca_nx^n + ca_{n-2}x^{n-2}+\cdots+ ca_2x^2 + ca_0\right)
\end{multline}
\begin{equation}
ca_n,ca_{n-2},\dots,ca_0 \in F
\end{equation}
\begin{equation}
\implies c\left(a_nx^n + a_{n-2}x^{n-2}+\cdots+ a_2x^2 + a_0\right) \in \mathsf{W}_1
\end{equation}
%%%%%%%%%%%%%%%%%%%%%%%%%%%%%%%%% 
\end{enumerate}
\subsubsection*{Case I}
Suppose $A \in \mathsf{P}(F)$, $A\neq 0_\mathsf{P}$\\
Claim: $A \notin \mathsf{W}_1 \cap \mathsf{W}_2$
\paragraph*{Case (i)} Suppose $A \in \mathsf{W}_2, A = \left(a_nx^n +
  a_{n-2}x^{n-2}+\cdots+ a_2x^2 + a_0\right)$ such that $a_n \neq 0$
By definition of $\mathsf{W}_2$ $2\nmid \left(n+1)\right \implies 2
\mid n$
\\
Claim: $A \notin \mathsf{W}_1$\\
Suppose $A \in \mathsf{W}_1$
\begin{equation}
2\mid n \implies a_n = 0\;\; \lightning \;\text{Contradiction!}
\end{equation}
\begin{equation}
\implies A \notin \mathsf{W}_1
\end{equation}
\newpage{}
\paragraph*{Case (ii)} Suppose $A \in \mathsf{W}_1, A = \left(a_nx^n +
  a_{n-2}x^{n-2}+\cdots+ a_2x^2 + a_1x\right)$ such that $a_n \neq 0$
By definition of $\mathsf{W}_1$ $2\nmid n\right \implies 2
\mid \left(n+1\right)$
\\
Claim: $A \notin \mathsf{W}_2$\\
Suppose $A \in \mathsf{W}_2$
\begin{equation}
2\mid \left(n+\right) \implies a_n = 0\;\; \lightning \;\text{Contradiction!}
\end{equation}
\begin{equation}
\implies A \notin \mathsf{W}_2
\end{equation}
\subsubsection*{Case II}
Suppose $A\in \mathsf{P}(F), A=0$\\
Claim: $A \in \mathsf{W}_1 \cap \mathsf{W}_2$
\begin{align}
0_{\mathsf{W}_1} \in \mathsf{W}_1\\
0_{\mathsf{W}_2} \in \mathsf{W}_2\\
0_{\mathsf{W}_1} = 0_\matsf{F} = 0_{\mathsf{W}_2} \\
0_{\mathsf{W}_1} = 0_{\mathsf{W}_2}\\
\implies 0 \in \mathsf{W}_1 \cap \mathsf{W}_2
\end{align}
\begin{equation}
\implies \mathsf{W}_1 \cap \mathsf{W}_2 = \left\{0\right\}
\end{equation}
Claim: $\mathsf{P}(F) \supseteq \mathsf{W}_1 + \mathsf{W}_2$
\begin{align}
\mathsf{W}_1 \subseteq \mathsf{P}(F)\\
\mathsf{W}_2 \subseteq \mathsf{P}(F)
\end{align}
Suppose $x\in \mathsf{W}_1, y\in \mathsf{W}_2$
\begin{align}
x + y \in \mathsf{P}(F) \\
\implies \mathsf{W}_1 + \mathsf{W}_2 \subseteq \mathsf{P}(F)
\end{align}
Claim: $\mathsf{P}(F) \subseteq \mathsf{W}_\1 + \mathsf{W}_2$\\
Suppose $h \in \mathsf{P}(F)$, $h = \left(a_nx^n +
  a_{n-1}x^{n-1}+\cdots+ a_1x + a_0\right)$
\paragraph{Case (i)} n is odd
\begin{multline}
h = \left(a_nx^n +  a_{n-2}x^{n-2}+\cdots+ a_3x^3 + a_1x\right)+ \\
\left(a_{n-1}x^{n-1} + a_{n-3}x^{n-3}+\cdots+ a_2x^2 + a_0\right)
\end{multline}
\begin{align}
\left(a_nx^n +  a_{n-2}x^{n-2}+\cdots+ a_3x^3 + a_1x\right) \in
\mathsf{W}_2\\
\left(a_{n-1}x^{n-1} + a_{n-3}x^{n-3}+\cdots+ a_2x^2 + a_0\right) \in
\mathsf{W}_1 
\end{align}

\paragraph{Case (ii)} n is even
\begin{multline}
h = \left(a_nx^n +  a_{n-2}x^{n-2}+\cdots+ a_2x^2 + a_0\right)+ \\
\left(a_{n-1}x^{n-1} + a_{n-3}x^{n-3}+\cdots+ a_3x^3 + a_1x\right)
\end{multline}
\begin{align}
\left(a_nx^n +  a_{n-2}x^{n-2}+\cdots+ a_2x^2 + a_0\right) \in
\mathsf{W}_2\\
\left(a_{n-1}x^{n-1} + a_{n-3}x^{n-3}+\cdots+ a_3x^3 + a_1x\right) \in
\mathsf{W}_1 
\end{align}
Claim: \vspace{-1.5em} \[\mathsf{P}(F) =\big\{\left(c_nx^n + c_{n-1}x^{n-1} + \cdots +
    c_1x + x_0\right) \colon c_i \in F \;\text{for integer}\; i \in
  [1,n]\big\} = \mathsf{W}_1 \oplus \mathsf{W}_2\]
\begin{align}
\mathsf{W}_1\cap\mathsf{W}_2 = \left\{0\right\}\\
\mathsf{W}_1 + \mathsf{W}_2 = \mathsf{P}(F)\\
\implies \mathsf{P}(F) = \mathsf{W}_1 \oplus \mathsf{W}_2
\end{align}
\end{enumerate}
\newpage{}
\subsection{}
\begin{enumerate}
A system of linear equations such as 
\begin{align*}
x_1 + 3x_2 + 3x_3 = 4\\
x_1 + 4x_2 + x_3 = 5\\
3x_1 + x_2 + 5x_3 = 2
\end{align*}
Can be rewritten as an augmented matrix, with the left most columns
representing the coefficients of the variables, and the right hand
column representing the right hand side of the linear equations.
\[+
\begin{pmatrix}
1 & 3 & 3 & 4\\
1 & 4 & 1 & 5\\
3 & 1 & 5 & 2
\end{pmatrix}
\]
\setcounter{enumi}{1}
\item Solve the following systems of linear equations, if possible.
%\vspace{-1.5em}
\begin{enumerate}[(a)]
\item \hfill \\
%\begin{align}
%  2x_1&+(-2x_2) &+ (-3x_3) & &= -2 &&&& x_1&+(-x_2)&+(-2x_3) &+ (-x_4) &= -3\\
%3x_1&+(-3x_2) &+ (-2x_3) &+ 5x_4 &= -2 &&&&  3x_1&+(-3x_2)&+(-2x_3) &+ 5x_4 &= -2  \\
%x_1&+(-x_2) &+ (-2x_3) &+ (-x_4) &= -3 &&&&   2x_1&+(-2x_2)&+(-3x_3) & &= -2  \\
%\end{align}
\begin{gmatrix}[p]
2 & -2 & -3 & 0  & -2\\
3 & -3 & -3 & 5  &  7\\
1 & -1 & -2 & -1 & -3
\rowops
\swap{0}{2}
\add[-2]{0}{2}
\add[-3]{0}{1}
\add[-1/4]{1}{3}
\mult{1}{\cdot 1/4}
\end{gmatrix}
$\rightarrow$
\begin{gmatrix}[p]
1 & -1 & -2 & -1 & -3\\
0  & 0 & 1 & 2 & 16\\
0 & 0 & 0 & 0 & 0 
\end{gmatrix}
\paragraph{}
Let $x_2=s, x_4=t$
\begin{align}
x_1 &= -35 + s + 3t \\
x_2 &= s \\
x_3 &= 16 - 2t \\
x_4 &= t 
\end{align}
\item\hfill\\
\begin{gmatrix}[p]
3 & -7 & 4 & 10\\
1 & -2 & 1 & 3\\
2 & -1 & -2 & 6
\rowops
\swap{0}{1}
\add[-3]{0}{1}
\add[-2]{0}{2}
\add[1]{1}{2}
\add[1]{0}{2}
\add[2]{1}{0}
\mult{2}{\cdot -1}
\end{gmatrix}
$\rightarrow$
\begin{gmatrix}[p]
1 & 0 & 0 & -2\\
0 & 1 & 0 & -4\\
0 & 0 & 1 & -3 
\end{gmatrix}
\begin{align}
x_1 &= -2\\
x_2 &= -4\\
x_3 &= -3
\end{align}
\item\hfill\\
\begin{gmatrix}[p]
1 & 2 & -1 & 1 & 5\\
1 & 4 & -3 & 1 & 5\\
2 & 3 & -1 & 4 & 8
\rowops
\add[-1]{0}{1}
\add[-2]{0}{2}
\swap{1}{2}
\end{gmatrix}
$\rightarrow$
\begin{gmatrix}[p]
1 & 2 & -1 & 1 & 5\\
0 & -1 & 1 & 2 & -2\\
0 & 0 & 0 & 0 & 3
\end{gmatrix}
\paragraph{}
Inconsistent; not solvable.
%\newpage{}
\item\hfill\\
\begin{gmatrix}[p]
1 & 2 & 2 & 0 & 2\\
1 & 0 & 8 & 5 & -6\\
1 & 1 & 5 & 5 & -3
\rowops
\add[-1]{0}{1}
\add[-1]{0}{2}
\add[1]{1}{0}
\add[-2]{2}{1}
\add[1]{1}{0}
\add[1]{1}{2}
\end{gmatrix}
$\rightarrow$
\begin{gmatrix}[p]
1 & 0 & 8 & 0 & -16\\
0 & 0 & 0 & -5 & -10\\
0 & -1 & 3 & 0 & -4
\end{gmatrix}
\paragraph{}
Let $s=x_3$
\begin{align}
x_1 &= -8s -16 \\
x_2 &= 3s +4 \\
x_3 &= s \\
x_4 &= 2
\end{align}
\newpage{}
\item\hfill\\
\begin{gmatrix}[p]
1 & 2 & -4 & -1 & 1 & 7\\
-1 & 0 & 10 & -3 & -4 & -16\\
2 & 5 & -5 & -4 & -1 & 2\\
4 & 11 & -7 & -10 & -2 & 7
\rowops
\add[1]{0}{1}
\add[-2]{2}{3}
\add[-2]{0}{2}
\add[-1]{3}{2}
\add[-2]{3}{1}
\mult{1}{\cdot 1/3}
\add[-2]{3}{1}
\add[-1]{2}{0}
\swap{1}{3}
\swap{2}{3}
\add[3]{2}{3}
\end{gmatrix}\\
\hfill\\\hfill\\
$\rightarrow$
\begin{gmatrix}[p]
1 & 0 & -10 & 3 & 0 & -4\\
0 & 1 & 3 & -2 & 0 & 3\\
0 & 0 & 0 & 0 & 1 & 5 
\end{gmatrix}
\paragraph{}
Let $x_3 = s,\; x_4 = t$
\begin{align}
x_1 &=10s - 3t -4\\
x_2 &= -3s + 2t + 3\\
x_3 &= s\\
x_4 &= t
\end{align}
%\newpage{}
\item\hfill\\
\begin{gmatrix}[p]
1 & 2 & 6 & -1 \\
2 & 1 & 1 &  8\\
3 & 1 & -1 & 15\\
1 & 3 & 10 & -5
\rowops
\add[-2]{0}{1}
\add[-3]{0}{2}
\add[-1]{0}{3}
\swap{3}{1}
\add[5]{1}{2}
\add[3]{1}{3}
\add[-4]{2}{1}
\add[-6]{2}{0}
\add[-2]{1}{0}
\add[-1]{2}{3}
\end{gmatrix}
$\rightarrow$
\begin{gmatrix}[p]
1 & 0 & 0 & 3\\
0 & 1 & 0 & 4 \\ 
0 & 0 & 1 & -2
\end{gmatrix}
\begin{align}
x_1 &= 3\\
x_2 &= 4\\
x_3 &= -2
\end{align}
\end{enumerate}
\newpage{}
\setcounter{enumi}{3}
\item\hfill\\
\vspace{-2.5em}
\begin{enumerate}[(a)]
\item $x^3 -3x +5 \? c_1\left(x^3 +2x^2 -x +1\right) + c_2\left(x^3
    +3x^2 -1\right)$
\begin{align}
c_1 +c_2 = 1\\
2c_1 +3c_2 = 0\\
-c_1 = -3\\
c_1 -c_2 = 5
\end{align}
\begin{gmatrix}[p]
1 & 1 & 1\\
2 & 3 & 0\\
-1 & 0 & -3\\
1 &  -1 & 5
\rowops
\add[-2]{0}{1}
\add[-1]{1}{0}
\add[1]{0}{2}
\add[-1]{1}{2}
\add[-1]{0}{3}
\add[-2]{1}{3}
\end{gmatrix}
$\rightarrow$
\begin{gmatrix}[p]
1 & 0 & 3\\
0 & 1 & -2
\end{gmatrix}
\begin{equation}
x^3 -3x +5 = 3\left(x^3 +2x^2 -x +1\right) - 2\left(x^3
    +3x^2 -1\right)
\end{equation}
\item $4x^3 + 2x^2 -6 \? c_1\left(x^3-2x^3+4x+1\right)
  +c_2\left(3x^3-6x^2+x+4\right)$
\begin{align}
c_1 +3c_2 = 4\\
-2c_1 + -6c_2 = 2\\
4c_1 +c_2 = 0\\
c_1 +4c_2 = -6
\end{align}
\begin{gmatrix}[p]
1 & 3 & 4\\
-2 & -6 & 2\\
4 & 1 & 0\\
1 & 4 & 6
\rowops
\add[2]{0}{1}
\end{gmatrix}
$\rightarrow$
\begin{gmatrix}[p]
1 & 3 & 4\\
0 & 0 & 10\\
4 & 1 & 0\\
1 & 4 & 6
\end{gmatrix}
\paragraph{} 
Inconsistent; no linear combinations.
\item $-2x^3 -11x^2+3x+2 \? c_1\left(x^3-ex^2+3x-1\right)
  +c_2\left(2x^3 +x^2 +3x -2\right)$
\begin{align}
c_1 +2c_2 = -2\\
-2c_1 +c_2 = -11\\
3c_1 +3c_2 = 3\\
-c_1 = 2
\end{align}
\begin{gmatrix}[p]
1 & 2 & -2\\
-2 & 1 & -11\\
3 & 3 & 3\\
-1 & 0 & 2
\rowops
\mult{2}{\cdot 1/3}
\add[1]{3}{0}
\add[-2]{3}{1}
\add[1]{3}{2}
\end{gmatrix}
$\rightarrow$
\begin{gmatrix}[p]
0 & 2 & 0\\
0 & 1 & -15\\
0 & 1 & 5\\
-1 & 0 & 2
\end{gmatrix}
\paragraph{}
Inconsistent; no linear combinations.
\item $x^3 +x^2 +2x +13 \? c_1\left(2x^3-2x^2+4x+1\right) +
  c_2\left(x^3 -x^2 +2x+3\right)$
\begin{align}
2c_1 + c_2 =1\\
-3c_1 - c_2 = 1\\
4c_1 +2c_2 = 2\\
c_1+3c_2=13
\end{align}
\begin{gmatrix}[p]
2 & 1 & 1\\
-3 & -1 & 1\\
4 & 2 & 2\\
1 & 3& 13
\rowops
\add[1]{1}{0}
\add[-3]{0}{1}
\mult{0}{\cdot -1}
\mult{1}{\cdot -1}
\add[-4]{0}{2}
\add[-2]{1}{2}
\add[-1]{0}{3}
\add[-3]{1}{3}
\end{gmatrix}
$\rightarrow$
\begin{gmatrix}[p]
1 & 0 & -2 \\
0 & 1 &  5 
\end{gmatrix}
\begin{equation}
x^3 +x^2 +2x +13 = -2\left(2x^3-2x^2+4x+1\right) +
  5\left(x^3 -x^2 +2x+3\right)
\end{equation}
\item $x^3 -8x^2 +4x \? c_1\left(x^3 -2x^2 +3x -1\right) +
  c_2\left(x^3-2x+3\right)$
\begin{align}
c_1+c_2 = 1\\
c_1 = 4\\
3c_1 -2c_2 =1\\
-c_1+3c_2 =0
\end{align}
\begin{gmatrix}[p]
1 & 1 & 1\\
1& 0 & 4\\
3 & -2 & 1\\
-1 & 3 & 0
\rowops
\add[1]{1}{3}
\add[-1]{1}{0}
\mult{3}{\cdot 1/3}
\add[-1]{3}{0}
\end{gmatrix}
$\rightarrow$
\begin{gmatrix}[p]
0 & 0 & ^{-16}/_3\\
1 & 0 & 4\\
3 & -2 & 1\\
0 & 1 & ^4/_3
\end{gmatrix}
\paragraph{}
Inconsistent; no linear combination.
%\newpage{}
\item $6x^3 - 3x^2 +x +2 \? c_1\left(x^3-x^2 + 2x +3\right) +
  c_2\left(2x^3 -3x+1\right)$
\begin{align}
c_1 +c_2 = 6\\
c_1 = 3\\
2c_1 -3c_2 =1\\
3c_1 +c_2 =2
\end{align}
\begin{gmatrix}[p]
1 & 1 & 6\\
1 & 0 & 3\\
2 & -3 & 1\\
3 & 1 & 2
\rowops
\mult{1}{\cdot -1}
\add[-1]{1}{0}
\add[-2]{1}{2}
\mult{2}{\cdot -1/3}
\add[-1]{2}{0}
\end{gmatrix}
$\rightarrow$
\begin{gmatrix}[p]
0 & 0 & ^4/_3\\
1 & 0 & 3\\
0 & 1 & ^5/_3\\
3 & 1 & 2
\end{gmatrix}
\paragraph{}
Inconsistent; no linear combinations.
\end{enumerate}
\newpage{}
\setcounter{enumi}{10}
\item Prove that $\text{span}\left(\left\{x\right\}\right) = \left\{ax
    \colon a \in F\right\}$ for any vector $x$ is a vector
  space. Interpret this result geometrically in $\mathbb{R}^3$.
\\
Suppose $\mathf{V}$ is a vector space.
\\
Claim: $\text{span}\left(\left\{x\right\}\right) \subseteq \left\{ax
    \colon a \in F\right\}$
Suppose $y \in \text{span}\left(\left\{x\right\}\right$
\begin{align}
y &= a_1x +a_2x + \cdots + a_nx \; \text{for}\; a_1,\dots,a_n \in F\\
&=\left(a_1 +a_2 +\cdots+ a_n\right)x
\end{align}
\begin{equation}
\left(a_1+a_2+\cdots+a_n\right) \in F
\end{equation}
\begin{equation}
\implies y \in \left\{ax \colon a \in F\right\}
\end{equation}
Claim: $\text{span}\left(\left\{x\right\}\right) \supseteq \left\{ax
    \colon a \in F\right\}$\\
Suppose $x \in \left\{ax \colon a \in F\right\}$
\begin{equation}
z = bx \;\text{for}\; b \in F
\end{equation}
$bx$ is a linear combination of 1 term.
\begin{equation}
\implies z \in \text{span}\left(\left\{x\right\}\right)
\end{equation}

\setcounter{enumi}{12}
\item Show that is $S_1$ and $S_2$ are subsets of a vector space
  $\mathsf{V}$ such that $S_1 \subseteq S_2$, then $\text{span}(S_1)
  \subseteq \text{span}(S_2)$. In particular, if $S_1\subseteq S_2$
  and $\text{span}(S_1) = \mathsf{V}$, deduce that $\text{span}(S_2) =
  \mathsf{V}$.\\
Claim: $\text{span}(S_1) \subseteq \text{span}(S_2)$\\
Suppose: $y \in \text{span}(S_1)$
\begin{equation}
y = a_1x_1 +a_2x_2 +\cdots+ a_nx_n \;\text{for}\; a_1,a_2,\dots,a_n
\in F\; \text{and}\; x_1,x_2,\dots,x_n \in S_1
\end{equation}
\begin{equation}
\implies S_1 \subseteq S_2\; \forall \;\text{integers}\; i \in [1,n],
x_i \in S_2
\end{equation}
\begin{equation}
\forall\; a_1,a_2,\dots,a_n \in F,\;
a_1x_1 +a_2x_2 +\cdots+ a_nx_n \in \text{span}(S_2)
\end{equation}
\begin{equation}
\implies \text{span}(S_1) \subseteq \text{span}(S_2)
\end{equation}
Claim: $\text{span}(S_2) \subseteq \mathsf{V}$
\\
Suppose $y \in \text{span}(S_2)$
\begin{equation}
y = a_1x_1 +a_2x_2 +\cdots+ a_nx_n, \; \forall \; a_1x_1
,a_2x_2,\dots,a_n \in F \; \text{and} \; x_1,x_2,\dots,x_n \in S_2
\end{equation}
Given $S_2 \subseteq \mathsf{V}$
\begin{align}
x_1,x_2,\dots,x_n \in \mathsf{V} &\implies a_1x_1 + a_2x_2 +\cdots+
a_nx_n \in \mathsf{V}\\
&\implies \text{span}(S_2) \subseteq \mathsf{V}
\end{align}
\end{enumerate}
\end{spacing}
\end{document}

