Prove the following variation of the second-deriviative test for the case of $n=2:$ define
\[
D - \left[\a\right]\left[\b\right]-\left[\c\right]^2
\]
\begin{enumerate}
\item If $D > 0$ and $\a >0,$ then $f$ has a local minimum at $p.$
\item If $D > 0$ and $\a <0,$ then $f$ has a local maximum at $p.$
\item If $D < 0,$ then $f$ has no local extremum at $p.$
\item If $D = 0,$ then the test is inconsclusive.
\end{enumerate}
\begin{gather}
\det{
\begin{pmatrix}[1.5]
\a -\lambda & \c\\
\c & \b - \lambda
\end{pmatrix}} = 0\\
\implies \lambda^2 - \lambda\left(\a + \b\right) + \a\b -\left(\c\right)^2
\end{gather}
\begin{multline}
\label{fuck you}
\implies 2\lambda = \a + \b \pm\\ \pm \sqrt{\left(\a\right)^2-\a\b+\left(\b\right)^2+4\left(\c\right)^2}
\end{multline}
\begin{enumerate}
\item Based on the assumptions and equation \ref{fuck you},
  $\lambda_1$ and $\lambda_2$ are stricly positive therefore $f$ has
  a local minimum at $p.$
\item Based on the assumptions and equation \ref{fuck you},
  $\lambda_1$ and $\lambda_2$ are stricly negative therefore $f$ has
  a local maximum at $p.$
\item Based on the assumptions and equation \ref{fuck you},
  $\lambda_1>0$ and $\lambda_2<0$ therefore $f$ has no local extrema
  at $p.$
\item 
\begin{equation}
\lambda^2 -\lambda\left(\a +\b\right)
\end{equation}
\begin{align}
\label{double fuck} \implies \lambda_1 &= 0\\
\lambda_2 &= \a + \b
\end{align}
Based on the assumptions and equation \ref{double fuck} the test is
inconclusive because at least one of the eigen values is zero.
\end{enumerate}
