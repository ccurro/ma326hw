For each of the give quadratic forms $K$ on a real inner product space
$\mathsf{V},$ find a bilinear form $H$ such that $K(x) = H(x,x)$ for
all $x \in \mathsf{V}.$ Then find an orthonormal basis $\beta$ for
$\mathsf{V}$ such that $\psi_\beta(H)$ is a diagonal matrix.
\begin{enumerate}
\item $K \colon \mathsf{R}^2 \to \mathbb{R}$ defined by
  $K\begin{pmatrix}t_1\\t_2\end{pmatrix} = -2t_1^2 + 4t_1t_2 +t_2^2$

Let $ A = \begin{pmatrix} -2&2\\2 & 1\end{pmatrix}$
\begin{equation}
\det{
\begin{pmatrix}
-2 & -t & 2\\
2  & 1- t
\end{pmatrix}
} = (t+3)(t-2)
\end{equation}
\begin{align}
\implies \lambda_1 &= -3\\
\lambda_2 &= 2
\end{align}
\begin{itemize}
\item For $\lambda_1= -3$
\begin{gather}
\begin{pmatrix}
1 & 2\\
2 & 4\\
\end{pmatrix}
\begin{pmatrix}
x_1\\x_2
\end{pmatrix}
=
\begin{pmatrix}
0\\0
\end{pmatrix}\\
\begin{pmatrix}
1 & 2\\
0 & 0
\end{pmatrix}
\begin{pmatrix}
x_1\\x_2
\end{pmatrix}
=
\begin{pmatrix}
0\\0
\end{pmatrix}\\
\implies x_1 + 2x_2 = 0\\
\implies E_{\lambda_1} = \left\{t\begin{pmatrix}1\\\mfrac{-1}{2}
  \end{pmatrix}\colon t \in \mathbb{R}\right\}
\end{gather}
\item For $\lambda_2 = 2$
\begin{gather}
\begin{pmatrix}
-4 & 2\\
2 & -1
\end{pmatrix}
\begin{pmatrix}
x_1\\x_2
\end{pmatrix}
=
\begin{pmatrix}
0\\0
\end{pmatrix}\\
\begin{pmatrix}
0 & 0\\
2 & -1
\end{pmatrix}
\begin{pmatrix}
x_1\\x_2
\end{pmatrix}
=
\begin{pmatrix}
0\\0
\end{pmatrix}\\
\implies 2x_1-x_2 = 0\\
\implies E_{\lambda_2} = \left\{
t\begin{pmatrix}\mfrac{1}{2}\\1\end{pmatrix}\colon t \in \mathbb{R}
\right\}
\end{gather}
\end{itemize}
\begin{gather}
\left\langle\begin{pmatrix}1\\\mfrac{-1}{2} \end{pmatrix}, 
\begin{pmatrix}\mfrac{1}{2}\\1\end{pmatrix}\right\rangle=0\\
\norm{\begin{pmatrix}1\\\mfrac{-1}{2} \end{pmatrix}} =
\norm{\begin{pmatrix}\mfrac{1}{2}\\1\end{pmatrix}} = \frac{\sqrt{5}}{2}
\end{gather}
\begin{align}
\implies v_1 &=
\frac{2}{\sqrt{5}}\begin{pmatrix}1\\\mfrac{-1}{2} \end{pmatrix}\\
v_2 & = \frac{2}{\sqrt{5}}\begin{pmatrix}\mfrac{1}{2}\\1\end{pmatrix}
\end{align}
\begin{equation}
\implies \beta = \left\{\frac{2}{\sqrt{5}}\begin{pmatrix}1\\\mfrac{-1}{2} \end{pmatrix}, \frac{2}{\sqrt{5}}\begin{pmatrix}\mfrac{1}{2}\\1\end{pmatrix}\right\}
\end{equation}
\begin{gather}
\implies Q = 
\frac{2}{\sqrt{5}} = \begin{pmatrix}
1 & \mfrac{1}{2}\\
\mfrac{-1}{2} & 1
\end{pmatrix}\\
Q^tAQ = \psi_\beta(H) = \frac{4}{5}\begin{pmatrix}
\mfrac{-15}{4} & 0\\
0 & \mfrac{5}{2}
\end{pmatrix}
= \begin{pmatrix}
-3 & 0\\
0 & 2
\end{pmatrix}
\end{gather}
\item $K \colon \mathsf{R}^2 \to \mathbb{R}$ defined by $K\begin{pmatrix}t_2\\t_2
  \end{pmatrix} = 7t_1^2-8t_1t_2+t_2^2$ 

Let $A = \begin{pmatrix}
7 & -4\\
-4 & 1
\end{pmatrix}$
\begin{gather}
\det{
\begin{pmatrix}
7-t & -4\\
-4 & 1-t 
\end{pmatrix}
} = (t+1)(t-9)
\end{gather}
\begin{align}
\implies \lambda_1 &= -1\\
\lambda_2 &= 9
\end{align}
\begin{itemize}
\item For $\lambda_1 = -1$
\begin{gather}
\begin{pmatrix}
8 & -4\\
-4 & 2
\end{pmatrix}
\begin{pmatrix}
x_1\\x_2
\end{pmatrix}
=
\begin{pmatrix}
0\\0
\end{pmatrix}\\
\begin{pmatrix}
0 & 0\\
-4 & 2
\end{pmatrix}
\begin{pmatrix}
x_1\\x_2
\end{pmatrix}
=
\begin{pmatrix}
0\\0
\end{pmatrix}\\
\implies -4x_1 + 2x_2 =0 \\
x_2 = 2x_1\\
\implies E_{\lambda_1} = \left\{
t\begin{pmatrix}1\\\mfrac{1}{2}
\end{pmatrix}\colon t \in \mathbb{R}
\right\}
\end{gather}
\item For $\lambda_2 = 9$ 
\begin{gather}
\begin{pmatrix}
-2 & -4\\
-4 & -8
\end{pmatrix}
\begin{pmatrix}
x_1\\x_2
\end{pmatrix}
=
\begin{pmatrix}
0\\0
\end{pmatrix}\\
\begin{pmatrix}
-2 & -4\\
0 & 0
\end{pmatrix}
\begin{pmatrix}
x_1\\x_2
\end{pmatrix}
=
\begin{pmatrix}
0\\0
\end{pmatrix}\\
\implies -2x_1 -4x_2 =0\\
-x_1 = 2x_2\\
\implies E_{\lambda_2} = \left\{
t\begin{pmatrix}
\mfrac{-1}{2}\\1
\end{pmatrix}
\colon t \in \mathbb{R}
\right\}
\end{gather}
\end{itemize}
\begin{gather}
\left\langle\begin{pmatrix}1\\\mfrac{-1}{2} \end{pmatrix}, 
\begin{pmatrix}\mfrac{1}{2}\\1\end{pmatrix}\right\rangle=0\\
\norm{\begin{pmatrix}1\\\mfrac{-1}{2} \end{pmatrix}} =
\norm{\begin{pmatrix}\mfrac{1}{2}\\1\end{pmatrix}} = \frac{\sqrt{5}}{2}
\end{gather}
\begin{align}
\implies v_1 &=
\frac{2}{\sqrt{5}}\begin{pmatrix}1\\\mfrac{-1}{2} \end{pmatrix}\\
v_2 & = \frac{2}{\sqrt{5}}\begin{pmatrix}\mfrac{1}{2}\\1\end{pmatrix}
\end{align}
\begin{equation}
\implies \beta =
\left\{\frac{2}{\sqrt{5}}\begin{pmatrix}1\\\mfrac{-1}{2} \end{pmatrix},
 \frac{2}{\sqrt{5}}\begin{pmatrix}\mfrac{1}{2}\\1\end{pmatrix}\right\}
\end{equation}
\begin{gather}
\implies Q = 
\frac{2}{\sqrt{5}} = \begin{pmatrix}
1 & \mfrac{1}{2}\\
\mfrac{-1}{2} & 1
\end{pmatrix}\\
Q^tAQ = \psi_\beta(H) = \frac{4}{5}\begin{pmatrix}
\mfrac{45}{4} & 0\\
0 & \mfrac{-5}{4}
\end{pmatrix}
= \begin{pmatrix}
9 & 0\\
0 & -1
\end{pmatrix}
\end{gather}
\item $K\colon \mathsf{R}^3 \to \mathbb{R}$ defined
  by$K\begin{pmatrix}t_1\\t_2\\t_3\end{pmatrix} = 3t_1^2 +3t_2^2 +
  3t_3^2 -2t_1t_3$ 

Let $A = \begin{pmatrix}
3 & 0 & -1\\
0 & 3 & 0\\
-1 & 0 & 3
\end{pmatrix}$
\begin{equation}
\det{
\begin{pmatrix}
3-t & 0 & -1\\
0 & 3- t & 0\\
-1 & 0 & 3-t
\end{pmatrix}
} = t^3+9t^2-26t+24
\end{equation}
\begin{align}
\implies \lambda_1 &= 2\\
\lambda_2 &= 3\\
\lambda_3 &= 4
\end{align}
\begin{itemize}
\item For $\lambda_1 = 2$
\begin{gather}
\begin{pmatrix}
1 & 0 & -1\\
0 & 1 & 0\\
-1 & 0 & -1
\end{pmatrix}
\begin{pmatrix}
x_1\\x_2\\x_3
\end{pmatrix}
=
\begin{pmatrix}
0\\0\\0
\end{pmatrix}\\
\begin{pmatrix}
0 & 0 & 0\\
0 & 1 & 0\\
-1 & 0 & -1
\end{pmatrix}
\begin{pmatrix}
x_1\\x_2\\x_3
\end{pmatrix}
=
\begin{pmatrix}
0\\0\\0
\end{pmatrix}\\
\implies x_2 = 0\\
-x_1 + x_3 = 0\\
\implies x_3 = x_1\\
\implies E_{\lambda_1} = \left\{
t\begin{pmatrix}
1\\0\\1
\end{pmatrix}\colon t \in \mathbb{R}
\right\}
\end{gather}
\item For $\lambda_2 = 3$
\begin{gather}
\begin{pmatrix}
0 & 0 & -1\\
0 & 0 & 0\\
-1 & 0 & 0
\end{pmatrix}
\begin{pmatrix}
x_1\\x_2\\x_3
\end{pmatrix}
=
\begin{pmatrix}
0\\0\\0
\end{pmatrix}\\
\implies x_1=0\\
x_3 = 0\\
x_2 = x_2\\
\implies E_{\lambda_2} =\left\{
t\begin{pmatrix}
0\\1\\0
\end{pmatrix}\colon t \in \mathbb{R}
\right\}
\end{gather}
\item For $\lambda_3 = 4$
\begin{gather}
\begin{pmatrix}
-1 & 0 & -1\\
0 & -1 & 0\\
-1 & 0 & -1
\end{pmatrix}
\begin{pmatrix}
x_1\\x_2\\x_3
\end{pmatrix}
=
\begin{pmatrix}
0\\0\\0
\end{pmatrix}\\
\begin{pmatrix}
0 & 0 & 0\\
0 & -1 & 0\\
-1 & 0 & -1
\end{pmatrix}
\begin{pmatrix}
x_1\\x_2\\x_3
\end{pmatrix}
\begin{pmatrix}
0\\0\\0
\end{pmatrix}\\
\implies x_2 = 0\\
-x_1 = x_3\\
\implies E_{\lambda_3} = \left\{
t\begin{pmatrix}
1\\0\\-1
\end{pmatrix}\colon t \in \mathbb{R}
\right\}
\end{gather}
\end{itemize}
\begin{align}
\implies v_1 &= \frac{1}{\sqrt{2}}\begin{pmatrix}1\\0\\1\end{pmatrix}\\
v_2 &= \begin{pmatrix}0\\1\\0\end{pmatrix}\\
v_3 &= \frac{1}{\sqrt{2}} \begin{pmatrix}1\\0\\-1\end{pmatrix}
\end{align}
\begin{equation}
\implies \beta = \left\{
\frac{1}{\sqrt{2}}\begin{pmatrix}1\\0\\1\end{pmatrix},
\begin{pmatrix}0\\1\\0\end{pmatrix},
\frac{1}{\sqrt{2}} \begin{pmatrix}1\\0\\-1\end{pmatrix}
\right\}
\end{equation}
\begin{gather}
\implies Q = \frac{1}{\sqrt{2}}\begin{pmatrix}
1 & 0 & 1\\
0 & \sqrt{2} & 0\\
1 & 0 & -1
\end{pmatrix}\\
Q^tAQ = \psi_\beta(H)=\frac{1}{2}
\begin{pmatrix}
4 & 0 & 0\\
0 & 6 & 0\\
0 & 0 & 8
\end{pmatrix}
=
\begin{pmatrix}
2 & 0 & 0\\
0 & 3 & 0\\
0 & 0 & 4
\end{pmatrix}
\end{gather}
\end{enumerate}
